%%%%%%%%%%%%%%%%%%%%%%%%%%%%%%%%%%%%%%%%%%%%%%%%%%
% Test of the LaTeX overarrows package
% Examples
%%%%%%%%%%%%%%%%%%%%%%%%%%%%%%%%%%%%%%%%%%%%%%%%%%

\documentclass{article}
\title{Package \textsf{overarrows} examples}
\author{Julien Labb\'e}

\usepackage[old-arrows, esvect, tikz, pstricks, pstarrows, debug]{overarrows}
\usepackage{overarrows-doc}
\usepackage{amssymb}
\usepackage{tabularx}

\DeclareMathSymbol{\fldra}{\mathrel}{esvector}{'021}
\DeclareMathSymbol{\fldrb}{\mathrel}{esvector}{'022}
\DeclareMathSymbol{\fldrc}{\mathrel}{esvector}{'023}
\DeclareMathSymbol{\fldrd}{\mathrel}{esvector}{'024}
\DeclareMathSymbol{\fldre}{\mathrel}{esvector}{'025}
\DeclareMathSymbol{\fldrf}{\mathrel}{esvector}{'026}
\DeclareMathSymbol{\fldrg}{\mathrel}{esvector}{'027}
\DeclareMathSymbol{\fldrh}{\mathrel}{esvector}{'030}

\begin{document}

\maketitle

\tableofcontents

\section{Default arrows}

\NewOverArrowCommand{oldamsvec}{
  amsmath, end=\varrightarrow, trim middle={2.5}, shift left={2},
}
\NewOverArrowCommand{amsvec}{
    amsmath, end=\rightarrow, trim middle={2.5}, shift left={2},
}
\NewOverArrowCommand{esvec}{esvect}
\NewOverArrowCommand[tikz]{tikzvec}{}
\NewOverArrowCommand[pstricks]{pstricksvec}{}
\NewOverArrowCommand[picture]{picvec}{}

\begin{center}
  \begin{tabularx}{\linewidth}{ *{8}{ >{\centering\arraybackslash}X }}
    \scriptsize \texttt{\textbackslash vec} &
    \scriptsize ams                         & \scriptsize old-arrows & \scriptsize esvect  &
    \scriptsize tikz                        & \scriptsize pstricks   & \scriptsize picture & \scriptsize pstarrows \\[1em]
    $\displaystyle \vec{v}$                 &
    $\displaystyle \amsvec{v}$              &
    $\displaystyle \oldamsvec{v}$           &
    $\displaystyle \esvec{v}$               &
    $\displaystyle  \tikzvec{v}$            &
    $\displaystyle \pstricksvec{v}$         &
    \ltxarrows $\displaystyle \picvec{v}$   &
    \pstarrows $\displaystyle \picvec{v}$
  \end{tabularx}
\end{center}

\section{Package \textsf{esvect} arrows}

\NewOverArrowCommand{esveca}{esvect, end={\fldra}}
\NewOverArrowCommand{esvecb}{esvect, end={\fldrb}}
\NewOverArrowCommand{esvecc}{esvect, end={\fldrc}}
\NewOverArrowCommand{esvecd}{esvect, end={\fldrd}}
\NewOverArrowCommand{esvece}{esvect, end={\fldre}}
\NewOverArrowCommand{esvecf}{esvect, end={\fldrf}}
\NewOverArrowCommand{esvecg}{esvect, end={\fldrg}}
\NewOverArrowCommand{esvech}{esvect, end={\fldrh}}

\begin{center}
  \begin{tabularx}{\linewidth}{ *{8}{ >{\centering\arraybackslash}X }}
    esveca & esvecb & esvecc & esvecd & esvece & esvecf & esvecg & esvech \\[1em]
    $\displaystyle \esveca{v}$
    & $\displaystyle \esvecb{v}$
    & $\displaystyle \esvecc{v}$
    & $\displaystyle \esvecd{v}$
    & $\displaystyle \esvece{v}$
    & $\displaystyle \esvecf{v}$
    & $\displaystyle \esvecg{v}$
    & $\displaystyle \esvech{v}$
  \end{tabularx}
\end{center}

\section{Arrow with an hook and two heads}

\begin{dispExample}
\NewOverArrowCommand{overhooktwoheadrightarrow}{%
  start=\lhook,
  end=\twoheadrightarrow,% from amssymb package
  trim start=1.5,
  trim end=2,
}
\TestOverArrow{overhooktwoheadrightarrow}
\end{dispExample}

\section{Left-right arrow}

Note: \texttt{\textbackslash overleftrightarrow} is already defined by
\textsf{amsmath} package.

\begin{dispExample}
\RenewOverArrowCommand{overleftrightarrow}{%
  start=\varleftarrow,% from old-arrows package
  end=\varrightarrow,% from old-arrows package
  trim start=7,
  trim end=7,
  center arrow,
}
\TestOverArrow{overleftrightarrow}
\end{dispExample}

\section{Arrow with a double line}

\begin{dispExample}
\NewOverArrowCommand{OverRightarrow}{%
  start={\smallermathstyle\Relbar},
  middle={\smallermathstyle\Relbar},
  end={\Rightarrow},
  trim=4,
  space after arrow=.25ex,
}
\TestOverArrow{OverRightarrow}
\end{dispExample}

\section{Right and left harpoons}

\begin{dispExample}
\NewOverArrowCommand{overrightharpoon}{%
  end=\rightharpoonup,
}
\TestOverArrow{overrightharpoon}
\end{dispExample}

\begin{dispExample}
\NewOverArrowCommand{overleftharpoon}{%
  start=\leftharpoonup,
  end=\relbar,
  shift left=0,
  shift right=2,
}
\TestOverArrow{overleftharpoon}
\end{dispExample}

\section{Double arrow, right and left inverted}

\begin{dispExample}
\NewOverArrowCommand{overrightleftarrow}{%
  start=\text{\usefont{U}{lasy}{m}{n}\symbol{41}},
  end=\text{\usefont{U}{lasy}{m}{n}\symbol{40}},
  trim start=0.7,
  trim end=0.7,
  min length=20,
  shift leftright=-2,
}
\TestOverArrow{overrightleftarrow}
\end{dispExample}

\section{Arrow with a triple tail}

\begin{dispExample}
\NewOverArrowCommand{overtttailrightarrow}{%
  start={\vcenter{\hbox{%
        $\smallermathstyle\succ\xjoinrel[10]\succ\xjoinrel[10]\succ$
      }}},
  middle={\relbareda},% from esvect package
  end={\rightarrow},
  trim start=12,
  center arrow,
  min length=24,
  space after arrow=.15ex,
}
\TestOverArrow{overtttailrightarrow}
\end{dispExample}

\section{Arched left and right arrow, with \textsf{tikz} package}

\begin{dispExample}
\NewOverArrowCommand[tikz]{overarchedleftrightarrow}{%
  arrows={<[scale=0.5]->[scale=0.5]},
  add tikz options={y=\overarrowlength},
  path={(0,0) arc (-250:70:0.5 and 0.1)},
  add path options={line width={\overarrowsmallerthickness}},
  center arrow,
  min length=25,
  space after arrow=0.4ex,
}
\TestOverArrow{overarchedleftrightarrow}
\end{dispExample}

\section{Loop and arrow, with \textsf{PSTricks}}

\begin{dispExample}
\NewOverArrowCommand[pstricks]{overloopandarrow}{
  pstricks command={%
    \pscurve{->}(0,0)
    (0.6\overarrowlength,0.05\overarrowlength)
    (0.5\overarrowlength,0.1\overarrowlength)
    (0.4\overarrowlength,0.05\overarrowlength)
    (\overarrowlength,0\overarrowlength)
  },
  geometry={(0,0)(\overarrowlength,0.2\overarrowlength)},
  space after arrow=2pt,
}
\TestOverArrow{overloopandarrow}
\end{dispExample}

\section{Banded arrow, with \texttt{picture} environment}

\begin{dispExample}
\NewOverArrowCommand[picture]{overbandedarrow}{
  picture command={%
    \qbezier
    (0.0\overarrowlength,0)
    (0.5\overarrowlength,0)
    (0.9\overarrowlength,0.2\overarrowlength)
    \put(0.9\overarrowlength,0.2\overarrowlength)
    {\vector(2,1){0.2\overarrowlength}}
  },
  geometry={(\overarrowlength,0.4\overarrowlength)(0,0)},
  center arrow,
  space after arrow=0.4ex,
}
\TestOverArrow{overbandedarrow}
\end{dispExample}

\section{Under arrow}

Note: \texttt{\textbackslash underleftrightarrow} is already defined by
\textsf{amsmath} package.

\begin{dispExample}
  \DeclareOverArrowCommand{underleftrightarrow}{%
    start=\varleftarrow,
    center arrow,
    arrow under,
  }
  \TestOverArrow{underleftrightarrow}
\end{dispExample}

\end{document}
