%%%%%%%%%%%%%%%%%%%%%%%%%%%%%%%%%%%%%%%%%%%%%%%%%%
% Test of the LaTeX overarrows package
% Loading package with all its options
%%%%%%%%%%%%%%%%%%%%%%%%%%%%%%%%%%%%%%%%%%%%%%%%%%
\documentclass{article}


\title{Test of the \textsf{overarrows} package\\with all options}
\author{Julien Labb\'e}

\usepackage[%
  old-arrows, esvecth,
  tikz, pstricks, pstarrows,
  subscripts, allcommands, debug
]{overarrows}
\usepackage{overarrows-doc}

\begin{document}
\maketitle

\section{Loading the package with many options}

\begin{dispListing}
\usepackage[%
  old-arrows, esvecth,
  tikz, pstricks, pstarrows,
  subscripts, allcommands, debug
]{overarrows}
\end{dispListing}

\section{Options \texttt{old-arrows} and \texttt{allcommands}}

\begin{dispExample}
\TestOverArrow*{\overrightarrow}
\end{dispExample}

\begin{dispExample}
\NewOverArrowCommand{\amsvec}{amsmath, end={\rightarrow}, shift left=2}
\TestOverArrow*{\amsvec}
\end{dispExample}

\section{Option \texttt{esvech}}

\begin{dispExample}
\NewOverArrowCommand{\esvec}{esvect}
\TestOverArrow*{\esvec}
\end{dispExample}

\section{Option \texttt{tikz}}

\begin{dispExample}
\NewOverArrowCommand[tikz]{\tikzvec}{}
\TestOverArrow*{\tikzvec}
\end{dispExample}

\begin{dispExample}
\NewOverArrowCommand[tikz]{\thinnertikzvec}{%
  thinner,
}
\NewOverArrowCommand[tikz]{\thickertikzvec}{%
  line thickness={2\overarrowthickness},
}

$$ \thinnertikzvec{v} \qquad \tikzvec{v} \qquad \thickertikzvec{v} $$
\end{dispExample}

\begin{dispExample}
\NewOverArrowCommand[tikz]{\tikzoptvec}{%
  tikz options={line width=2\overarrowthickness},
  path options={arrows={<->}},
  path={(0,0)--(0.5,0.05)},
}
\TestOverArrow*{\tikzoptvec}
\end{dispExample}

\begin{dispExample}
\NewOverArrowCommand[tikz]{\tikzaddoptvec}{%
  add tikz options={blue},
  add path options={thick},
  arrows={->>}, min length=20,
}
\TestOverArrow*{\tikzaddoptvec}
\end{dispExample}

\begin{dispExample}
\NewOverArrowCommand[tikz]{\tikcodevec}{%
  tikz command={\draw[Circle-Circle](0,0).. controls (0.5,0.3).. (1,0);},
  shift left=-2, space after arrow=0.2ex,
}
\TestOverArrow*{\tikcodevec}
\end{dispExample}


\section{Option \texttt{pstricks}}

\begin{dispExample}
\NewOverArrowCommand[pstricks]{\pstvec}{}
\TestOverArrow*{\pstvec}
\end{dispExample}

\begin{dispExample}
\NewOverArrowCommand[pstricks]{\pstTbarandarrows}{arrow={|<->|}, center arrow}
\TestOverArrow*{\pstTbarandarrows}
\end{dispExample}

\begin{dispExample}
\NewOverArrowCommand[pstricks]{\pstsmallbluearrow}{
  psset = {arrowscale=0.5, arrowinset=0, linecolor=blue},
  thinner,
}
\TestOverArrow*{\pstsmallbluearrow}
\end{dispExample}

\begin{dispExample}
\NewOverArrowCommand[pstricks]{\psellipticarrow}{
  pstricks command={%
    \psellipticarcn{->}%
    (0.5\overarrowlength,0.2\overarrowlength)%
    (0.5\overarrowlength,0.2\overarrowlength)%
    {170}{10}
  },
  geometry={(0,0.2\overarrowlength)(\overarrowlength,0.4\overarrowlength)},
  center arrow,
}
\TestOverArrow*{\psellipticarrow}
\end{dispExample}

\section{Option \texttt{pstarrows}}

\begin{dispExample}
\NewOverArrowCommand[picture]{\picvec}{}
\TestOverArrow*{\picvec}
\end{dispExample}

\section{Option \texttt{subscripts}}

\begin{dispExample}
\NewOverArrowCommand{\subvec}{min length=30}
\NewOverArrowCommand{\nosubvec}{min length=30, detect subscripts=false}
\TestOverArrow*{\subvec}
\TestOverArrow*{\nosubvec}
\end{dispExample}

\end{document}
