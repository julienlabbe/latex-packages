% aspectratio=3218 donne du 16:9 sans changer la hauteur
\documentclass[10pt, aspectratio=3218]{beamer}
% \documentclass[handout, 10pt, aspectratio=3218]{beamer} % remove all overlay

%%%%%%%%%%%%%%%%%%%%%%%%%%%%%%%%%%%%%%%%%%%%%%%%%%%%%%%%%%%%%%%%%%%%%%%%%%%%%%%%
%%% packages

\usepackage[no-math]{fontspec}
\usepackage{unicode-math}

\usepackage[french]{babel}
\frenchsetup{ThinColonSpace=true}

\usepackage{amsmath}
\usepackage{amssymb}% for \twoheadrightarrow
\usepackage{metalogo}
\usepackage{ragged2e}

\usepackage{tikz}
\usetikzlibrary{tikzmark, calc}

\usepackage{etoolbox}
% \usepackage{lua-visual-debug}

%%%%%%%%%%%%%%%%%%%%%%%%%%%%%%%%%%%%%%%%%%%%%%%%%%%%%%%%%%%%%%%%%%%%%%%%%%%%%%%%
%%% Configuration beamer

\title{Le package overarrows}
\subtitle{des flèches personnalisées au-dessus -- ou au-dessous -- d'expressions mathématiques}
\author{Julien Labbé}
\institute[UGA]{Université Grenoble-Alpes}
\date{Exposé mensuel de l'association GUTenberg\linebreak 6 juin 2024}

\mode<presentation> {

  %% theme
  \usetheme[right, width=20mm]{Goettingen}

  %% margins
  \setbeamersize{text margin left=10mm,text margin right=10mm}

  %% frame title
  \setbeamertemplate{frametitle}
  {%
    \ifbeamercolorempty[bg]{frametitle}{}{\nointerlineskip}%
    \begin{beamercolorbox}
      [wd=\dimeval{\paperwidth-20mm},sep=3pt]
      {frametitle}
      \hspace*{1ex}%
      \usebeamerfont{frametitle}%
      \strut\insertframetitle\strut%
      \ifcsempty{insertframesubtitle}{}
      {%
        ~/~%
        \usebeamerfont{framesubtitle}\usebeamercolor[fg]{framesubtitle}%
        \strut\insertframesubtitle\strut%
      }\par
    \end{beamercolorbox}%
  }
  \setbeamerfont{framesubtitle}{size=\large}
  \setbeamercolor{framesubtitle}{use={frametitle}, fg=frametitle.fg!50!red}

  %% numéros page
  %% source : https://tex.stackexchange.com/a/137028/316068
  \addtobeamertemplate{navigation symbols}{}{%
    \usebeamerfont{footline}%
    \usebeamercolor[fg]{footline}%
    \hspace{1em}%
    \insertframenumber/\inserttotalframenumber
  }

  %% colors
  \usecolortheme{spruce}
  \usecolortheme{rose}      % inner theme (blocks)
  % \setbeamercolor{itemize subitem}{use={palette quaternary}, fg=palette quaternary.fg}
  \setbeamercolor{itemize subitem}{use={itemize item}, fg=white!70!itemize item.fg}
  \setbeamercolor{itemize subsubitem}{use={itemize subitem}, fg=white!70!itemize subitem.fg}

  %% fonts
  \usefonttheme[onlymath]{serif}

  %% overlays
  \setbeamercovered{transparent=65}

  %% lists
  \settowidth{\leftmargini}{\usebeamertemplate{itemize item}}
  \addtolength{\leftmargini}{\labelsep}
  \setlength{\leftmarginii}{\leftmargini}
  \setlength{\leftmarginiii}{\leftmarginii}

  %% blocks
  \addtobeamertemplate{block begin}{}{\leftskip=0pt \rightskip=0pt plus 0cm}
  \addtobeamertemplate{block example begin}{}{\leftskip=0pt \rightskip=0pt plus 0cm}
  \addtobeamertemplate{block alerted begin}{}{\leftskip=0pt \rightskip=0pt plus 0cm}
  \addtobeamertemplate{block begin}{}{\vspace*{-\smallskipamount}}
  \addtobeamertemplate{block end}{\vspace*{-\smallskipamount}}{}
  \addtobeamertemplate{block example begin}{}{\vspace*{-\smallskipamount}}
  \addtobeamertemplate{block example end}{\vspace*{-\smallskipamount}}{}
  \addtobeamertemplate{block alerted begin}{}{\vspace*{-\smallskipamount}}
  \addtobeamertemplate{block alerted end}{\vspace*{-\smallskipamount}}{}
}

% \AtBeginSubsection[] {
%   \begin{frame}<beamer>{Sommaire}
%     \tableofcontents[currentsection, currentsubsection]
%   \end{frame}
% }

\makeatletter \NewDocumentCommand{\shortinlinetoc}{ } { \begingroup
  \long\def\beamer@sectionintoc##1##2##3##4##5 % taken from beamerbasetoc.sty
  {%
    \usebeamerfont{section in toc}\usebeamercolor[fg]{section in toc}%
    \ifnum ##1 > 1\relax \hfill \fi%
    \hyperlink{Navigation##3}{$\triangleright$~##2}%
  }%
  \def\inserttocsection{}%
  \par\tableofcontents[hideallsubsections]%
  \vspace*{-3cm}%
  \endgroup
}
\makeatother

%%%%%%%%%%%%%%%%%%%%%%%%%%%%%%%%%%%%%%%%%%%%%%%%%%%%%%%%%%%%%%%%%%%%%%%%%%%%%%%%
%%% Formattage

\RequirePackage[many]{tcolorbox}
\tcbuselibrary{listings}
\tcbuselibrary{documentation}

\definecolor{ovar_lavender}{rgb}{0.92,0.92,1}
\definecolor{ovar_darkblue}{rgb}{0.1,0.2,0.5}
\definecolor{ovar_darkgreen}{rgb}{0,0.39,0}
\definecolor{ovar_beige}{rgb}{.96,.96,.86}
\colorlet{ovar_commands}{ovar_darkblue}
\colorlet{ovar_keys}{ovar_darkgreen}
\colorlet{ovar_lengths}{violet}
\colorlet{ovar_options}{Definition} % from tcolorbox documentation

\lstdefinestyle{lstovardoc}{%
  language        = [LaTeX]TeX,
  columns         = flexible,
  keepspaces      = true,
  keywordstyle    = {\bfseries\color{ovar_darkblue}},
  texcsstyle      = *{\bfseries\color{ovar_darkblue}},
  commentstyle    = {\color{gray}},
  identifierstyle = {\color{ovar_darkgreen}},
  morekeywords    = {},% keywordsprefix needs a morekeywords before
  keywordsprefix  = {\\},
  literate        = *{\$}{{\textcolor{red}{\$}}}{1}
                    {\&}{{\textcolor{red}{\&}}}{1}
                    {\}}{{\textcolor{darkgray}{\}}}}{1}
                    {\{}{{\textcolor{darkgray}{\{}}}{1}
                    {\\\\}{{\textcolor{red}{\textbackslash\textbackslash}}}{2}
                    {\%\ \ \ \ }{}0,
  basicstyle      = \ttfamily\footnotesize,
  frame           = none,
  framesep        = 0pt,
  aboveskip       = 0pt,
  belowskip       = 0pt,
}

\tcbset{% copied and adapted from tcolorbox.doc.s_main.sty
  documentation listing style=lstovardoc,%
  doc keypath=overarrows,
  size=small,%
  % verbatim ignore percent,
  %% styles
  docexample/.style={%
    bicolor jigsaw,
    beforeafter skip balanced={2pt plus 6pt},
    fonttitle=\bfseries,
    fontlower=\normalfont,
    halign lower=center,
    colframe=ovar_darkblue,
    colback=ovar_lavender,
    colbacklower=white,
    drop fuzzy shadow,
    left=0mm
  },
  color key=ovar_keys,
  color command=ovar_commands,
  color length=ovar_lengths,
  doc head key={fontlower=\footnotesize, collower=darkgray},
  before doc body={\parskip=\smallskipamount},
}

\DeclareTotalTCBox{\code}{ v }
  {
    verbatim,
    colframe=ovar_lavender,
    colback=ovar_lavender,
    sharp corners,
    size=tight,
    left=2pt,
    right=2pt,
  }
  {%
    \strut\lstinline[basicstyle=\ttfamily\small\color{ovar_darkblue}]!#1!%
  }

\NewCommandCopy{\oldcs}{\cs}
\RenewDocumentCommand{\cs}{ m }{{\small\oldcs{#1}}}

\newcommand*{\pkg}[1]{\textcolor{violet}{\textsf{#1}}}
\newcommand*{\opt}[1]{\textcolor{Definition}{\texttt{\small #1}}}
\newcommand*{\key}[1]{\textcolor{Option}{\texttt{\small #1}}}
\newcommand*{\symb}[1]{\colorbox{yellow!50}{\rule{0pt}{1ex}$#1$}}
\newcommand{\rmq}[1]{{\small #1}}

%%%%%%%%%%%%%%%%%%%%%%%%%%%%%%%%%%%%%%%%%%%%%%%%%%%%%%%%%%%%%%%%%%%%%%%%%%%%%%%%
%%% Arrows

\usepackage[old-arrows, esvect, tikz, pstarrows, allcommands]{overarrows}

\makeatletter
\AtBeginDocument{%
  %% sauvegarder la définition de \overrightarrow de unicode-math
  \NewCommandCopy{\umoverrightarrow}{\overrightarrow}
  %% redéfinir toutes les commandes \AtBeginDocument pour éviter les interférences
  %% de unicode-math
  \ifovar@option@overrightarrow@
    \DeclareOverArrowCommand{overrightarrow}{%
      amsmath, middle config=relbar,
      end=\ovar@rightarrow,
      right arrow,
    }
  \fi
  \ifovar@option@underrightarrow@
    \DeclareOverArrowCommand{underrightarrow}{%
      amsmath, middle config=relbar,
      end=\ovar@rightarrow,
      right arrow,
      arrow under,
    }
  \fi
  \ifovar@option@overleftarrow@
    \DeclareOverArrowCommand{overleftarrow}{%
      amsmath, middle config=relbar,
      start=\ovar@leftarrow,
      end=\relbar,
      left arrow,
    }
  \fi
  \ifovar@option@underleftarrow@
    \DeclareOverArrowCommand{underleftarrow}{%
      amsmath, middle config=relbar,
      start=\ovar@leftarrow,
      end=\relbar,
      left arrow,
      arrow under,
    }
  \fi
  \ifovar@option@overleftrightarrow@
    \DeclareOverArrowCommand{overleftrightarrow}{%
      amsmath, middle config=relbar,
      start=\ovar@leftarrow,
      end=\ovar@rightarrow,
      center arrow,
    }
  \fi
  \ifovar@option@underleftrightarrow@
    \DeclareOverArrowCommand{underleftrightarrow}{%
      amsmath, middle config=relbar,
      start=\ovar@leftarrow,
      end=\ovar@rightarrow,
      center arrow,
      arrow under,
    }
  \fi
  \ifovar@option@overrightharpoonup@
    \DeclareOverArrowCommand{overrightharpoonup}{%
      amsmath, middle config=relbar,
      end=\rightharpoonup,
      right arrow,
    }
  \fi
  \ifovar@option@underrightharpoonup@
    \DeclareOverArrowCommand{underrightharpoonup}{%
      amsmath, middle config=relbar,
      end=\rightharpoonup,
      right arrow,
      arrow under,
    }
  \fi
  \ifovar@option@overrightharpoondown@
    \DeclareOverArrowCommand{overrightharpoondown}{%
      amsmath, middle config=relbar,
      end=\rightharpoondown,
      right arrow,
    }
  \fi
  \ifovar@option@underrightharpoondown@
    \DeclareOverArrowCommand{underrightharpoondown}{%
      amsmath, middle config=relbar,
      end=\rightharpoondown,
      right arrow,
      arrow under,
    }
  \fi
  \ifovar@option@overleftharpoonup@
    \DeclareOverArrowCommand{overleftharpoonup}{%
      amsmath, middle config=relbar,
      start=\leftharpoonup,
      end=\relbar,
      left arrow,
    }
  \fi
  \ifovar@option@underleftharpoonup@
    \DeclareOverArrowCommand{underleftharpoonup}{%
      amsmath, middle config=relbar,
      start=\leftharpoonup,
      end=\relbar,
      left arrow,
      arrow under,
    }
  \fi
  \ifovar@option@overleftharpoondown@
    \DeclareOverArrowCommand{overleftharpoondown}{%
      amsmath, middle config=relbar,
      start=\leftharpoondown,
      end=\relbar,
      left arrow,
    }
  \fi
  \ifovar@option@underleftharpoondown@
    \DeclareOverArrowCommand{underleftharpoondown}{%
      amsmath, middle config=relbar,
      start=\leftharpoondown,
      end=\relbar,
      left arrow,
      arrow under,
    }
  \fi
  \ifovar@option@overbar@
    \DeclareOverArrowCommand{overbar}{%
      amsmath, middle config=relbar,
      start={\std@minus}, end={\std@minus},% \relbar is defined with \mathsm@sh
      shift leftright=0,
      space after arrow=-0.3ex,
    }
  \fi
  \ifovar@option@underbar@
    %%
    %% \underbar ne fonctionne pas avec unicode-math. Ajouter \vphatom{+} pour
    %% obtenir la position désirée
    %%
    \DeclareOverArrowCommand{underbar}{%
      amsmath, middle config=relbar,
      start={\vphantom{+}-}, end={-},% \relbar is defined with \mathsm@sh
      shift leftright=0,
      arrow under,
      space before arrow=-0.3ex,
    }
  \fi
}
\makeatother

\NewOverArrowCommand{amsoverrightarrow}{amsmath=strict, end={\rightarrow}}
\NewOverArrowCommand{overrightnewarrow}{%
    amsmath, middle config=relbar,  end=\rightarrow,
    right arrow,
  }
\DeclareMathSymbol{\fldra}{\mathrel}{esvector}{'021}
\DeclareMathSymbol{\fldrb}{\mathrel}{esvector}{'022}
\DeclareMathSymbol{\fldrc}{\mathrel}{esvector}{'023}
\DeclareMathSymbol{\fldrd}{\mathrel}{esvector}{'024}
\DeclareMathSymbol{\fldre}{\mathrel}{esvector}{'025}
\DeclareMathSymbol{\fldrf}{\mathrel}{esvector}{'026}
\DeclareMathSymbol{\fldrg}{\mathrel}{esvector}{'027}
\DeclareMathSymbol{\fldrh}{\mathrel}{esvector}{'030}
\NewOverArrowCommand{esveca}{esvect=strict, end={\fldra}}
\NewOverArrowCommand{esvecb}{esvect=strict, end={\fldrb}}
\NewOverArrowCommand{esvecc}{esvect=strict, end={\fldrc}}
\NewOverArrowCommand{esvecd}{esvect=strict, end={\fldrd}}
\NewOverArrowCommand{esvece}{esvect=strict, end={\fldre}}
\NewOverArrowCommand{esvecg}{esvect=strict, end={\fldrg}}
\NewOverArrowCommand{esvech}{esvect=strict, end={\fldrh}}
\NewOverArrowCommand{vva}{esvect, middle config=auto, end=\fldra}
\NewOverArrowCommand{esvecf}{esvect=strict, end={\fldrf}}
\NewOverArrowCommand{vvb}{esvect, middle config=auto, end=\fldrb}
\NewOverArrowCommand{vvc}{esvect, middle config=auto, end=\fldrc}
\NewOverArrowCommand{vvd}{esvect, middle config=auto, end=\fldrd}
\NewOverArrowCommand{vve}{esvect, middle config=auto, end=\fldre}
\NewOverArrowCommand{vvf}{esvect, middle config=auto, end=\fldrf}
\NewOverArrowCommand{vvg}{esvect, middle config=auto, end=\fldrg}
\NewOverArrowCommand{vvh}{esvect, middle config=auto, end=\fldrh}

\makeatletter
\newcommand{\unsetoldarrows}{%
  \let\ovar@rightarrow\rightarrow%
  \let\ovar@leftarrow\leftarrow%
}
\newcommand{\setoldarrows}{%
  \let\ovar@rightarrow\varrightarrow%
  \let\ovar@leftarrow\varleftarrow%
}
\unsetoldarrows
\makeatother

%% patch \ovar@set@arrowthickness pour fonctionner avec unicode-math
\makeatletter
\def\ovar@set@arrowthickness@lua#1{% use rule thickness=\fontdimen 8 font family 3
  \overarrowthickness = \Umathoverbarrule #1
  \ifx#1\displaystyle%
    \overarrowsmallerthickness = \Umathoverbarrule \textstyle%
  \else\ifx#1\textstyle%
    \overarrowsmallerthickness = \Umathoverbarrule \scriptstyle%
  \else%
    \overarrowsmallerthickness = \Umathoverbarrule \scriptscriptstyle%
  \fi\fi%
}
\let\ovar@set@arrowthickness\ovar@set@arrowthickness@lua
\makeatother

%%%%%%%%%%%%%%%%%%%%%%%%%%%%%%%%%%%%%%%%%%%%%%%%%%%%%%%%%%%%%%%%%%%%%%%%%%%%%%%%
%%% Début du document
\begin{document}

\begingroup
  \setbeamertemplate{sidebar canvas right}{}
  \setbeamertemplate{sidebar right}{}
  \begin{frame}
    \hspace*{.5\beamersidebarwidth}
    \begin{minipage}{\textwidth}
      \maketitle

      \begin{center}
        \rule{6cm}{0.4pt}

        \url{https://ctan.org/pkg/overarrows}
      \end{center}
      \vspace*{6pt}%

      \setbeamercolor{section in toc}{use=subtitle, fg=subtitle.fg}
      \setbeamerfont{section in toc}{size=\footnotesize}
      \shortinlinetoc
    \end{minipage}
  \end{frame}
\endgroup

% \begin{frame}
%   \frametitle{Sommaire}
%   \tableofcontents
% \end{frame}


\section{Introduction}

\subsection{Motivations}

\begin{frame}[fragile]
  \frametitle{Introduction}
  \framesubtitle{Motivations}

  \begin{block}<+>{Enseigner la mécanique du point en Licence}
    \begin{itemize}
     \item Représenter un vecteur avec une \structure{flèche extensible}.

       \begin{itemize}
        \item \alert{Exemples :} vecteur position $\vv{r} = \vv{OM}$, vecteur
          moment cinétique $\vv*{L}_O = \vv{OM} \wedge \vv*{v}$.
       \end{itemize}

     \item Adapter la flèche aux quatre \structure{styles mathématiques} :

      \begin{center}
        \cs{displaystyle}, \cs{textstyle}, \cs{scriptstyle},
        \cs{scriptscriptstyle}.
      \end{center}

     \item Corriger \structure{l'espacement des indices} (\rmq{position
       identique avec ou sans flèche}).

     \begin{itemize}
      \item \alert{Exemple,} sans correction :
        $\vv{v\tikzmark{v1}}_{\tikzmark{c1}C}$ ;
      \begin{tikzpicture}[remember picture, overlay]
        \coordinate (A) at ([yshift=-2pt]pic cs:v1);
        \coordinate (B) at ([yshift=-2pt, xshift=\scriptspace]pic cs:c1);
        \draw[thick, red, >={Bar[width=3pt]}, <->, shorten >=-0.8pt, shorten <=-0.8pt]
        (A |- B) -- (B);
      \end{tikzpicture}
      avec correction : $\vv*{v\tikzmark{v2}}_{\tikzmark{c2}C}$.
      \begin{tikzpicture}[remember picture, overlay]
        \coordinate (A) at ([yshift=-2pt]pic cs:v2);
        \coordinate (B) at ([yshift=-2pt, xshift=\scriptspace]pic cs:c2);
        \draw[thick, red, >={Bar[width=3pt]}, <->, shorten >=-0.8pt, shorten <=-0.8pt]
        (A |- B) -- (B);
      \end{tikzpicture}
     \end{itemize}

    \end{itemize}

  \end{block}

  \begin{exampleblock}<+>{Créer un nouveau package \LaTeX}
    \begin{itemize}
     \item Créer des \structure{commandes personnalisées} avec une
       \structure{interface clé-valeur}

       (\rmq{\href{https://ctan.org/pkg/pgfkeys}{package \pkg{pgfkeys}} ; lors
       de la création uniquement}).

     \item Permettre d'utiliser \structure{plusieurs méthodes} (\rmq{assemblage
       de symboles ; PGF/TikZ})

       et de positionner la \structure{flèche au-dessous}.

      \item Fournir des \structure{commandes pré-définies} :

       \begin{center}
       \small\setoldarrows
         $ \overrightarrow{AB}/\underrightarrow{AB}, $
         $ \overleftarrow{AB}/\underleftarrow{AB}, $
         $ \overleftrightarrow{AB}/\underleftrightarrow{AB}, $
         $ \overrightharpoonup{AB}/\underrightharpoonup{AB}, $
         $ \overleftharpoonup{AB}/\underleftharpoonup{AB}, $
         $ \overrightharpoondown{AB}/\underrightharpoondown{AB}, $
         $ \overleftharpoondown{AB}/\underleftharpoondown{AB}, $
         $ \overbar{AB}/\underbar{AB} $%
       .
       \end{center}
     \end{itemize}
   \end{exampleblock}

\end{frame}

\subsection{Alternatives}

\begin{frame}[fragile]
  \frametitle{Introduction}
  \framesubtitle{Alternatives}

  \begin{itemize}

   \item<+,4> \structure{Accent mathématique} avec \cs{vec} : $\vec{v}$, $\vec{AB}$,
    $\vec{\mathrm{grad}}$ ;
    \begin{itemize}
     \item \rmq{pas de commande \cs{widevec} analogue à \cs{widehat} ($\widehat{v}$,
       $\widehat{\!AB\!}$, $\widehat{\mathrm{grad}}$)}.
    \end{itemize}

   \item<+> Commande \structure{\cs{overightarrow}} : {\unsetoldarrows
     $\overrightarrow{AB}$}, tête de flèche trop large (\rmq{avec \emph{Computer
     Modern}}) ;
    \begin{itemize}
     \item \rmq{mieux avec le \href{https://ctan.org/pkg/old-arrows}{package
       \pkg{old-arrows}} de Riccardo Dossena : {\setoldarrows
       $\overrightarrow{AB}$}}.
    \end{itemize}

   \item<+> Commande \cs{vv} (\structure{\href{https://ctan.org/pkg/esvect}{package
     \pkg{esvect}}} d'Eddie Saudrais) : $\esvectvv{v}$, $\esvectvv{AB}$
     $\esvectvv{\mathrm{grad}}$ ;
    \begin{itemize}
     \item \rmq{commande \cs{vv*} pour les indices :
       $\esvectvv*{v\tikzmark{v3}}{\tikzmark{c3}C}$}.
    \begin{tikzpicture}[remember picture, overlay]
      \coordinate (A) at ([yshift=-2pt]pic cs:v3);
      \coordinate (B) at ([yshift=-2pt, xshift=\scriptspace]pic cs:c3);
      \draw[thick, red, >={Bar[width=3pt]}, <->, shorten >=-0.8pt, shorten <=-0.8pt]
      (A |- B) -- (B);
    \end{tikzpicture}
    \end{itemize}

  \end{itemize}

  \setbeamercovered{invisible}
  
  \begin{alertblock}<+>{\LuaTeX/\XeTeX{} et \pkg{unicode-math}}%

    Avec \LuaTeX{} et \XeTeX{} : définition de nouveaux accents avec
    \cs{Umathaccent}. Utilisé par
    \href{https://www.ctan.org/pkg/unicode-math}{le package \pkg{unicode-math}}
    pour redéfinir \cs{overrightarrow} de manière cohérente avec \cs{vec}.

    \let\overrightarrow\umoverrightarrow
    \begin{dispExample}
$ \vec{v} \qquad \overrightarrow{ABCD} \qquad
  {\Umathaccent 0 0 "020D7 {VVV}}_{sub} \qquad {VVV}_{sub} $
    \end{dispExample}

  \end{alertblock}

\end{frame}

\section{Un peu de \TeX{}nique}

\subsection{Flèches extensibles}

\begin{frame}[fragile]

\frametitle{Un peu de \TeX{}nique}
\framesubtitle{Flèches extensibles}

\begin{block}<+>{\cs{cleaders} : remplir avec un motif centré}
  \begin{dispExample}
A$ \lhook \cleaders\hbox{$-$}\hskip 3cm \relax \rightarrow $B \par
A$ \lhook \cleaders\hbox{$-$}\hfill     \relax \rightarrow $B
  \end{dispExample}
\end{block}

\begin{block}<+>{\cs{mathrel} et \cs{joinrel} : recoller les morceaux}
  \begin{dispExample}
A$ \mathrel{\lhook} \joinrel
  \cleaders\hbox{$\joinrel\mathrel{-}\joinrel$}\hfill \relax
  \joinrel \mathrel{\rightarrow} $B
  \end{dispExample}
\end{block}

\begin{alertblock}<+>{Nouvelle commande \cs{xjoinrel} : plus flexible}
  \begin{dispExample}
$ \mathrel\succ\joinrel\mathrel\succ\joinrel\mathrel\succ $\qquad
$ \succ\xjoinrel[10]\succ\xjoinrel[10]\succ $
  \end{dispExample}
\end{alertblock}

\end{frame}

\subsection{Empilements}

\begin{frame}[fragile]

\frametitle{Un peu de \TeX{}nique}
\framesubtitle{Empilements}

\begin{block}<+>{\cs{ialign} : empiler verticalement}
  \begin{itemize}
   \item \cs{noalign} : insérer du contenu vertical entre deux lignes ;\par
   \item \cs{nointerlineskip} : pas d'interligne.
  \end{itemize}
  \begin{dispExample}
$ \vbox{\ialign{#\crcr $\rightarrow$ \crcr $v$ \crcr }} $ \qquad
$ \vbox{\ialign{#\crcr $\mskip -1mu \rightarrow$ \crcr
    \noalign{\kern -0.7pt\nointerlineskip} $v$ \crcr}} $
  \end{dispExample}
\end{block}

\begin{block}<+>{\cs{vbox} et \cs{vtop} : choix de la ligne de base}
  \begin{dispExample}
$ \vbox{\ialign{#\crcr $\mskip -1mu \rightarrow$ \crcr
    \noalign{\kern -0.7pt\nointerlineskip} $v$ \crcr}} $ \qquad
$ \vtop{\ialign{#\crcr $v$ \crcr
    \noalign{\kern -0.7pt\nointerlineskip}
    $\mskip -1mu \rightarrow$ \crcr}} $
  \end{dispExample}
\end{block}

\end{frame}

\subsection{Styles mathématiques}

\begin{frame}[fragile]

\frametitle{Un peu de \TeX{}nique}
\framesubtitle{Styles mathématiques}

\begin{block}{\cs{mathpalette} : jongler avec les styles}
  \begin{dispListing}
\def\mathpalette#1#2{% définition donnée par latex.ltx
  \mathchoice {#1\displaystyle{#2}}%
              {#1\textstyle{#2}}%
              {#1\scriptstyle{#2}}%
              {#1\scriptscriptstyle{#2}}}
  \end{dispListing}
\begin{itemize}
 \item Style passé à la macro \texttt{\#1} (\rmq{remarque : les quatre styles
   sont évalués par \TeX{}}).

  \alert{Exemple :}

\end{itemize}
  \begin{dispExample}
\newcommand{\applystyle}[2]{#1 #2 \text{\normalfont\normalsize ~avec \detokenize{#1}}}

\newcommand{\teststyle}{\mathpalette\applystyle{\sum_{i=1}^n x^i}}

$ \displaystyle \teststyle \qquad \textstyle \teststyle $\\[1ex]
$ \scriptstyle \teststyle \qquad \scriptscriptstyle \teststyle $
  \end{dispExample}
\end{block}
\end{frame}

\subsection{Synthèse}

\begin{frame}[fragile]

\frametitle{Un peu de \TeX{}nique}
\framesubtitle{Synthèse}

\begin{itemize}
 \item \structure{Définition de \cs{overrightarrow}} utilisée par le package
   \pkg{amsmath} :

   \tikzset
   {
     block/.style={rectangle, rounded corners, draw=#1, fill=#1!5, inner sep=2pt},
     >={Triangle}
   }

   \colorlet{mathpalettecolor}{red}
   \usebeamercolor{structure}
   \colorlet{stackmacrocolor}{structure.fg}
   \usebeamercolor{frametitle}
   \colorlet{arrowmacrocolor}{frametitle.fg}

   \begin{tcolorbox}
     [
       docexample,
       fontupper=\bfseries\ttfamily\footnotesize,
       before skip=12pt,
       after skip=1.5cm,
       grow to left by=3ex,
       top=3pt, bottom=3pt,
     ]
    \cs{long} macro:->%
    \tikzmarknode[block=mathpalettecolor]{mathpalette}{\cs{mathpalette}}
    \{%
    \tikzmarknode[block=stackmacrocolor]{stack macro}{\cs{overarrow@}}
    \tikzmarknode[block=arrowmacrocolor]{arrow macro}{\cs{rightarrowfill@}}
    \}
  \end{tcolorbox}

  \begin{tikzpicture}[overlay, remember picture]
    \draw[mathpalettecolor, <-, thick] (mathpalette) -- +(0,-8mm)
    node[below, xshift=-2em]{\cs{mathpalette}};

    \draw[stackmacrocolor, <-, thick] (stack macro) -- +(0,-8mm)
    node[below, align=center]
    {macro d'empilement\\ {\footnotesize (\cs{ialign})}};

    \draw[arrowmacrocolor, <-, thick] (arrow macro) -- +(0,-8mm)
    node[below, xshift=2em, align=center]
    {macro de flèche\\ {\footnotesize (\cs{cleaders})}};

  \end{tikzpicture}

 \item Définition analogue pour la commande \cs{vv} du \structure{package
   \pkg{esvect}}.

 \item Peut nécessiter des ajustements.

   \alert{Exemple :} (\rmq{agrandissement ×4})
  {
    \let\overrightarrow\amsoverrightarrow
    \let\vv\esvectvv
  \begin{dispExample}
\scalebox{4}{$\scriptscriptstyle \overrightarrow{vecteur}$}\qquad
\scalebox{4}{$\scriptscriptstyle \vv{vecteur}$}
  \end{dispExample}
  }

  % \begin{itemize}
  %  \item \alert{Remarque :} \pkg{amsmath} utilise \cs{relbar} \symb{\relbar},
  %    \pkg{esvect} \cs{relbareda} \symb{\relbareda}
  % \end{itemize}

\end{itemize}
\end{frame}

\section{Utilisation}

\subsection{Mise en route}

\begin{frame}[fragile]

\frametitle{Utilisation}
\framesubtitle{Mise en route}

\begin{block}<+>{Chargement du package}
  \begin{dispListing}
\usepackage[allcommands, old-arrows, noesvect]{overarrows}
  \end{dispListing}
  \begin{itemize}
   \item \opt{allcommands} pour avoir toutes les commandes pré-définies ;
     \opt{old-arrows} et \opt{noesvect} pour utiliser le package
     \pkg{old-arrows} mais pas le package \pkg{esvect}.
  \end{itemize}
\end{block}

\begin{block}<+>{\cs{NewOverArrowCommand} : créer une commande}
  \begin{itemize}
   \item Définir les commandes \cs{mafleche} et \cs{mafleche*}.
\undef\mafleche% \NewOverArrowCommand doesn't work inside beamer frame
    \begin{dispExample}
\NewOverArrowCommand{mafleche}{end=\rightarrow}
$ \mafleche{test} $
    \end{dispExample}
   \item La commande étoilée \cs{mafleche*} gère les indices
    \begin{dispExample}
$ v_{sub} \qquad \mafleche{v}_{sub} \qquad \mafleche*{v}_{sub} $
    \end{dispExample}
   \item Variantes : \cs{Renew\dots}, \cs{Provide\dots} et \cs{Declare\dots}.
  \end{itemize}
\end{block}

\end{frame}

\subsection{Assemblage}

\begin{frame}[fragile]

\frametitle{Utilisation}
\framesubtitle{Assemblage}

\begin{block}<+>{Clés \key{start}, \key{middle}, \key{end}}
\undef\overhooktwoheadightarrow% \NewOverArrowCommand doesn't work inside beamer frame
  \begin{dispExample}
\NewOverArrowCommand{overhooktwoheadightarrow}{%
  start=\lhook, end=\twoheadrightarrow, middle=\relbar,
}
$\overhooktwoheadightarrow{v} \qquad \overhooktwoheadightarrow{AB}$
 \end{dispExample}

 \smallskip

 \rmq{\alert{Remarque :} \cs{twoheadrightarrow} nécessite le package \pkg{amssymb}}.
\end{block}

\begin{block}<+>{Clés \key{trim start}, \key{trim middle}, \key{trim end}}
\undef\overhooktwoheadightarrow% \NewOverArrowCommand doesn't work inside beamer frame
  \begin{dispExample}
\NewOverArrowCommand{overhooktwoheadightarrow}{%
  start=\lhook, end=\twoheadrightarrow, middle=\relbar,
  trim start=0, trim end=3, trim middle=5,
}
$ \overhooktwoheadightarrow{v} \qquad \overhooktwoheadightarrow{AB} $
 \end{dispExample}
\end{block}

\end{frame}

\begin{frame}[fragile]

\frametitle{Utilisation}
\framesubtitle{Assemblage}

\setlength{\textwidth}{\dimeval{\paperwidth-20mm}}

\begin{block}<+>{\cs{TestOverArrow} : tester l'assemblage}
  \begin{dispExample}
\TestOverArrow{overhooktwoheadightarrow}
 \end{dispExample}
\end{block}

\end{frame}

\subsection{Taille et position}

\begin{frame}[fragile]

\frametitle{Utilisation}
\framesubtitle{Taille et position}

\begin{block}<+>{\cs{smallermathstyle} : une flèche plus petite}
\undef\OverLeftarrow% \NewOverArrowCommand doesn't work inside beamer frame
  \begin{dispExample}
\NewOverArrowCommand{OverLeftarrow}{%
  start={\smallermathstyle\Leftarrow}, end=\Relbar,
  middle={\smallermathstyle\Relbar}, trim=4,
}
$ \Leftarrow\joinrel\Relbar \quad \OverLeftarrow{v} \qquad \OverLeftarrow{AB} $
 \end{dispExample}

 \rmq{\alert{Remarque :} le style utilisé pour \key{end} est le même que pour
   \key{start}}.
\end{block}

\begin{block}<+>{Clés \key{space after arrow}, \key{shift left}, \key{shift right}}
\undef\OverLeftarrow% \NewOverArrowCommand doesn't work inside beamer frame
  \begin{dispExample}
\NewOverArrowCommand{OverLeftarrow}{%
  start={\smallermathstyle\Leftarrow}, end=\Relbar,
  middle={\smallermathstyle\Relbar}, trim=4,
  space after arrow=0.25ex, shift left=-1, shift right=2,
}
$ \OverLeftarrow{v} \qquad \OverLeftarrow{AB} $
 \end{dispExample}
\end{block}

\end{frame}

\begin{frame}[fragile]

\frametitle{Utilisation}
\framesubtitle{Taille et position}

\begin{block}<+>{Clé \key{arrow under} : placer la flèche en dessous}
% \undef\UnderLeftRightarrow% \NewOverArrowCommand doesn't work inside beamer frame
%   \begin{dispExample}
% \NewOverArrowCommand{UnderLeftRightarrow}{%
%   start={\smallermathstyle\Leftarrow},
%   middle={\smallermathstyle\Relbar},
%   end=\Rightarrow,
%   trim=4,
%   arrow under,
%   space before arrow=0.5ex,
%   shift left=0, shift right=0,
% }
% $ \UnderLeftRightarrow{v} \qquad \UnderLeftRightarrow{AB} $
%  \end{dispExample}
\undef\tttail
\undef\undertttailtwoheadrightarrow% \NewOverArrowCommand doesn't work inside beamer frame
  \begin{dispExample}
\newcommand*{\tttail}{\succ\xjoinrel[10]\succ\xjoinrel[10]\succ}

\NewOverArrowCommand{undertttailtwoheadrightarrow}{%
  start={\vcenter{\hbox{$\smallermathstyle\tttail$}}},
  middle=\relbar,
  end=\twoheadrightarrow,
  trim=4,
  arrow under,
  space before arrow=0.5ex,
  shift left=0, shift right=0,
  min length=30
}

$ \undertttailtwoheadrightarrow{v} \qquad \undertttailtwoheadrightarrow{ABCD} $
 \end{dispExample}
\end{block}

\end{frame}

\subsection{TikZ}

\begin{frame}[fragile]

\frametitle{Utilisation}
\framesubtitle{TikZ}

\begin{block}<+>{Choisir la méthode TikZ}
\begin{itemize}
 \item \cs{NewOverArrowCommand} possède un paramètre optionnel :
  \begin{itemize}
   \item choix de la méthode de dessin :
    \begin{itemize}
     \item \texttt{symb} : assemblage de symboles (défaut),
     \item \texttt{\texttt{tikz}} : code PGF/TikZ,
     \item \texttt{picture} : environnement \LaTeX{} \texttt{picture}.
    \end{itemize}

   \item Avec \texttt{\texttt{tikz}} :
   \undef\overtikzarrow% \NewOverArrowCommand doesn't work inside beamer frame
  \begin{dispExample}
\NewOverArrowCommand[tikz]{overtikzarrow}{}
$ \overtikzarrow{v} \qquad \overtikzarrow{AB} $
 \end{dispExample}
\smallskip
\item charger le package \pkg{tikz} avant (ou utiliser l'option \opt{tikz}).
  \end{itemize}
 \item 3 longueurs disponibles, utilisables dans le code PGF/TikZ :
\begin{itemize}
 \item \structure{\cs{overarrowlength} :} \rmq{basé sur la longueur du
   contenu} ;
 \item \structure{\cs{overarrowthickness} :} \rmq{épaisseur de trait par défaut
   du style utilisé} ;
 \item \structure{\cs{overarrowsmallerthickness} :} \rmq{épaisseur du style
   inférieur}.
\end{itemize}

\end{itemize}
\end{block}

\end{frame}

\begin{frame}[fragile]

\frametitle{Utilisation}
\framesubtitle{TikZ}

\begin{block}{Exemples}

% \onslide<+>

\undef\overdotteddoublearrow% \NewOverArrowCommand doesn't work inside beamer frame
  \begin{dispExample}
\NewOverArrowCommand[tikz]{overdotteddoublearrow}{%
  add tikz options={blue}, add path options={densely dotted},
  arrows={->[scale=0.5]>[scale=0.5]}, thinner,
  min length=20, space after arrow={0.3ex},
}
$ \overdotteddoublearrow{v} \qquad \overdotteddoublearrow{ABCD} $
  \end{dispExample}

% \onslide<+>

\undef\overparabola% \NewOverArrowCommand doesn't work inside beamer frame
  \begin{dispExample}
\NewOverArrowCommand[tikz]{overparabola}{%
  path options={
    x=\overarrowlength, line width=\overarrowsmallerthickness
  },
  path={(0,0) parabola[parabola height=0.2\overarrowlength] (1,0)},
  arrows={-}, center arrow, min length=30,
}
$\displaystyle \overparabola{v} \quad \overparabola{ABCD} $\qquad
$\scriptscriptstyle \overparabola{v} \quad \overparabola{ABCD} $
 \end{dispExample}

\end{block}

\end{frame}

\subsection{Précisions}

\DeclareOverArrowCommand[picture]{overpictarrow}{}

\DeclareOverArrowCommand{lookback}{%
  start=\leftarrow, end=\rightharpoondown,
  shift left=-20, shift right=-10,
}
\DeclareMathVersion{FM}
\setmathfont[version=FM]{Fira Math}

\begin{frame}[fragile]

\frametitle{Utilisation}
\framesubtitle{Précisions}

\begin{itemize}
 \item<+> Code flexible : d'autres méthodes peuvent être ajoutées\\ (\rmq{en
   plus de \texttt{symb}, \texttt{tikz} et \texttt{picture}}).

 \item<+> Package \pkg{esvect} chargé par défaut ; inadapté si la police
   mathématique est modifiée.
\begin{itemize}

 \item
  \begin{tabular}[t]{@{}lcl@{}}
    \alert{Exemple :} & $\lookback{ABCD}$                                          & avec la police \structure{Computer Modern}, \\
    & {\begingroup \mathversion{FM} $\lookback{ABCD}$ \endgroup} & avec \structure{Fira Math}.
  \end{tabular}

 \item Utiliser l'option \opt{noesvect} ou la clé \code{middle config = relbar}

   (\rmq{\pkg{esvect} définit \cs{relbareda} \symb{\relbareda} en remplacement
   de \cs{relbar} \symb{\relbar}}).

\end{itemize}

\item<+> La clé \key{detect subscripts} gère automatiquement les indices,% \alert{Exemple :}
\undef\autosub% \NewOverArrowCommand doesn't work inside beamer frame
   \begin{dispExample}
\NewOverArrowCommand{autosub}{detect subscripts, end=\rightarrow}

$ \imath_0 \qquad \autosub{\imath}_0 \qquad
  {\autosub{\imath}}_0 \qquad {\autosub*{\imath}}_0 $
   \end{dispExample}

\begin{itemize}
 \item ou utiliser l'option \opt{subscripts}.
 \item \rmq{Attention, si le caractère \structure{trait de soulignement}
   \symb{\_} est altéré\\ (\rmq{\alert{exemple} : caractère actif comme avec
   \href{https://ctan.org/pkg/altsubsup}{le package \pkg{altsubsup}}}) : voir
   manuel}.
\end{itemize}

\end{itemize}

\end{frame}


\section{Conclusion}

\let\overrightarrow\umoverrightarrow
\RenewOverArrowCommand{vv}{esvect, middle config=auto, detect subscripts}

\begin{frame}[fragile]

\frametitle{Conclusion}

\begin{block}{Pourquoi utiliser le package \pkg{overarrows} ?}
\begin{itemize}
 \item Parce qu'il est \structure{totalement personnalisable}.
 \item Parce qu'il accepte \structure{TikZ}.
 \item Parce qu'il sait \structure{détecter les indices}.
 \item Parce qu'il fournit un jeu de \structure{commande pré-définies} unifiées.
 \item Parce qu'il possède une \structure{documentation} complète et illustrée.
\end{itemize}
\end{block}

\begin{alertblock}<+>{Pourquoi ne pas utiliser le package \pkg{overarrows} ?}
\begin{itemize}
 \item Parce que les vecteurs se notent \structure{en gras}.
 \item Parce que \structure{\pkg{unicode-math}} suffit.
  \begin{dispExample}
$ \vv{v}_0 ~/~ \overrightarrow{v}_0 $ \qquad
$ \vv{AB}_0 ~/~ \overrightarrow{AB}_0 $
  \end{dispExample}
\end{itemize}
\end{alertblock}

\begin{center}
  \rule{6cm}{0.4pt}

  \url{https://ctan.org/pkg/overarrows}
\end{center}

\end{frame}

\end{document}
