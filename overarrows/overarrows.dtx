% \iffalse meta-comment
%
% Copyright (C) 2023 by Julien Labbé <Julien.Labbe@univ-grenoble-alpes.fr>
%
% This file may be distributed and/or modified under the conditions of the
% LaTeX Project Public License (LPPL), either version 1.3c of this license or
% (at your option) any later version. The latest version of this license is in
% the file https://www.latex-project.org/lppl.txt
%
% \fi
%
% \iffalse
%<*driver>
\ProvidesFile{overarrows.dtx}
%</driver>
%<package>\NeedsTeXFormat{LaTeX2e}
%<package>\ProvidesPackage{overarrows}
%<*package>
[2023/02/15 v1.1 Custom extensible arrows over math expressions]
%</package>
%
%<*driver>
\documentclass[nohyperref]{ltxdoc}
\usepackage{metalogo}

\usepackage{overarrows-doc} %^^A defined at end of this file, with guard <docstyle>
\usepackage[old-arrows, esvect, tikz, pstricks, pstarrows, allcommands]{overarrows}

\EnableCrossrefs
\PageIndex
% \CodelineIndex
\RecordChanges
% \OnlyDescription

%% symbols
\newcommand*{\symb}[1]{\colorbox{yellow!50}{\rule{0pt}{1ex}$#1$}}

%% Arrows used in doc
\usepackage{amssymb}% for \twoheadrightarrow
\NewOverArrowCommand{\amsoverrightarrow}{amsmath=strict, end={\rightarrow}}
\NewOverArrowCommand{\overrightnewarrow}{%
    amsmath, middle config=relbar,  end=\rightarrow,
    right arrow,
  }
\DeclareMathSymbol{\fldra}{\mathrel}{esvector}{'021}
\DeclareMathSymbol{\fldrb}{\mathrel}{esvector}{'022}
\DeclareMathSymbol{\fldrc}{\mathrel}{esvector}{'023}
\DeclareMathSymbol{\fldrd}{\mathrel}{esvector}{'024}
\DeclareMathSymbol{\fldre}{\mathrel}{esvector}{'025}
\DeclareMathSymbol{\fldrf}{\mathrel}{esvector}{'026}
\DeclareMathSymbol{\fldrg}{\mathrel}{esvector}{'027}
\DeclareMathSymbol{\fldrh}{\mathrel}{esvector}{'030}
\NewOverArrowCommand{\esveca}{esvect=strict, end={\fldra}}
\NewOverArrowCommand{\esvecb}{esvect=strict, end={\fldrb}}
\NewOverArrowCommand{\esvecc}{esvect=strict, end={\fldrc}}
\NewOverArrowCommand{\esvecd}{esvect=strict, end={\fldrd}}
\NewOverArrowCommand{\esvece}{esvect=strict, end={\fldre}}
\NewOverArrowCommand{\esvecg}{esvect=strict, end={\fldrg}}
\NewOverArrowCommand{\esvech}{esvect=strict, end={\fldrh}}
\NewOverArrowCommand{\vva}{esvect, middle config=auto, end=\fldra}
\NewOverArrowCommand{\esvecf}{esvect=strict, end={\fldrf}}
\NewOverArrowCommand{\vvb}{esvect, middle config=auto, end=\fldrb}
\NewOverArrowCommand{\vvc}{esvect, middle config=auto, end=\fldrc}
\NewOverArrowCommand{\vvd}{esvect, middle config=auto, end=\fldrd}
\NewOverArrowCommand{\vve}{esvect, middle config=auto, end=\fldre}
\NewOverArrowCommand{\vvf}{esvect, middle config=auto, end=\fldrf}
\NewOverArrowCommand{\vvg}{esvect, middle config=auto, end=\fldrg}
\NewOverArrowCommand{\vvh}{esvect, middle config=auto, end=\fldrh}

\makeatletter
\newcommand{\unsetoldarrows}{
  \let\ovar@rightarrow\rightarrow
  \let\ovar@leftarrow\leftarrow
}
\newcommand{\setoldarrows}{
  \let\ovar@rightarrow\varrightarrow
  \let\ovar@leftarrow\varleftarrow
}
\makeatother

\begin{document}

\DocInput{overarrows.dtx}

\PrintIndex
\PrintChanges

\end{document}
%</driver>
% \fi
%
% \changes{v1.0}{2023/01/18}{Initial version}
%
% \GetFileInfo{\jobname.dtx}
%
% \title{The \pkg{overarrows} package\thanks{This document
% corresponds to \pkg{overarrows}~\fileversion, dated \filedate.}}
% \author{Julien Labb\'e \\ \texttt{Julien.Labbe@univ-grenoble-alpes.fr}}
%
% \maketitle
%
% \begin{abstract}
%   A \LaTeX{} package to create custom arrows over math expressions, mainly for
%   vectors (but arrows can as well be drawn below). Arrows stretch with
%   content, scale with math styles, and have a correct kerning when a subscript
%   follows.
%
% \bigskip
%
% Short example:
% \iffalse
%<*example>
% \fi
\begin{dispExample}
\NewOverArrowCommand{\overrightharpoon}{%
  end=\rightharpoonup
}

\begin{align*}
  &\overrightharpoon{v} && \overrightharpoon{v}_{subscript} \\
  &\overrightharpoon{ABCD} && \overrightharpoon*{v}_{subscript}
\end{align*}
\end{dispExample}
% \iffalse
%</example>
% \fi
%
%\bigskip
%
% Predefined commands are also provided:
% \begin{itemize}
%  \item to typeset vectors:
%   $$ \vv{v} \qquad \vv{AB}, $$
%
% \item to draw arrows of various shapes above math expressions:
% \begin{align*}
%   \overrightarrow{AB}
%   && \overleftarrow{AB}
%   && \overleftrightarrow{AB}
%   && \overrightharpoonup{AB}
%   && \overleftharpoonup{AB}
%   && \overrightharpoondown{AB}
%   && \overleftharpoondown{AB}
%   && \overbar{AB},
% \end{align*}
%
% \item to draw arrows of various shapes under math expressions:
% \begin{align*}
%   \underrightarrow{AB}
%   && \underleftarrow{AB}
%   && \underleftrightarrow{AB}
%   && \underrightharpoonup{AB}
%   && \underleftharpoonup{AB}
%   && \underrightharpoondown{AB}
%   && \underleftharpoondown{AB}
%   && \underbar{AB}.
% \end{align*}
% \end{itemize}
%
% \end{abstract}
%
% \clearpage
%
% \tableofcontents{}
%\clearpage
%
% \section{Presentation of the package}
% \label{sec:package-description}
%
% The \pkg{overarrows} package allows to create commands for drawing arrows over
% math expressions. These arrows:
% \begin{itemize}
%  \item are fully customisable, at command definition, through a key-value
%   interface;
%  \item stretch with the content and can cover many characters, like in $\vv{AB}$;
%  \item scale with math styles\footnotemark{}, like in
%  $\vv*{v}_{\vv*{u}_{\vv*{w}}}$.
% \footnotetext{\cs{displaystyle}, \cs{textstyle},  \cs{scriptstyle} and
% \cs{scriptscriptstyle}.}
% \end{itemize}
%
% Commands created with the \pkg{overarrows} package are provided with a starred
% variant, that removes the extra end space generated by the arrow. This is
% particularly useful when the command is followed by a subscript. For example,
% the velocity of the center of mass can be written with exactly the same
% kerning when scalar $v_{\text{cm}}$ or vector $\vv*{v}_{\text{cm}}$ (no extra
% space before the subscript, unlike the output of the unstarred variant:
% $\vv{v}_{\text{cm}}$).
%
% The \pkg{overarrows} package was primitively written for vectors, but in
% a highly customisable way. It can be used to define a large variety of arrows,
% using math symbols, or drawing commands from PGF/TikZ or PSTricks. It's also
% possible to create commands that draw the arrows under. Some predefined
% commands are provided, giving\footnotemark{}, for arrow over:
% \footnotetext{Displayed here with the \refOpt{old-arrows} option.}
% \begin{align*}
%   \overrightarrow{\alpha+\beta}
%   && \overleftarrow{\alpha+\beta}
%   && \overleftrightarrow{\alpha+\beta}
%   && \overrightharpoonup{\alpha+\beta}
%   && \overleftharpoonup{\alpha+\beta}
%   && \overrightharpoondown{\alpha+\beta}
%   && \overleftharpoondown{\alpha+\beta}
%   && \overbar{\alpha+\beta}
% \end{align*}
% and for arrow under :
% \begin{align*}
%   \underrightarrow{\alpha+\beta}
%   && \underleftarrow{\alpha+\beta}
%   && \underleftrightarrow{\alpha+\beta}
%   && \underrightharpoonup{\alpha+\beta}
%   && \underleftharpoonup{\alpha+\beta}
%   && \underrightharpoondown{\alpha+\beta}
%   && \underleftharpoondown{\alpha+\beta}
%   && \underbar{\alpha+\beta}.
% \end{align*}
%
% \section{Introduction}
%
% \subsection{Vector arrows}
%
% Vectors are commonly typeset in bold face, or with an arrow
% above\footnotemark{}. For this second convention, \TeX/\LaTeX provides the
% command \cs{vec}, which accents its content (using the \cs{mathaccent}
% command) with the character~\symb{\mathchar "017E\relax}
% (\cs{mathchar}|"017E| in Computer Modern font).
% But \symb{\mathchar "017E\relax} isn't extensible, and gives: $\vec{v}$,
% $\vec{AB}$ or $\vec{\text{grad}}$ (there's no command \cs{widevec} analogous
% to \cs{widehat}).
%
% \footnotetext{See, for example: International Organization for
% Standardization. (2019). \emph{Quantities and units -- Part 2: Mathematics}
% (ISO Standard No. 80000-2:2019).
% \url{https://www.iso.org/standard/64973.html}.}
%
% An extensible alternative is given by the command \cs{overrightarrow},
% available in \TeX/\LaTeX, and which is redefined by the commonly used
% \pkg{amsmath} package. But its arrow, built with the \cs{rightarrow} symbol
% \symb{\rightarrow}, is too large with the default \emph{Computer Modern} font:
% $\amsoverrightarrow{AB}$. Another alternative is the \pkg{esvect} package,
% which provides the \cs{vv} command and a set of custom arrows: $\esveca{AB}$,
% $\esvecb{AB}$, $\esvecc{AB}$, $\esvecd{AB}$, $\esvece{AB}$, $\esvecf{AB}$,
% $\esvecg{AB}$, $\esvech{AB}$.
%
% \subsection{Stack and arrow macros}
%
% It's worth looking at the definition of \pkg{amsmath} \cs{overrightarrow}
% command:
% \iffalse
%<*example>
% \fi
\begin{dispListing}
\long macro:->\mathpalette {\overarrow@ \rightarrowfill@ }
\end{dispListing}
% \iffalse
%</example>
% \fi
% \noindent Three macros are used here:
% \begin{description}
%    \item[\cs{mathpalette}] adapts the output to the current math style;
%    \item[\cs{overarrow@}] is the \emph{stack macro}, that puts the arrow above
%    the content;
%    \item[\cs{rightarrowfill@}] is the \emph{arrow macro}, that holds the
%    content of the arrow.
% \end{description}
% The command \cs{vv} from \pkg{esvec} is defined with a very similar way, using
% its own stack macro (\cs{overvect@}) and arrow macro (\cs{vectfill@}).
%
% \medskip
%
% The \pkg{overarrows} package uses the same mechanism. Arrow and stack macros
% are set, at command creation, through a key-value interface provided by the
% \pkg{pgfkeys} package (after creation, however, the command definition is
% static and the key-value interface is not used).
%
% \subsection{Extensible arrows}
% \label{sec:extensible-arrows}
%
% Arrows drawn by the commands \cs{overrightarrow} or \cs{vv} are built by
% joining math symbols, and made extensible by repetition of the central
% symbol\footnotemark{}. Thus, the line of the macro \cs{overrightarrow} is made
% by repetition of command \cs{relbar} \symb{\relbar} (which simply corresponds
% to the minus sign), while \cs{vv} use its own command \cs{relbareda}
% \symb{\relbareda}.
%
%\footnotetext{Using the \TeX{} \cs{cleaders} command.}
%
% This method may generate some undesirable spacing issues, when symbols
% badly overlap. See, for example, the output of \pkg{amsmath}
% \cs{overrightarrow} (left) and \pkg{esvect} \cs{vv} (right) in
% \cs{scriptscriptstyle} math style (scaled by a factor~ 4):
% \begin{center}
%   \scalebox{4}{$\scriptscriptstyle \amsoverrightarrow{long~vector}$}
%   \qquad
%   \scalebox{4}{$\scriptscriptstyle \esvectvv{long~vector}$}.
% \end{center}
% While the arrow on the left lets guess where the symbols \symb{\relbar}
% overlap, the arrow on the right present unwanted spaces and show clearly its
% composition as association of the symbols \symb{\relbaredd}, \symb{\relbareda}
% and \symb{\fldr}.
%
% \medskip
%
% By default, the \pkg{overarrows} package uses the same mechanism to extend
% arrows according to their contents. Settings and tools are provided to perform
% fine tuning and avoid spacing issues. As example, see below the
% \cs{overrightarrow} and \cs{vv} commands, as redefined by \pkg{overarrows} (in
% \cs{scriptscriptstyle} and scaled by a factor~4):
% \begin{center}
%   \scalebox{4}{$\scriptscriptstyle \overrightnewarrow{long~vector}$}
%   \qquad
%   \scalebox{4}{$\scriptscriptstyle \vv{long~vector}$}
% \end{center}
%
% The \pkg{overarrows} package also provides an alternative mechanism.
% When used, the length \cs{overarrowlength} is set, according to the arrow
% command content, and can be employed, for example, to draw arrows using
% PGF/TikZ or the \LaTeX{} picture environment.
%
% \section{Quick start}
%
% \subsection{Loading the package \pkg{overarrows}}
%
% To load the \pkg{overarrows}, simply add in preamble, before the
% ``\cs{begin}|{document}|":
% \iffalse
%<*example>
% \fi
\begin{dispListing}
\usepackage{overarrows}
\end{dispListing}
% \iffalse
%</example>
% \fi
%
% Options can be given, in a comma-separated list. For example, to use the
% predefined commands shown in the section~\ref{sec:package-description},
% page~\pageref{sec:package-description}, write:
% \iffalse
%<*example>
% \fi
\begin{dispListing}
\usepackage[allcommands, old-arrows]{overarrows}
\end{dispListing}
% \iffalse
%</example>
% \fi
% \noindent This define the commands (described in
% section~\ref{sec:predefined-commands},
% page~\pageref{sec:predefined-commands}):
% \begin{center}
% \begin{minipage}{0.5\linewidth}
%   \begin{itemize}
%   \item \refCom{overrightarrow}
%   \item \refCom{overleftarrow}
%   \item \refCom{overleftrightarrow}
%   \item \refCom{overrightharpoonup}
%   \item \refCom{overrightharpoondown}
%   \item \refCom{overleftharpoonup}
%   \item \refCom{overleftharpoondown}
%   \item \refCom{overbar}
%   \end{itemize}
% \end{minipage}\hfill
% \begin{minipage}{0.5\linewidth}
%   \begin{itemize}
%   \item \refCom{underrightarrow}
%   \item \refCom{underleftarrow}
%   \item \refCom{underleftrightarrow}
%   \item \refCom{underrightharpoonup}
%   \item \refCom{underrightharpoondown}
%   \item \refCom{underleftharpoonup}
%   \item \refCom{underleftharpoondown}
%   \item \refCom{underbar}
%   \end{itemize}
% \end{minipage}
% \end{center}
% Note that the \refOpt{old-arrows} option may give bad results, if math fonts
% have been changed. Simply remove the option in this case.
%
% Many other options are available. See the complete list,
% page~\pageref{sec:package-options}.
%
% \subsection{Commands creation}
%
% Commands are created with \refCom{NewOverArrowCommand}. This macro take two
% mandatory arguments : the name of the command (without backslash), and the
% arrow configuration as comma-separated list of key-values. By default, a right
% arrow is set:
% \iffalse
%<*example>
% \fi
\begin{dispExample}
\NewOverArrowCommand{\myovercmd}{}
$\myovercmd{test}$
\end{dispExample}
% \iffalse
%</example>
% \fi
% \noindent Commands are defined with a starred variant, designed to handle
% subscripts:
% \iffalse
%<*example>
% \fi
\begin{dispExample}
$ v_{sub} \qquad \myovercmd{v}_{sub} \qquad \myovercmd*{v}_{sub} $
\end{dispExample}
% \iffalse
%</example>
% \fi
%
% \subsection{Start and end of the arrow}
%
% Extremities of the arrow are set by the keys \refKey{start} and \refKey{end}.
% For example, an arrow starting with a hook (symbols \cs{lhook} \symb{\lhook})
% and ending with two heads (symbol \cs{twoheadrightarrow}
% \symb{\twoheadrightarrow}) is defined by:
% \iffalse
%<*example>
% \fi
\begin{dispListing}
\NewOverArrowCommand{\overhooktwoheadrightarrow}{%
  start=\lhook, end=\twoheadrightarrow,
}
\end{dispListing}
% \iffalse
%</example>
% \fi
% \noindent Note that \cs{twoheadrightarrow} must be defined, as it is not in
% \LaTeX{}. This can be done with the package \pkg{amssymb}, by adding in
% preamble:
% \iffalse
%<*example>
% \fi
\begin{dispListing}
\usepackage{amssymb}
\end{dispListing}
% \iffalse
%</example>
% \fi
% With the previous definition, the result of the command
% \cs{overhooktwoheadrightarrow} is faulty:
% \iffalse
%<*example>
% \fi
\NewOverArrowCommand{\overhooktwoheadrightarrow}{%
  start=\lhook, end=\twoheadrightarrow,
}
\begin{dispExample}
$ \overhooktwoheadrightarrow{v} \qquad \overhooktwoheadrightarrow{AB} $
\end{dispExample}
% \iffalse
%</example>
% \fi
% \noindent The problem comes from symbols junction and the trimming used to
% obtain their overlap. It can be solved with the keys \refKey{trim start} and
% \refKey{trim end}, which are numbers and set the corresponding trimming in
% math units (typically |1/18 em|). Appropriate values gives better results:
% \iffalse
%<*example>
% \fi
\csundef{overhooktwoheadrightarrow}
\begin{dispExample}
\NewOverArrowCommand{\overhooktwoheadrightarrow}{%
  start=\lhook, end=\twoheadrightarrow,
  trim start=1.5, trim end=2,
}
$ \overhooktwoheadrightarrow{v} \qquad \overhooktwoheadrightarrow{AB} $
\end{dispExample}
% \iffalse
%</example>
% \fi
%
% \medskip
%
% If the math font differs from the default \emph{Computer Modern}, the central
% part of the arrow may have inappropriate position or line width. This is
% because the default symbol used for the arrow line is \cs{relbareda}
% \symb{\relbareda} from the \pkg{esvect} package. If needed, try to set the
% \refKey{middle} key with the symbol \cs{relbar} \symb{\relbar}. The trimming
% should also be adapted:
% \iffalse
%<*example>
% \fi
\csundef{overhooktwoheadrightarrow}
\begin{dispExample}
\NewOverArrowCommand{\overhooktwoheadrightarrow}{%
  start=\lhook, end=\twoheadrightarrow, middle=\relbar, %
  trim start=0, trim end=3, trim middle=5,
}
$ \overhooktwoheadrightarrow{v} \qquad \overhooktwoheadrightarrow{AB} $
\end{dispExample}
% \iffalse
%</example>
% \fi
%
% Finding the correct values for \refKey{trim start}, \refKey{trim end} and
% \refKey{trim middle} may need many trials. For this purpose, the macro
% \refCom{TestOverArrow} displays the result of a command for different lengths
% and math styles:
% \iffalse
%<*example>
% \fi
\begin{dispExample}
\TestOverArrow{\overhooktwoheadrightarrow}
\end{dispExample}
% \iffalse
%</example>
% \fi
%
% \subsection{Size and position of the arrow}
%
% A command \cs{OverRightarrow}, built with the symbols \cs{Relbar}
% \symb{\Relbar} and \cs{Rightarrow} \symb{\Rightarrow}, gives:
%\iffalse
%<*example>
% \fi
\begin{dispExample}
\NewOverArrowCommand{\OverRightarrow}{%
  start=\Relbar,
  middle=\Relbar,
  end=\Rightarrow,
  trim=4,
}
$ \OverRightarrow{v} \qquad \OverRightarrow{AB} $
\end{dispExample}
% \iffalse
%</example>
% \fi
%\noindent The key \refKey{trim} sets \refKey{trim start}, \refKey{trim middle} and
%\refKey{trim end} with the same value.
%
% The previous arrow is visually too big. The macro \refCom{smallermathstyle}
% allows to obtain a better result:
% \iffalse
%<*example>
% \fi
\csundef{OverRightarrow}
\begin{dispExample}
\NewOverArrowCommand{\OverRightarrow}{%
  start={\smallermathstyle\Relbar},
  middle={\smallermathstyle\Relbar},
  end=\Rightarrow,
  trim=4,
}
$ \OverRightarrow{v} \qquad \OverRightarrow{AB} $
\end{dispExample}
% \iffalse
%</example>
% \fi
% \noindent Note that \refCom{smallermathstyle} should not be used for
% \refKey{end}, because this last is formatted with the same math style as
% \refKey{start}.
%
% It would be better to add an extra space between the arrow and the content of
% the command. This can be done with the key \refKey{space after arrow}:
% \iffalse
%<*example>
% \fi
\csundef{OverRightarrow}
\begin{dispExample}
\NewOverArrowCommand{\OverRightarrow}{%
  start={\smallermathstyle\Relbar},
  middle={\smallermathstyle\Relbar},
  end=\Rightarrow,
  trim=4,
  space after arrow=0.25ex,
}
$ \OverRightarrow{v} \qquad \OverRightarrow{AB} $
\end{dispExample}
% \iffalse
%</example>
% \fi
%
% \medskip
%
% Default arrows are slightly shifted to the right. For a left arrow, this
% should be reversed, using the keys \refKey{shift left} and \refKey{shift
% right}. These keys set the corresponding shifts, in math units. Example:
% \iffalse
%<*example>
% \fi
\begin{dispExample}
\NewOverArrowCommand{\OverLeftarrow}{%
  start={\smallermathstyle\Leftarrow},
  middle={\smallermathstyle\Relbar},
  end=\Relbar,
  trim=4,
  space after arrow=0.25ex,
  shift left=0, shift right=2,
}
$ \OverLeftarrow{v} \qquad \OverLeftarrow{AB} $
\end{dispExample}
% \iffalse
%</example>
% \fi
%
% \medskip
%
% Finally, the key \refKey{arrow under} places the arrow below the content,
% instead of above (and \refKey{space before arrow} sets the space upon it):
% \iffalse
%<*example>
% \fi
\begin{dispExample}
\NewOverArrowCommand{\UnderLeftRightarrow}{%
  start={\smallermathstyle\Leftarrow},
  middle={\smallermathstyle\Relbar},
  end=\Rightarrow,
  trim=4,
  arrow under,
  space before arrow=0.5ex,
  shift left=0, shift right=0,
}
$ \UnderLeftRightarrow{v} \qquad \UnderLeftRightarrow{AB} $
\end{dispExample}
% \iffalse
%</example>
% \fi
%
% \subsection{Symbols assemblage}
%
% Many \LaTeX{} math symbols are built by assemblage, using the macro
% \cs{joinrel}\footnotemark{} which remove 3 math units of horizontal space.
% The \pkg{overarrows} package provides a flexible version of \cs{joinrel},
% called \refCom{xjoinrel}, which remove an arbitrary number of math units,
% given as optional argument.
%
%\footnotetext{For example, the symbol \cs{models}
% \smash{\symb{\models}} is defined as
% \cs{mathrel}\texttt{\{\textbar\}}\cs{joinrel}\cs{Relbar} and corresponds to
% the assemblage of a  vertical line \symb{\vert} and the symbol \cs{Relbar}
% \symb{\Relbar}. The command \cs{mathrel} modifies the spacing according to the
% math relation class ; \cs{Relbar} corresponds to the equal sign (it's
% definition is \cs{mathrel}\texttt{\{=\}}).}
%
% Symbols association is then simple. As example, one can define a triple tail
% macro \cs{tttail} from the symbol \cs{succ} \symb{\succ}:
% \iffalse
%<*example>
% \fi
\begin{dispExample}
\newcommand*{\tttail}{\succ\xjoinrel[10]\succ\xjoinrel[10]\succ}
$ \tttail $
\end{dispExample}
\newcommand*{\tttail}{\succ\xjoinrel[10]\succ\xjoinrel[10]\succ}
% \iffalse
%</example>
% \fi
% \noindent Thus defined, the macro \cs{tttail} can be used in arrow definition:
%\iffalse
%<*example>
% \fi
\begin{dispExample}
\NewOverArrowCommand{\overtttailrightarrow}{%
  start={\tttail},
  end={\rightarrow},
  trim start=12,
  shift left=0, shift right=0,
  space after arrow=.2ex,
  min length=24,
}
$ \overtttailrightarrow{v} \qquad \overtttailrightarrow{AB} $
\end{dispExample}
% \iffalse
%</example>
% \fi
% \noindent Here the \refKey{min length} key was added to ensure a minimum length
% (in math units) when the content of the command is small (as for a single
% character).
%
% \medskip
%
% The previous arrow would be better with a smaller tail, and this can be done
% with the macro \refCom{smallermathstyle}. But a small tail and a normal sized
% head are not aligned; as |{\smallermathstyle\tttail}\xjoinrel[8]\rightarrow}|
% gives:
% \begin{center}
%      \scalebox{2}{${\smallermathstyle\tttail}\xjoinrel[8]\rightarrow$}
% \end{center}
% \noindent The solution comes from the command \cs{vcenter} which centers
% materials on math axis. The tail must then be wrapped in a \cs{hbox}:
% \iffalse
%<*example>
% \fi
\csundef{overtttailrightarrow}
\begin{dispExample}
\NewOverArrowCommand{\overtttailrightarrow}{%
  start={\vcenter{\hbox{$\smallermathstyle\tttail$}}},
  end={\rightarrow},
  trim start=12,
  shift left=0, shift right=0,
  space after arrow=.2ex,
  min length=24,
}
$ \overtttailrightarrow{v} \qquad \overtttailrightarrow{AB} $
\end{dispExample}
% \iffalse
%</example>
% \fi
%
% \medskip
%
% Text symbols, namely symbols that are not defined in math mode, can also be
% used. They should yet be enclosed in the \cs{text} macro, from the
% \pkg{amsmath} package, to be correctly displayed and correctly scaled
% according to math style. With, for example, the arrow heads given by the
% symbols 40 and 41 of the \emph{lasy} font:
% \iffalse
%<*example>
% \fi
\begin{dispExample}
\newcommand*{\leftarrowhead}{\usefont{U}{lasy}{m}{n}\symbol{40}}
\newcommand*{\righttarrowhead}{\usefont{U}{lasy}{m}{n}\symbol{41}}
\NewOverArrowCommand{\overrightleftarrow}{%
  start=\text{\righttarrowhead},
  end=\text{\leftarrowhead},
  trim start=0.7,  trim end=0.7,
  min length=20,
  shift leftright=-2,
}
$ \overrightleftarrow{AB} \qquad \scriptstyle\overrightleftarrow{AB} $
\end{dispExample}
% \iffalse
%</example>
% \fi
%
% \subsection{Drawing the arrow with TikZ}
%
% In addition to the default method presented previously (assemblage of symbols,
% as described in section~\ref{sec:extensible-arrows},
% page~\pageref{sec:extensible-arrows}), the \pkg{overarrows} package has an
% alternative method to draw the arrow. This one allows the use of graphic
% languages such as PGF/TikZ.
%
% Drawing arrows with TikZ requires to load the \pkg{tikz} package and its
% library |arrows.meta|. This can be simply done by passing the \refOpt{tikz}
% option to the \pkg{overarrows} package\footnotemark{}:
%
% \footnotetext{Note that the \refOpt{tikz} option isn't mandatory to use TikZ
% commands in \pkg{overarrows}. The \pkg{tikz} package and its library
% |arrows.meta| can be loaded independently.}
% \iffalse
%<*example>
% \fi
\begin{dispListing}
\usepackage[tikz]{overarrows}
\end{dispListing}
% \iffalse
%</example>
% \fi
%
% \medskip
%
% To use PGF/TikZ language, the optional argument |tikz| must be passed to
% \refCom{NewOverArrowCommand} . TikZ picture are not extensible. That's why the
% \pkg{overarrows} package provides three lengths that can be used in TikZ
% commands:
% \begin{itemize}
%  \item \refCom{overarrowlength} for the arrow length,
%  \item \refCom{overarrowthickness} and \refCom{overarrowsmallerthickness} for
%   the arrow thickness.
% \end{itemize}
% These lengths are computed at each utilisation of a command created with the
% |tikz| optional argument.
%
% Without any other configuration, a right arrow is drawn:
% \iffalse
%<*example>
% \fi
\begin{dispExample}
\NewOverArrowCommand[tikz]{\overtikzarrow}{}
$ \overtikzarrow{v} \qquad \overtikzarrow{AB} $
\end{dispExample}
% \iffalse
%</example>
% \fi
%
% \medskip
%
% Keys to use Tikz are described in section~\ref{sec:tikz-keys},
% page~\pageref{sec:tikz-keys}. Main keys are: \refKey{tikz options},
% \refKey{path options} and \refKey{path}. It's also possible to append settings
% with \refKey{add tikz options} and \refKey{add path options}. The full TikZ
% command used to draw the arrow can as well be entirely redefined
% with the key \refKey{tikz command}
%
% Here is a example of an arrow drawn with TikZ\footnotemark{}:
%
% \footnotetext{TikZ arrows are very powerfull, but much slower to draw than the
% default method using assemblage of math symbols.}
% \iffalse
%<*example>
% \fi
\begin{dispExample}
\NewOverArrowCommand[tikz]{\overarchedleftrightarrow}{%
  add tikz options={y=\overarrowlength},
  add tikz options={line width={\overarrowsmallerthickness}},
  path options={arrows={<[scale=0.5]->[scale=0.5]}},
  path={(0,0) arc (-250:70:0.5 and 0.1)},
  center arrow,
  min length=25,
  space after arrow=0.4ex,
}
$ \overarchedleftrightarrow{v} \qquad \overarchedleftrightarrow{ABCD} $

\end{dispExample}
% \iffalse
%</example>
% \fi
%
% \subsection{Drawing the arrow with PSTricks}
%
% PSTricks commands can also be used to draw the arrow. For this, the optional
% argument |pstricks| must be passed to \refCom{NewOverArrowCommand}. Like for
% TikZ, the three lengths \refCom{overarrowlength}, \refCom{overarrowthickness}
% and \refCom{overarrowsmallerthickness} can be used in PSTricks commands.
% By default, a right arrow is drawn:
% \iffalse
%<*example>
% \fi
\begin{dispExample}
\NewOverArrowCommand[pstricks]{\overpstarrow}{}
$ \overpstarrow{v} \qquad \overpstarrow{AB} $
\end{dispExample}
% \iffalse
%</example>
% \fi
% \noindent The \pkg{pstricks} package has to be loaded (for example, using the
% \refOpt{pstricks} option to the \pkg{overarrows} package)
%
% Keys to use PSTricks commands are described in section~\ref{sec:pstricks-keys},
% page~\pageref{sec:pstricks-keys}. The main keys are \refKey{pstricks command},
% \refKey{psset}, \refDoc{key:pstricks}{arrow}, \refDoc{key:pstricks}{geometry}
% an \refDoc{key:pstricks}{line thickness}. Examples:
% \iffalse
%<*example>
% \fi
\begin{dispExample}
\NewOverArrowCommand[pstricks]{\overreddisks}{%
  psset={linecolor=red}, arrow=*-*, center arrow,
}
$ \overreddisks{v} \qquad \overreddisks{AB} $
\end{dispExample}
\begin{dispExample}
\NewOverArrowCommand[pstricks]{\ellipticarrow}{%
  pstricks command={%
    \psellipticarcn{->}%^^A strip spaces
    (0.5\overarrowlength,0.2\overarrowlength)%^^A strip spaces
    (0.5\overarrowlength,0.2\overarrowlength)%^^A strip spaces
    {170}{10}
  },
  geometry={(0,0.2\overarrowlength)(\overarrowlength,0.4\overarrowlength)},
  line thickness={\overarrowsmallerthickness},
  center arrow,
}
$ \ellipticarrow{v} \qquad \ellipticarrow{AB} $
\end{dispExample}
% \iffalse
%</example>
% \fi
%
% \subsection{Drawing the arrow with \LaTeX{} picture environment}
%
% Without any other package, arrows can also be drawn with the \LaTeX{}
% |picture| environment. In this case, the optional argument |picture| must be
% passed to \refCom{NewOverArrowCommand}. As for TikZ or PSTricks, the three
% lengths \refCom{overarrowlength}, \refCom{overarrowthickness} and
% \refCom{overarrowsmallerthickness} are available and can to be used with
% |picture| drawing commands. By default, a right vector is drawn:
% \iffalse
%<*example>
% \fi
\ltxarrows%^^A classic arrows
\begin{dispExample}
\NewOverArrowCommand[picture]{\overpictarrow}{}
$ \overpictarrow{v} \qquad \overpictarrow{AB} $
\end{dispExample}
% \iffalse
%</example>
% \fi
% \noindent If \pkg{overarrows} is loaded with the option \refOpt{pstarrows},
% the package \pkg{pict2e} is used and a PSTricks style vector arrows is set.
% This gives:
% \iffalse
%<*example>
% \fi
\pstarrows%^^A pst arrows
\csundef{overpictarrow}
\begin{dispExample}
\NewOverArrowCommand[picture]{\overpictarrow}{}
$ \overpictarrow{v} \qquad \overpictarrow{AB} $
\end{dispExample}
% \iffalse
%</example>
% \fi
%
% Keys to use \LaTeX{} |picture| environment are described in
% section~\ref{sec:picture-keys}, page~\pageref{sec:picture-keys}. The main keys
% are \refKey{picture command}, \refDoc{key:picture}{geometry} an
% \refDoc{key:picture}{line thickness}. Here is an example:
% \iffalse
%<*example>
% \fi
\begin{dispExample}
\NewOverArrowCommand[picture]{\overbandedarrow}{
  picture command={%
    \qbezier
    (0.0\overarrowlength,0)
    (0.5\overarrowlength,0)
    (0.9\overarrowlength,0.2\overarrowlength)
    \put(0.9\overarrowlength,0.2\overarrowlength)
    {\vector(2,1){0.2\overarrowlength}}
  },
  geometry={(\overarrowlength,0.4\overarrowlength)(0,0)},
  line thickness={\overarrowsmallerthickness},
  center arrow,
  space after arrow=0.4ex,
}
$ \overbandedarrow{v} \qquad \overbandedarrow{AB} $
\end{dispExample}
% \iffalse
%</example>
% \fi
%
% \section{User interface}
%
% \subsection{Package options}
% \label{sec:package-options}
%
% The \pkg{overarrows} package accepts many options, given as a comma-separated
% list \meta{options} at package loading:
% \cs{usepackage}|[|\meta{options}|]{overarrows}|.
%
% The option \refOpt*{esvect} is set by default. This can be overridden with
% \refOpt*{noesvect}.
%
% \subsubsection{\pkg{esvect} configuration}
% \label{sec:esvect-conf}
%
% \begin{docOption}{esvect}
%   Loads the \pkg{esvect} package and redefines its vector commands \refCom{vv}
%   through the \pkg{overarrows} mechanism. Original \pkg{esvect} \cs{vv} macro
%   is still available with \refCom{esvectvv}. The \pkg{esvect} font description
%   is fixed to allow any font sizes.
%
%   The \pkg{esvect} package provides the symbol \cs{relbareda} \symb{\relbareda}
%   which is smaller and often more flexible than the classic one \cs{relbar}
%   \symb{\relbar}. \cs{relbareda} fits with the standard \emph{Computer Modern}
%   math font, but can be unsuitable with other fonts.
%
%   The \pkg{esvect} package also provides the right arrow command \cs{fldr}.
%   The shape of the arrow depends on the option passed to the \pkg{esvect}
%   package: \symb{\fldra} (option~|a|), \symb{\fldrb} (option~|b|), \symb{\fldrc}
%   (option~|c|), \symb{\fldrd} (option~|d|), \symb{\fldre} (option~|e|),
%   \symb{\fldrf} (option~|f|), \symb{\fldrg} (option~|g|) or \symb{\fldrh}
%   (option~|h|). Note that by default \pkg{overarrows} loads the \pkg{esvect}
%   package with the option~|f| (while \pkg{esvect} default is~|d|). This can be
%   changed with one of the eight options described bellow: \refOpt*{esvecta},
%   \refOpt*{esvectb}, \refOpt*{esvectc}, \refOpt*{esvectd}, \refOpt*{esvecte},
%   \refOpt*{esvectf}, \refOpt*{esvectg} and \refOpt*{esvecth}.
%
%   This option is set by default and can be unset with \refOpt*{noesvect}.
% \end{docOption}
%
% \begin{docOption}{noesvect}
%   Prevents the loading of the \pkg{esvect} package and the definition of the
%   command \refCom{vv}.
% \end{docOption}
%
% \begin{docOption}{esvecta}
%   Loads the \pkg{esvect} package with the |a| option.
%
%   \cs{fldr} corresponds the to the symbol \symb{\fldra}. \cs{vv} command
%   gives : $\vva{v} \quad \vva{AB} \quad \vva{\mathrm{grad}}$.
% \end{docOption}
%
% \begin{docOption}{esvectb}
%   Loads the \pkg{esvect} package with the |b| option.
%
%   \cs{fldr} corresponds the to the symbol \symb{\fldrb}. \cs{vv} command
%   gives : $\vvb{v} \quad \vvb{AB} \quad \vvb{\mathrm{grad}}$.
% \end{docOption}
%
% \begin{docOption}{esvectc}
%   Loads the \pkg{esvect} package with the |c| option.
%
%   \cs{fldr} corresponds the to the symbol \symb{\fldrc}. \cs{vv} command
%   gives : $\vvc{v} \quad \vvc{AB} \quad \vvc{\mathrm{grad}}$.
% \end{docOption}
%
% \begin{docOption}{esvectd}
%   Loads the \pkg{esvect} package with the |d| option.
%
%   \cs{fldr} corresponds the to the symbol \symb{\fldrd}. \cs{vv} command
%   gives : $\vvd{v} \quad \vvd{AB} \quad \vvd{\mathrm{grad}}$.
% \end{docOption}
%
% \begin{docOption}{esvecte}
%   Loads the \pkg{esvect} package with the |e| option.
%
%   \cs{fldr} corresponds the to the symbol \symb{\fldre}. \cs{vv} command
%   gives : $\vve{v} \quad \vve{AB} \quad \vve{\mathrm{grad}}$.
% \end{docOption}
%
% \begin{docOption}{esvectf}
%   Loads the \pkg{esvect} package with the |f| option.
%
%   \cs{fldr} corresponds the to the symbol \symb{\fldrf}. \cs{vv} command
%   gives : $\vvf{v} \quad \vvf{AB} \quad \vvf{\mathrm{grad}}$.
% \end{docOption}
%
% \begin{docOption}{esvectg}
%   Loads the \pkg{esvect} package with the |g| option.
%
%   \cs{fldr} corresponds the to the symbol \symb{\fldrg}. \cs{vv} command
%   gives : $\vvg{v} \quad \vvg{AB} \quad \vvg{\mathrm{grad}}$.
% \end{docOption}
%
% \begin{docOption}{esvecth}
%   Loads the \pkg{esvect} package with the |h| option.
%
%   \cs{fldr} corresponds the to the symbol \symb{\fldrh}. \cs{vv} command
%   gives : $\vvh{v} \quad \vvh{AB} \quad \vvh{\mathrm{grad}}$.
% \end{docOption}
%
% \subsubsection{Predefined commands}
%\label{sec:predefined-commands-opt}
%
% The \pkg{overarrows} package provides sixteen predefined commands, eight with
% the arrow over, and eight with the arrow under. By default, theses commands
% are not defined, and must be activated by the corresponding option.
% Beware that commands are created without checking if already defined by
% another package (\cs{overleftarrow}, \cs{overrightarrow},
% \cs{overleftrightarrow}, \cs{underleftarrow}, \cs{underrightarrow} and
% \cs{underleftrightarrow} are, for example, part of the \pkg{amsmath} package).
%
% Three options are also available to define set of commands.
%
% \medskip
% \unsetoldarrows
%
% \noindent\textbf{Set of commands}
%
% \begin{docOption}{allcommands}
%   Defines all sixteen predefined commands.
%\end{docOption}
% \begin{docOption}{overcommands}
%   Defines all eight predefined commands with arrow over.
%\end{docOption}
% \begin{docOption}{undercommands}
%   Defines all eight predefined commands with arrow under.
% \end{docOption}
%
% \medskip
%
% \noindent\textbf{Over arrows}
%
% \begin{docOption}{overrightarrow}
%   Defines the \refCom{overrightarrow} command: $\overrightarrow{v}$,
%   $\overrightarrow{AB}$, $\overrightarrow{\textrm{grad}}$.
% \end{docOption}
% \begin{docOption}{overleftarrow}
%   Defines the \refCom{overleftarrow} command: $\overleftarrow{v}$, $\overleftarrow{AB}$, $\overleftarrow{\textrm{grad}}$.
% \end{docOption}
% \begin{docOption}{overleftrightarrow}
%   Defines the \refCom{overleftrightarrow} command: $\overleftrightarrow{v}$,
%   $\overleftrightarrow{AB}$, $\overleftrightarrow{\textrm{grad}}$.
% \end{docOption}
% \begin{docOption}{overrightharpoonup}
%   Defines the \refCom{overrightharpoonup} command: $\overrightharpoonup{v}$,
%   $\overrightharpoonup{AB}$, $\overrightharpoonup{\textrm{grad}}$.
% \end{docOption}
% \begin{docOption}{overrightharpoondown}
%   Defines the \refCom{overrightharpoondown} command:
%   $\overrightharpoondown{v}$, $\overrightharpoondown{AB}$,
%   $\overrightharpoondown{\textrm{grad}}$.
% \end{docOption}
% \begin{docOption}{overleftharpoonup}
%   Defines the \refCom{overleftharpoonup} command: $\overleftharpoonup{v}$,
%   $\overleftharpoonup{AB}$, $\overleftharpoonup{\textrm{grad}}$.
% \end{docOption}
% \begin{docOption}{overleftharpoondown}
%   Defines the \refCom{overleftharpoondown} command: $\overleftharpoondown{v}$,
%   $\overleftharpoondown{AB}$, $\overleftharpoondown{\textrm{grad}}$.
% \end{docOption}
% \begin{docOption}{overbar}
%   Defines the \refCom{overbar} command: $\overbar{v}$, $\overbar{AB}$,
%   $\overbar{\textrm{grad}}$.
% \end{docOption}
%
% \medskip
%
% \noindent\textbf{Under arrows}
%
% \begin{docOption}{underrightarrow}
%   Defines the \refCom{underrightarrow} command: $\underrightarrow{v}$,
%   $\underrightarrow{AB}$, $\underrightarrow{\textrm{grad}}$.
% \end{docOption}
% \begin{docOption}{underleftarrow}
%   Defines the \refCom{underleftarrow} command: $\underleftarrow{v}$,
%   $\underleftarrow{AB}$, $\underleftarrow{\textrm{grad}}$.
% \end{docOption}
% \begin{docOption}{underleftrightarrow}
%   Defines the \refCom{underleftrightarrow} command: $\underleftrightarrow{v}$,
%   $\underleftrightarrow{AB}$, $\underleftrightarrow{\textrm{grad}}$.
% \end{docOption}
% \begin{docOption}{underrightharpoonup}
%   Defines the \refCom{underrightharpoonup} command: $\underrightharpoonup{v}$,
%   $\underrightharpoonup{AB}$, $\underrightharpoonup{\textrm{grad}}$.
% \end{docOption}
% \begin{docOption}{underrightharpoondown}
%   Defines the \refCom{underrightharpoondown} command:
%   $\underrightharpoondown{v}$, $\underrightharpoondown{AB}$,
%   $\underrightharpoondown{\textrm{grad}}$.
% \end{docOption}
% \begin{docOption}{underleftharpoonup}
%   Defines the \refCom{underleftharpoonup} command: $\underleftharpoonup{v}$,
%   $\underleftharpoonup{AB}$, $\underleftharpoonup{\textrm{grad}}$.
% \end{docOption}
% \begin{docOption}{underleftharpoondown}
%   Defines the \refCom{underleftharpoondown} command:
%   $\underleftharpoondown{v}$, $\underleftharpoondown{AB}$,
%   $\underleftharpoondown{\textrm{grad}}$.
% \end{docOption}
% \begin{docOption}{underbar}
%   Defines the \refCom{underbar} command: $\underbar{v}$, $\underbar{AB}$,
%   $\underbar{\textrm{grad}}$.
% \end{docOption}
%
% \subsubsection{Other options}
%
% \begin{docOption}{old-arrows}\setoldarrows
%   Loads the \pkg{old-arrows} package with its option |old|. This provides the
%   symbols \cs{varleftarrow} \symb{\varleftarrow} and \cs{varrightarrow}
%   \symb{\varrightarrow}, used then by default for predefined command.
%
%   When the \refOpt*{old-arrows} option is set, the commands
%   \refCom{overrightarrow}, \refCom{overleftarrow},
%   \refCom{overleftrightarrow}, \refCom{underrightarrow},
%   \refCom{underleftarrow} and \refCom{underleftrightarrow} give respectively :
%   $\overrightarrow{AB}$, $\overleftarrow{AB}$, $\overleftrightarrow{AB}$,
%   $\underrightarrow{AB}$, $\underleftarrow{AB}$ and $\underleftrightarrow{AB}$
% \end{docOption}
%
% \begin{docOption}{tikz}
%   Loads the package \pkg{tikz} with its library |arrows.meta|.
%
%   Note that TikZ arrows, drawn with the |tikz| method, are always available,
%   even if this option is not set, provided the \pkg{tikz} package and its
%   library are loaded independently.
% \end{docOption}
%
% \begin{docOption}{pstricks}
%   Loads the package \pkg{pstricks-add}.
%
%   Note that, as it, this will compile with \LaTeX{}, \LuaLaTeX{} and
%   \XeLaTeX{}, but not compile \makebox{pdf\LaTeX} (see the PSTricks
%   documentation). PSTricks arrows, drawn with the |pstricks| method, are
%   always available, even if this option is not set, provided the
%   \pkg{pstricks} package is loaded independently.
% \end{docOption}
%
% \begin{docOption}{pstarrows}
%   Loads the \pkg{pict2e} package, with its option |pstarrows|. Vectors using
%   \LaTeX{} |picture| environment gives then \pstarrows$\overpictarrow{AB}$
%   instead of \ltxarrows$\overpictarrow{AB}$.
%
%   Note that this affect all vectors drawn in \LaTeX{} |picture| environments,
%   and that this setting can be changed on the fly with the commands \cs{pstarrows}
%   and \cs{ltxarrows} from the \pkg{pict2e} package.
% \end{docOption}
%
% \begin{docOption}{subscripts}
%   Sets the default value of the key \refKey{detect subscripts} to |true|.
%
%   This option also impacts the command \refCom{vv} and all predefined
%   commands, so that they automatically use their starred variant when
%   a subscript follows.
% \end{docOption}
%
% \begin{docOption}{subother}
%   Sets to 12 (\emph{other} catcode category) the catcode of the ``|_|" symbol
%   used for subscript detection, when this is enabled by the key \refKey{detect
%   subscripts} (see the section~\ref{sec:non-standard-subscripts},
%   page~\pageref{sec:non-standard-subscripts}).
% \end{docOption}
%
% \begin{docOption}{subactive}
%   Sets to 13 (\emph{active} catcode category) the catcode of the ``|_|" symbol
%   used for subscript detection, when this is enabled by the key \refKey{detect
%   subscripts} (see the section~\ref{sec:non-standard-subscripts},
%   page~\pageref{sec:non-standard-subscripts}).
% \end{docOption}
%
% \begin{docOption}{debug}
%   Writes the meaning of defined commands in \LaTeX{} log.
% \end{docOption}
%
% \subsection{Commands}
%
% \subsubsection{Macro for commands creation}
% \label{sec:macro-for-commands-creation}
%
% \begin{docCommands}{%
%   {
%     doc name=NewOverArrowCommand,
%     doc parameter=\oarg{method}\marg{name}\marg{keys},
%   },
%   {
%     doc name=RenewOverArrowCommand,
%     doc parameter=\oarg{method}\marg{name}\marg{keys},
%   },
%   {
%     doc name=ProvideOverArrowCommand,
%     doc parameter=\oarg{method}\marg{name}\marg{keys},
%   },
%   {
%     doc name=DeclareOverArrowCommand,
%     doc parameter=\oarg{method}\marg{name}\marg{keys},
%   },
% }
%   Creates the command \cs{\meta{name}} and its starred variant
%   \cs{\meta{name}*}. The starred variant \cs{\meta{name}*} removes the extra
%   end space generated by the arrow, which is suitable, as example, when
%   a subscript follows.
%
% \begin{description}
%   \item[\cs{NewOverArrowCommand}] raises an error if \cs{\meta{name}} is
%   already defined.
%
%   \item[\cs{RenewOverArrowCommand}] raises an error if \cs{\meta{name}} is undefined.
%
%   \item[\cs{ProvideOverArrowCommand}] sets \cs{\meta{name}} if the command is
%   undefined and does nothing if it is already defined, without raising
%   any error.
%
%   \item[\cs{DeclareOverArrowCommand}] sets \cs{\meta{name}}, whether the
%   command is already defined
%   or not, without raising any error.
%   \end{description}
%
%   The \meta{method} used to draw the arrow must be:
%   \begin{description}
%    \item[|symb|] to draw the arrow by symbols assemblage (default);
%    \item[|tikz|] to draw the arrow with PGF/TikZ;
%    \item[|pstricks|] to draw the arrow with PSTricks;
%    \item[|picture|] to draw the arrow with the \LaTeX{}
%     |picture| environment.
%   \end{description}
%   With no \meta{method} argument, the |symb| method is chosen.
%
%   \smallskip
%
%   \meta{keys} is a comma-separated list of keys-values. Available keys depends
%   of the \meta{method} chosen and are described in section~\ref{sec:keys},
%   page~\pageref{sec:keys}.
%
%\iffalse
%<*example>
% \fi
\begin{dispExample}
\NewOverArrowCommand[tikz]{\myoverarrow}{arrows={Bar-Bar}, center arrow}
$ \myoverarrow{v} \qquad \myoverarrow{ABCD} $
\end{dispExample}
% \iffalse
%</example>
% \fi
% \end{docCommands}
%
% \begin{docCommands}{%
%   {
%     doc name=TestOverArrow,
%     doc parameter=\oarg{pattern}\marg{name},
%   },
%   {
%     doc name=TestOverArrow*,
%     doc parameter=\oarg{pattern}\marg{name},
%   },
% }
%   Displays the result of the command \meta{name} for patterns of various
%   lengths and for the four math styles. A custom \meta{pattern} can be added
%   to the predefined ones.
%
%   The starred variant \cs{TestOverArrow*} displays a full report, including
%   kerning tests of the commands \cs{\meta{name}} and \cs{\meta{name}*} .
%
% \iffalse
%<*example>
% \fi
\begin{dispExample}
\TestOverArrow*[my~pattern]{vv}
\end{dispExample}
% \iffalse
%</example>
% \fi
%
% \end{docCommands}
%
% \subsubsection{Useful macros for symbols assemblage}
%
% Math symbols assemblage is the default method used to draw arrows. The macros
% \refCom*{xjoinrel} and \refCom*{smallermathstyle} are designed to help combine
% and format math symbols.
%
% \begin{docCommand}{xjoinrel}{\oarg{number}}
%   Removes an horizontal space of \meta{number} math units (|3.5 mu| by
%   default). Must be used in math mode. Useful to assemble math symbols and
%   create new ones.
%
% \iffalse
%<*example>
% \fi
\begin{dispExample}
\newcommand*{\triplebar}{\Relbar\xjoinrel[14]\relbar}
\newcommand*{\triplebararrow}{\Relbar\xjoinrel[15]\rightarrow}
\scalebox{2}{$ \triplebar \quad \triplebararrow $} \par
\scalebox{2}{$ \triplebar\xjoinrel\triplebararrow $}
\end{dispExample}
% \iffalse
%</example>
% \fi
% \end{docCommand}
%
% \begin{docCommand}{smallermathstyle}{}
%   Applies the next math style, smaller than the current. That is:
% \begin{itemize}
%     \item sets \cs{scriptstyle} if the current math style is \cs{displaystyle}
%           or \cs{textstyle};
%     \item sets \cs{scriptscriptstyle} if the current math style is
%           \cs{scriptstyle};
%     \item does nothing if the current math style is \cs{scriptscriptstyle}.
% \end{itemize}
%
%\iffalse
%<*example>
% \fi
\begin{dispExample}
$ \displaystyle AB \quad \textstyle AB
 \quad \scriptstyle AB \quad \scriptscriptstyle AB $\par
$ \displaystyle AB \quad \smallermathstyle AB
 \quad \smallermathstyle AB \quad \smallermathstyle AB $
\end{dispExample}
% \iffalse
%</example>
% \fi
%
% \end{docCommand}
%
% \subsubsection{Useful lengths for TikZ, PSTricks or \texttt{picture} environment}
%
% Arrows drawn with graphic languages, like PGF/TikZ, PSTricks or the \LaTeX{}
% |picture| environment, are not extensible. The three lengths
% \refLen*{overarrowlength}, \refLen*{overarrowthickness} and
% \refLen*{overarrowsmallerthickness} are computed at each utilisation of
% a command set with the |tikz|, |pstricks| or |picture| method, so they can be
% used in drawing commands.
%
% \iffalse
%<*example>
% \fi
\begin{dispExample*}{grow to left by=-2em}
\NewOverArrowCommand[tikz]{\overparabola}{%
  path options={x=\overarrowlength, line width=\overarrowsmallerthickness},
  path={(0,0) parabola[parabola height=0.2\overarrowlength] (1,0)},
  arrows={-}, center arrow, min length=30,
}
$\displaystyle \overparabola{v} \qquad \overparabola{ABCD} $ \par
$\scriptstyle \overparabola{v} \qquad \overparabola{ABCD} $ \par
\end{dispExample*}
% \iffalse
%</example>
% \fi
%
% \begin{docLength}{overarrowlength}
%   Is set to the width of the arrow command content, or, if larger, to the
%   minimal arrow length set through the key \refKey{min length}.
% \end{docLength}
%
% \begin{docLength}{overarrowthickness}
%   Is set to the default rule thickness of the current math style. That is:
%   \begin{itemize}
%    \item |\fontdimen 8 \textfont 3| in \cs{displaystyle} or \cs{textstyle};
%    \item |\fontdimen 8 \scriptfont 3| in \cs{scriptstyle};
%    \item |\fontdimen 8 \scriptscriptfont 3| in
%     \cs{scriptscriptstyle}.
%   \end{itemize}
%
%   Theses settings are adapted when the package \pkg{unicode-math} is loaded
%   (using \cs{Umathoverbarrule} with \LuaLaTeX{} or |\fontdimen 54|, family~2
%   with \XeLaTeX{} --- see the manual of \pkg{unicode-math}).
% \end{docLength}
%   
%  \begin{docLength}{overarrowsmallerthickness}
%    Is set to the default rule thickness of the next smaller math style.
%    That is:
%    \begin{itemize}
%     \item |\fontdimen 8 \scriptfont 3| in \cs{displaystyle} or \cs{textstyle};
%     \item |\fontdimen 8 \scriptscriptfont 3| in \cs{scriptstyle} or
%      \cs{scriptscriptstyle}.
%    \end{itemize}
%
%   Theses settings are adapted when the package \pkg{unicode-math} is loaded
%   (using \cs{Umathoverbarrule} with \LuaLaTeX{} or |\fontdimen 54|, family~2
%   with \XeLaTeX{} --- see the manual of \pkg{unicode-math}).
%  \end{docLength}
%
% \subsubsection{Vectors macros}
%
% The macro \cs{vv}, dedicated to vectors, is automatically defined when the
% option \refOpt{esvect} is set (which is the default). It is a clone of the
% \cs{vv} command provided by the \pkg{esvect} package, but its starred variant
% has a correct kerning when followed by a subscript.
%
% \begin{docCommands}{%
%   {
%     doc name=vv,
%     doc parameter=\marg{content},
%   },
%   {
%     doc name=vv*,
%     doc parameter=\marg{content},
%   },
% }
%   Draws a vector arrow upon math \meta{content}. The shape of the arrow
%   depends on the corresponding options described in
%   section~\ref{sec:esvect-conf}, page~\pageref{sec:esvect-conf} :
%   \refOpt{esvecta}, \refOpt{esvectb}, \refOpt{esvectc}, \refOpt{esvectd},
%   \refOpt{esvecte}, \refOpt{esvectf}, \refOpt{esvectg}, \refOpt{esvecth}.
%
%   The starred variant \cs{vv*} suppresses the end space created by the arrow.
%
% \iffalse
%<*example>
% \fi
\begin{dispExample}
$ \vv{\imath}_{0}  \quad \vv{e}_r \quad \vv{L}_\Delta $\par
$ \vv*{\imath}_{0}  \quad \vv*{e}_r \quad \vv*{L}_\Delta $
\end{dispExample}
% \iffalse
%</example>
% \fi
%
%\end{docCommands}
%
% \begin{docCommand}{esvectvv}{}
%   Is simply the backup of the original \pkg{esvect} \cs{vv} command.
% \iffalse
%<*example>
% \fi
\begin{dispExample}
$ \esvectvv{\imath}_{0}  \quad \esvectvv{e}_{r} \quad \esvectvv{L}_\Delta $\par
$ \esvectvv*{\imath}{0}  \quad \esvectvv*{e}{r} \quad \esvectvv*{L}{\Delta} $
\end{dispExample}
% \iffalse
%</example>
% \fi
%
% \end{docCommand}
%
% \subsubsection{Predefined commands}
% \label{sec:predefined-commands}
%
% Predefined commands are defined if the corresponding option is set (see
% section~\ref{sec:predefined-commands-opt},
% page~\pageref{sec:predefined-commands-opt}). The commands
% \refCom*{overrightarrow}, \refCom*{overleftarrow},
% \refCom*{overleftrightarrow}, \refCom*{underrightarrow},
% \refCom*{underleftarrow} and \refCom*{underleftrightarrow} are affected by the
% option \refOpt{old-arrows}.
%
% \medskip
%
% \noindent\textbf{Over arrows}
%
%  \begin{docCommand}{overrightarrow}{}
%    $$\overrightarrow{v} \qquad \overrightarrow{AB} \qquad \overrightarrow{\mathrm{grad}}$$
%
%   The shape of the arrow is smaller if the option \refOpt{old-arrows} is set.
%  \end{docCommand}
%
%  \begin{docCommand}{overleftarrow}{}
%    $$\overleftarrow{v} \qquad \overleftarrow{AB} \qquad \overleftarrow{\mathrm{grad}}$$
%
%   The shape of the arrow is smaller if the option \refOpt{old-arrows} is set.
%  \end{docCommand}
%
%  \begin{docCommand}{overleftrightarrow}{}
%  $$\overleftrightarrow{v} \qquad \overleftrightarrow{AB} \qquad\overleftrightarrow{\mathrm{grad}}$$
%
%   The shape of the arrows is smaller if the option \refOpt{old-arrows} is set.%
%  \end{docCommand}
%
%\begin{docCommand}{overrightharpoonup}{}
%  $$\overrightharpoonup{v} \qquad \overrightharpoonup{AB} \qquad \overrightharpoonup{\mathrm{grad}}$$
%  \end{docCommand}
%
%  \begin{docCommand}{overrightharpoondown}{}
%  $$\overrightharpoondown{v} \qquad \overrightharpoondown{AB} \qquad \overrightharpoondown{\mathrm{grad}}$$
%  \end{docCommand}
%
%  \begin{docCommand}{overleftharpoonup}{}
%    $$\overleftharpoonup{v} \qquad \overleftharpoonup{AB} \qquad \overleftharpoonup{\mathrm{grad}}$$
%  \end{docCommand}
%
%  \begin{docCommand}{overleftharpoondown}{}
%  $$\overleftharpoondown{v} \qquad \overleftharpoondown{AB} \qquad \overleftharpoondown{\mathrm{grad}}$$
%  \end{docCommand}
%
%  \begin{docCommand}{overbar}{}
%  $$\overbar{v} \qquad \overbar{AB} \qquad \overbar{\mathrm{grad}}$$
%  \end{docCommand}
%
%
% \medskip
%
% \noindent\textbf{Under arrows}
%
%  \begin{docCommand}{underrightarrow}{}
%    $$\underrightarrow{v} \qquad \underrightarrow{AB} \qquad \underrightarrow{\mathrm{grad}}$$
%
%   The shape of the arrow is smaller if the option \refOpt{old-arrows} is set.
%  \end{docCommand}
%
%  \begin{docCommand}{underleftarrow}{}
%    $$\underleftarrow{v} \qquad \underleftarrow{AB} \qquad \underleftarrow{\mathrm{grad}}$$
%
%   The shape of the arrow is smaller if the option \refOpt{old-arrows} is set.
%  \end{docCommand}
%
%  \begin{docCommand}{underleftrightarrow}{}
%  $$\underleftrightarrow{v} \qquad \underleftrightarrow{AB} \qquad \underleftrightarrow{\mathrm{grad}}$$
%
%   The shape of the arrows is smaller if the option \refOpt{old-arrows} is set.
%  \end{docCommand}
%
%  \begin{docCommand}{underrightharpoonup}{}
%  $$\underrightharpoonup{v} \qquad \underrightharpoonup{AB} \qquad \underrightharpoonup{\mathrm{grad}}$$
%  \end{docCommand}
%
%  \begin{docCommand}{underrightharpoondown}{}
%  $$\underrightharpoondown{v} \qquad \underrightharpoondown{AB} \qquad \underrightharpoondown{\mathrm{grad}}$$
%  \end{docCommand}
%
%  \begin{docCommand}{underleftharpoonup}{}
%  $$\underleftharpoonup{v} \qquad \underleftharpoonup{AB} \qquad \underleftharpoonup{\mathrm{grad}}$$
%  \end{docCommand}
%
%  \begin{docCommand}{underleftharpoondown}{}
%  $$\underleftharpoondown{v} \qquad \underleftharpoondown{AB} \qquad \underleftharpoondown{\mathrm{grad}}$$
%  \end{docCommand}
%
%  \begin{docCommand}{underbar}{}
%  $$\underbar{v} \qquad \underbar{AB} \qquad \underbar{\mathrm{grad}}$$
%  \end{docCommand}
%
% \subsection{Keys}
%\label{sec:keys}
%
% The customisation of arrows is done at command creation through a key-value
% interface provided by the \pkg{pgfkeys} package (with |/overarrows/| as key
% path).
%
% \subsubsection{Arrow position and length settings}
%
% These keys are available whatever the method chosen at command creation (see
% section~\ref{sec:macro-for-commands-creation},
% page~\pageref{sec:macro-for-commands-creation} for the documentation of
% commands creation).
%
% \medskip\par\noindent\textbf{Length}
%
% \begin{docKey}{min length}{=\marg{number}}
% {no default, see below for the initial value}
% Sets the minimal arrow length to \meta{number} math units. The arrow length is
% set from content width, or, if larger, to this value.
%
% The initial value of \refKey*{min length} depends on the \meta{method} chosen
% at command creation (see section~\ref{sec:macro-for-commands-creation},
% page~\pageref{sec:macro-for-commands-creation} for the documentation of
% commands creation):
% \begin{itemize}
%  \item \meta{number}| = 0 | for the |symb| method (default);
%  \item \meta{number}| = 12| for the |tikz| method;
%  \item \meta{number}| = 12| for the |pstricks| method;
%  \item \meta{number}| = 18| for the |picture| method.
% \end{itemize}
%
%\iffalse
%<*example>
% \fi
\begin{dispExample}
\NewOverArrowCommand{\overlongarrow}{min length=50}
$ \overlongarrow{v} \qquad \overlongarrow{ABCDEF} $
\end{dispExample}
% \iffalse
%</example>
% \fi
%\end{docKey}
%
% \medskip\par\noindent\textbf{Placement}
%
% \begin{docKeys}{
%   {
%       doc name={arrow under},
%       doc description={default |autoconfig|, initially unset},
%   },
%   {
%       doc name={arrow under},
%       doc parameter={=autoconfig\textbar noconfig},
%       doc no index,
%   }
% }
% Places the arrow under, instead of over.
% \begin{description}
%  \item[\refKey*{arrow under} or \refKey*{arrow under}|=autoconfig|] also
%   configures suitably the key \refKey{detect subscripts} to |false| and the
%   key \refKey{before arrow} to get an additional space over the arrow.
%  \item[\refKey*{arrow under}|=noconfig|] does not do any
%   additional configuration.
% \end{description}
%
% \iffalse
%<*example>
% \fi
\begin{dispExample}
\NewOverArrowCommand{\underhooks}{%
  start={\lhook}, end={\rhook},  trim=1,
  arrow under, shift leftright=-4,
}
$ \underhooks{v} \qquad \underhooks{AB} $
\end{dispExample}
% \iffalse
%</example>
% \fi
%
% \end{docKeys}
%
% \medskip\par\noindent\textbf{Horizontal shifts}
%
% \begin{docKey}{shift left}{=\marg{number}}
% {no default, initially |2|}
% Shifts the left side of the arrow by \meta{number} math units (positive number
% means a shift to the right).
%\end{docKey}
%
% \begin{docKey}{shift right}{=\marg{number}}
% {no default, see below for the initial value}
% Shifts the right side of the arrow by \meta{number} math units (positive
% number means a shift to the left).
%
% The initial value of \refKey*{shift right} depends on the \meta{method} chosen
% at command creation (see section~\ref{sec:macro-for-commands-creation},
% page~\pageref{sec:macro-for-commands-creation} for the documentation of
% commands creation):
% \begin{itemize}
%  \item \meta{number}| = 0 | for the |symb| method  (default);
%  \item \meta{number}| = -2| for the |tikz|, |pstricks| and |picture| methods.
% \end{itemize}
%
%\end{docKey}
%
%\iffalse
%<*example>
% \fi
\begin{dispExample*}{grow to left by=-2em}
\NewOverArrowCommand{\lookback}{%
  start={\leftarrow}, end={\rightharpoondown},
  shift left=-50, shift right=-10,
}
$ \lookback{\text{look back}} $
\end{dispExample*}
% \iffalse
%</example>
% \fi
%
%
% \begin{docKey}{shift leftright}{=\oarg{number}}{no default}
%   Sets \refKey{shift left} and \refKey{shift right} to the same
%   \meta{number} value.
%\end{docKey}
%
%
% \begin{docKey}{center arrow}{}{}
%   Sets \refKey{shift left} and \refKey{shift right} to zero.
%\end{docKey}
%
% \begin{docKey}{left arrow}{}{default |2|}
%   Sets \refKey{shift left} to zero and \refKey{shift right} to \meta{number}.
% \end{docKey}
%
% \begin{docKey}{right arrow}{}{default |2|}
%   Sets \refKey{shift right} to zero and \refKey{shift left} to \meta{number}.
% \end{docKey}
%
% \medskip\par\noindent\textbf{Vertical adjunct}
%
% \begin{docKeys}{
%   {
%       doc name={before arrow},
%       doc parameter={=\marg{vertical material}},
%       doc description={initially empty},
%   },
%   {
%       doc name={after arrow},
%       doc parameter={=\marg{vertical material}},
%       doc description={initially empty},
%   },
% }
%   Adds the \meta{vertical material} before or after the arrow.
%
%   Over and under arrow commands are typeset through the \TeX{} \cs{ialign}
%   command, which aligns contents, like a tabular. The \meta{vertical material}
%   is inserted \emph{between} the rows, with \TeX{} \cs{noalign} command.
%
%   These keys are essentially used to add some extra space between the arrow
%   and the content of the command. They can be set in a handier way with the
%   keys \refKey{space before arrow} and \refKey{space after arrow}.
% \end{docKeys}
%
% \begin{docKey}{space before arrow}{=\marg{length}}{no default}
%   Adds a space of \meta{length} before the arrow. This sets the keys
%   \refKey{before arrow}.
% \end{docKey}
%
% \begin{docKey}{space after arrow}{=\marg{length}}{no default}
%   Adds a space of \meta{length} after the arrow. This sets the keys
%   \refKey{after arrow}.
% \end{docKey}
%
% \iffalse
%<*example>
% \fi
\begin{dispExample*}{grow to left by=-2em}
\NewOverArrowCommand{\overharpoonsdown}{%
  start=\leftharpoondown, end=\rightharpoondown, center arrow,
  space before arrow=-0.2ex, space after arrow=0.3ex,
}
$ \dot{\overharpoonsdown{v}} \qquad \ddot{\overharpoonsdown{AB}}$
\end{dispExample*}
% \iffalse
%</example>
% \fi
%
% \subsubsection{Subscripts detection setting}
%
% This key is available whatever the method chosen at command creation (see
% section~\ref{sec:macro-for-commands-creation},
% page~\pageref{sec:macro-for-commands-creation} for the documentation of
% commands creation).
%
% \begin{docKey}{detect subscripts}{=true\textbar{}false}
% {default |true|, see below for the initial value}
%
% Removes automatically the extra end space created by the arrow, if a subscript
% immediately follows the command.
%
% By default, the initial value of \refKey*{detect subscripts} is |false|.
% When the option \refOpt{subscripts} is set, the initial value of
% \refKey*{detect subscripts} is |true|.
%
% Note that the detection may fail when the standard subscript command is
% changed or altered (see the section~\ref{sec:non-standard-subscripts},
% page~\pageref{sec:non-standard-subscripts}).
%
% \iffalse
%<*example>
% \fi
\begin{dispExample}
\NewOverArrowCommand{\autosub}{detect subscripts}
$ \imath_0 \qquad \autosub{\imath}_0 \qquad
 {\autosub{\imath}}_0 \qquad {\autosub*{\imath}}_0 $
\end{dispExample}
% \iffalse
%</example>
% \fi
%
%\end{docKey}
%
% \subsubsection{Symbols assemblage settings}
%
% The following keys are available for arrows drawn with the default |symb|
% method (see section~\ref{sec:macro-for-commands-creation},
% page~\pageref{sec:macro-for-commands-creation} for the documentation of
% commands creation).
%
% \begin{docKeys}{
%   {
%       doc name={start},
%       doc parameter={=\marg{command}},
%       doc description={no default, initially \cs{relbar}},
%   },
%   {
%       doc name={middle},
%       doc parameter={=\marg{command}},
%       doc description={no default, initially set by \refKey*{middle config}|=auto|},
%   },
%   {
%       doc name={end},
%       doc parameter={=\marg{command}},
%       doc description={no default, see below for the initial value},
%   },
% }
%   Sets the \meta{command} used to draw the start (left), middle (center) or
%   end (right) part of the arrow. The \refKey*{middle} one is repeated, if
%   necessary, to extend the arrow. It is set, initially by \refKey*{middle
%   config}|=auto|.
%   By default, the \refKey*{end} symbols is initially \cs{rightarrow}
%   \symb{\rightarrow}. When the option \refOpt{old-arrows} is set, the initial
%   value of \refKey*{end} is \cs{varrightarrow} \symb{\varrightarrow}.
%
%   \refKey*{start} and \refKey*{end} symbols are typeset in the same group.
%   \refKey*{middle} is typeset alone. This means that, if a command, like
%   \refCom{smallermathstyle}, is used to alter the symbols, it should be
%   applied both to \refKey*{start} and \refKey*{middle} (but not to
%   \refKey*{end}).
%
%\iffalse
%<*example>
% \fi
\begin{dispExample}
\NewOverArrowCommand{\smalleroverrightarrow}{%
  start={\smallermathstyle\relbar},
  middle={\smallermathstyle\relbareda},
  end={\rightarrow},
  space after arrow={0.2ex},
}
$ \smalleroverrightarrow{v} \qquad \smalleroverrightarrow{AB} $
\end{dispExample}
% \iffalse
%</example>
% \fi
%\end{docKeys}
%
%
%\begin{docKey}{trim start}{=\marg{number}}{no default, initially |7|}
%  Trims \meta{number} math units from the right side of the
%  \refKey{start} symbol.
% \end{docKey}
%
% \begin{docKey}{trim middle}{=\marg{number}}
% {no default, initially set by \refKey*{middle config}|=auto|}
%   Trims \meta{number} math units from both left and right sides of the
%   \refKey{middle} symbol.
% \end{docKey}
%
% \begin{docKey}{trim end}{=\marg{number}}{no default, initially |7|}
%   Trims \meta{number} math units from the left side of the
%   \refKey{end} symbol.
% \end{docKey}
%
% \begin{docKey}{trim}{=\marg{number}}{no default}
%   Sets \refKey{trim start}, \refKey{trim middle} and \refKey{trim end} to the
%   same \meta{number} value.
% \end{docKey}
%
% \begin{docKey}{no trimming}{}{}
%   Clears \refKey{trim start}, \refKey{trim middle} and \refKey{trim end}.
% \end{docKey}
%
%\begin{docKey}{middle config}{=auto\textbar relbar\textbar relbareda \textbar harrowextender}{no default}
%  Sets a suitable configuration for the keys \refKey{middle} and \refKey{trim middle}:
%  \begin{description}
%   \item[For \refKey*{middle config}| = relbar|,] \refKey{middle} is set to
%    \cs{relbar} \symb{\relbar} and \refKey{trim middle} to |2.5|.
%   \item[For \refKey*{middle config}| = relbareda|,] \refKey{middle} is set to
%    \cs{relbareda} \symb{\relbareda} and \refKey{trim middle} to |1|.
%   \item[For \refKey*{middle config}| = harrowextender|,] \refKey{middle} is
%   set to \cs{harrowextender} and \refKey{trim middle} to |0|.
%    \item[For \refKey*{middle config}| = auto|,] if \cs{harrowextender} is
%      provided by the math font\footnote{See the documentation of the package
%      \pkg{unicode-math}.}, \refKey{middle} is set with \refKey*{middle
%      config}| = harrowextender|. If \cs{harrowextender} isn't availlable,
%      \refKey{middle} is set with \refKey*{middle config}| = relbareda| if the
%      option \refOpt{esvect} is set (which is the default) and \refKey*{middle
%      config}| = relabar| if not.
%   \end{description}
%\end{docKey}
%
% \begin{docKeys}{
%   {
%       doc name={amsmath},
%       doc description={default |mimic|},
%   },
%   {
%       doc name={amsmath},
%       doc parameter={=mimic\textbar strict},
%       doc no index,
%   }
% }
%   Loads a configuration coherent with \pkg{amsmath}
%   \cs{overrightarrow} command.
% \begin{description}
%  \item[\refKey*{amsmath} or \refKey*{amsmath}|=mimic|] sets the corresponding
%   keys suitably:
%   \begin{center}
%     \begin{tabular*}{\linewidth}{ @{\extracolsep{\fill}}lll }
%     \refKey*{start}|={\relbar}|
%     & \refKey*{middle}=|{\relbar}|
%     & \refKey*{end}|={\rightarrow}| \\
%     \refKey*{trim start}|=7|
%     & \refKey*{trim middle}|=2|
%     & \refKey*{trim end}|=7| \\
%     \refKey*{shift leftright}|=0|
%     & \refKey*{after arrow}|={}|
%     & \refKey*{before arrow}|={}|
%       \end{tabular*}
%   \end{center}
%  \item[\refKey*{amsmath}|=strict|] makes, in addition, the command uses the
%  internal macros of  \pkg{amsmath}
%  \cs{overrightarrow} (\refKey*{no trimming}, \refKey*{fill
%  macro}|={\arrowfill@}|, \refKey*{stack macro}|={\overarrow@}|). Note that
%  many configuration keys becomes ineffective.
% \end{description}
% \end{docKeys}
%
% \begin{docKeys}{
%   {
%       doc name={esvect},
%       doc description={default |mimic|},
%   },
%   {
%       doc name={esvect},
%       doc parameter={=mimic\textbar strict},
%       doc no index,
%   }
% }
%   Loads a configuration coherent with \pkg{amsmath}
%   \cs{vv} command.
% \begin{description}
%  \item[\refKey*{esvect} or \refKey*{esvect}|=mimic|] sets the corresponding
%   keys suitably:
%   \begin{center}
%     \begin{tabular*}{\linewidth}{ @{\extracolsep{\fill}}lll }
%     \refKey*{start}|={\relbaredd}|
%     & \refKey*{middle}=|{\relbareda}|
%     & \refKey*{end}|={\fldr}| \\
%     \refKey*{trim start}|=1.5|
%     & \refKey*{trim middle}|=0|
%     & \refKey*{trim end}|=1.5| \\
%     \refKey*{space before arrow}|=-.7pt|
%     & \refKey*{space after arrow}|=-.3pt|
%     &\refKey*{right arrow}|=2|
%       \end{tabular*}
%   \end{center}
%  \item[\refKey*{esvect}|=strict|] makes, in addition, the command uses the
%  internal macros of  \pkg{esvect}
%  \cs{vv} (\refKey*{no trimming}, \refKey*{fill
%  macro}|={\traitfill@}|, \refKey*{stack macro}|={\overvect@}|). Note that
%  many configuration keys becomes ineffective.
% \end{description}
%
%\end{docKeys}
%
% \subsubsection{TikZ settings}
% \label{sec:tikz-keys}
%
% If, at command creation (see section~\ref{sec:macro-for-commands-creation},
% page~\pageref{sec:macro-for-commands-creation} for the documentation of
% commands creation), the |tikz| method is chosen, then the arrow is drawn by
% the command:
% \begin{center}
%   \cs{tikz}|[|\refKey*{tikz options}|]{|\refKey*{tikz command}|}|
% \end{center}
% \noindent where \refKey{tikz options} and \refKey{tikz command} are two keys
% described below. When \refKey*{tikz command} is let unset, the drawing command
% turns into:
% \begin{center}
%   \cs{tikz}|[|\refKey*{tikz options}|]{|\cs{draw}|[|\refKey*{path options}|]|
%   \refKey*{path}|;}|
% \end{center}
%
% The best way to customise |tikz| arrows is then to set the keys \refKey{tikz
% options}, \refKey{path options} and \refKey{path}, preferably through the
% handy alternatives: \refKey{add tikz options}, \refKey{add path options},
% \refKey{arrows}, \refKey{line thickness} or \refKey{thinner}.
%
%\iffalse
%<*example>
% \fi
\begin{dispExample*}{grow to left by=-2em}
\NewOverArrowCommand[tikz]{\overdotteddoublearrow}{%
  add tikz options={blue}, add path options={densely dotted},
  arrows={->[scale=0.5]>[scale=0.5]}, thinner,
  min length=20, space after arrow={0.3ex},
}
$ \overdotteddoublearrow{v} \qquad \overdotteddoublearrow{AB} $
\end{dispExample*}
% \iffalse
%</example>
% \fi
%
% \bigskip
%
% The following keys are available when the |tikz| method is chosen.
%
% \begin{docKey}[][before lower=\bigskip\par]
% {tikz options}{=\marg{TikZ options}}
% {no default, initially |x=|\cs{overarrowlength}|, line
% width=|\cs{overarrowthickness}}
%   Sets TikZ options to \meta{TikZ options}.
%\end{docKey}
%
% \begin{docKey}[][before lower=\bigskip\par]
% {path options}{=\marg{path options}}
% {no default, initially |arrows={-Classical TikZ Rightarrow}, cap=round|}
%   Sets TikZ path options to \meta{path options}.
% \end{docKey}
% \begin{docKey}{path}{=\marg{path specification}}
% {no default, initially |(0,0)--(1,0)|}
% Sets TikZ path specification to \meta{path} (the ending semicolon is
% automatically appended).
%\end{docKey}
%
% \begin{docKey}{add tikz options}{=\marg{TikZ options}}{no default}
%   Appends the options \meta{TikZ options} to the key \refKey{tikz options}.
% \end{docKey}
%
% \begin{docKey}{add path options}{=\marg{path options}}{no default}
%   Appends the options \meta{path options} to the key \refKey{path options}.
% \end{docKey}
%
% \begin{docKey}{arrows}{=\marg{arrow specification}}{no default}
%   Appends the option |arrows={|\meta{arrow specification}|}| to the key
%   \refKey{path options}.
% \end{docKey}
%
% \begin{docKey}{line thickness}{=\marg{length}}{no default}
%   Appends the option |line width={|\meta{length}|}| to the key \refKey{path
%   options}.
% \end{docKey}
%
% \begin{docKey}{thinner}{}{}
%   Sets the keys \refKey{line thickness} with \cs{overarrowsmallerthickness}.
% \end{docKey}
%
% \begin{docKey}{tikz command}{=\marg{TikZ command}}{initially unset}
%   Sets the \meta{TikZ command} used to draw the arrow. If left unset, the
%   value \cs{draw}|[|\refKey*{path options}|] |\refKey*{path}|;| is used.
%\end{docKey}
%
% \subsubsection{PSTricks settings}
% \label{sec:pstricks-keys}
%
% If, at command creation (see section~\ref{sec:macro-for-commands-creation},
% page~\pageref{sec:macro-for-commands-creation} for the documentation of
% commands creation), the |pstricks| method is chosen, then the arrow is drawn
% with by:
% \begin{flushleft}\leftskip=3cm
%   \cs{begin}|{pspicture}|\refDoc{key:pstricks}*{geometry}|%|\\
%    \quad\cs{psset}|{linewidth=|\refDoc{key:pstricks}*{line thickness}|}%|\\
%    \quad\cs{psset}|{|\refKey*{psset}|}%|\\
%    \quad\refKey*{pstricks command}|%|\par
%   \cs{end}|{pspicture}%|
% \end{flushleft}
% \noindent where \refDoc{key:pstricks}{geometry}, \refDoc{key:pstricks}{line
% thickness} \refKey*{psset} and \refKey*{pstricks command} are four keys
% described below.
%
%\iffalse
%<*example>
% \fi
\begin{dispExample*}{grow to left by=-2em}
\NewOverArrowCommand[pstricks]{\overloopandarrow}{
  pstricks command={%
    \pscurve{->}(0,0)
    (0.6\overarrowlength,0.05\overarrowlength)
    (0.5\overarrowlength,0.1\overarrowlength)
    (0.4\overarrowlength,0.05\overarrowlength)
    (\overarrowlength,0)
  },
  geometry={(0,0)(\overarrowlength,0.2\overarrowlength)},
  space after arrow=2pt, min length=20,
  geometry={(0,0)(\overarrowlength,0.2\overarrowlength)},
}
$ \overloopandarrow{v} \qquad \overloopandarrow{AB} $
\end{dispExample*}
% \iffalse
%</example>
% \fi
%
% \bigskip
%
% The following keys are available when the |pstricks| method is chosen.
%
% \begin{docKey}[][before lower=\bigskip\par]
% {pstricks command}{=\marg{pstricks command}}
% {no default, initially \cs{psline}|\{->\}(0,0)(|\refCom*{overarrowlength}|,0)|}
%  Sets the |pspicture| command to \meta{pstricks command}.
% \end{docKey}
%
% \begin{docKey}[][doc label={pstricks:arrow}]%^^A same name as for TikZ
% {arrow}{=\marg{arrow}}
% {no default, initially |->|}
% Sets \refKey{pstricks command} with
% |\psline{|\meta{arrow}|}(0,0)(|\refCom*{overarrowlength}|,0)|.
% \end{docKey}
%
% \begin{docKey}[][]
% {psset}{=\marg{pstricks setting}}
% {no default, initially empty}
%  Sets \meta{pstricks setting} with \cs{psset}.
% \end{docKey}
%
% \begin{docKey}[][before lower=\bigskip\par, doc label={pstricks:geometry}]%^^A same name as for picture
% {geometry}{=\marg{pstricks geometry specification}}
% {no default, initially |(0,-0.5ex)(|\refCom*{overarrowlength}|,1ex)|}
%   Sets the |pspicture| geometry to \meta{pstricks geometry specification}.
% \end{docKey}
%
% \begin{docKey}[][doc label={pstricks:line thickness}]%^^A same name as for TikZ
% {line thickness}{=\marg{length}}{no default}
% Sets the line thickness to \meta{length}.
% \end{docKey}
%
% \begin{docKey}[][doc label={pstricks:thinner}]%^^A same name as for TikZ
% {thinner}{}{}
% Sets the keys \refDoc{key:pstricks}{line thickness} with
% \cs{overarrowsmallerthickness}.
% \end{docKey}
%
% \subsubsection{Picture environment settings}
% \label{sec:picture-keys}
%
% If, at command creation (see section~\ref{sec:macro-for-commands-creation},
% page~\pageref{sec:macro-for-commands-creation} for the documentation of
% commands creation), the |picture| method is chosen, then the arrow is drawn
% with by:
% \begin{flushleft}\leftskip=3cm
%   \cs{begin}|{picture}|\refDoc{key:picture}*{geometry}|%|\\
%    \quad\cs{linethickness}|{|\refDoc{key:picture}*{line thickness}|}%|\\
%    \quad\refKey*{picture command}|%|\par
%   \cs{end}|{picture}%|
% \end{flushleft}
% \noindent where \refDoc{key:picture}{geometry}, \refDoc{key:picture}{line
% thickness} and \refKey*{picture command} are three keys described below.
%
%\iffalse
%<*example>
% \fi
\begin{dispExample*}{grow to left by=-2em}
  % ^^A \arc and \roundcap commands are from the pict2e package
  % ^^A this example needs \usepackage{pict2e} in the preamble
  \NewOverArrowCommand[picture]{\overarc}{%
    picture command={%
      \roundcap
      \put(0.5\overarrowlength,0){\arc[180,0]{0.6\overarrowlength}}
    },
    geometry={%
      (1.2\overarrowlength,0.5\overarrowlength)(-0.1\overarrowlength,0.2ex)
    },
    thinner, center arrow,
  }
  $ \overarc{v} \qquad \overarc{AB} $
\end{dispExample*}
% \iffalse
%</example>
% \fi
%
% \bigskip
% 
%The following keys are available when the |picture| method is chosen.
%
% \begin{docKey}[][before lower=\bigskip\par]
% {picture command}{=\marg{picture command}}
% {no default, initially \cs{put}|(0,0)\{|\cs{vector}|(1,0)\{|\refCom*{overarrowlength}|\}\}|}
%   Sets picture command to \meta{picture command}.
% \end{docKey}
%
% \begin{docKey}[][before lower=\bigskip\par, doc label={picture:geometry}]%^^A same name as for pstricks
% {geometry}{=\marg{picture geometry specification}}
% {no default, initially |(|\refCom*{overarrowlength}|,1ex)(0,-0.5ex)|}
%   Sets picture geometry to \meta{picture geometry specification}.
% \end{docKey}
%
% \begin{docKey}[][doc label={picture:line thickness}]%^^A same name as for TikZ
% {line thickness}{=\marg{length}}{no default}
% Sets the picture line thickness to \meta{length}.
% \end{docKey}
%
% \begin{docKey}[][doc label={picture:thinner}]%^^A same name as for TikZ
% {thinner}{}{no default}
% Sets the keys \refDoc{key:picture}{line thickness} with
% \cs{overarrowsmallerthickness}.
% \end{docKey}
%
% \subsection{Advanced commands and keys}
%
% The following commands and keys are used in the implementation of the
% \pkg{overarrows} package. They can also be employed for an advanced
% configuration of the commands created, although unnecessary in the vast
% majority of cases.
%
% \subsubsection{Advanced commands}
%
% \begin{docCommand}{SetOverArrowsSubscriptCommand}{\marg{command}}
%   Sets to \meta{command} the command used for subscript detection, when this
%   is enabled by the key \refKey{detect subscripts} (see the
%   section~\ref{sec:non-standard-subscripts},
%   page~\pageref{sec:non-standard-subscripts}).
% \end{docCommand}
%
% \begin{docCommands}{%
%   {
%     doc name=SetOverArrowsMethod,
%     doc parameter=\oarg{stack mechanism}\marg{name}\oarg{pre code}\marg{keys def},
%   },
%   {
%     doc name=SetOverArrowsMethod*,
%     doc parameter=\marg{name}\oarg{pre code}\marg{keys def},
%   },
% }
%   Defines the method \meta{name}, to be used in commands
%   \refCom{NewOverArrowCommand}, \refCom{RenewOverArrowCommand},
%   \refCom{ProvideOverArrowCommand} or \refCom{DeclareOverArrowCommand}.
%   When the \meta{name} method is chosen, corresponding keys are defined by
%   \meta{keys def}. This must set, in particular, the keys \refKey{no stack
%   macro hook} and \refKey{no arrow macro hook}. Optional code \meta{pre code}
%   is evaluated before the keys definition.
%
%   The unstarred variant automatically defines the key \refKey{no stack macro
%   hook}, according to the value of the optional \meta{stack mechanism}.
%   This one must be:
%   \begin{description}
%    \item[|fill|] if \refKey{arrow macro} creates extensible arrows (typically
%     with \cs{cleaders}). In this case, the arrow macro (defined by \refKey{no
%     arrow macro hook}) is called with the math style, passed as argument (it
%     can be, for example, the macro \cs{rightarrowfill@} used by \pkg{amsmath}
%     \cs{overrightarrow}). |fill| is the mechanism used by the |symb| method.
%    \item[|lens|] if \refKey{arrow macro} creates fixed-length arrows, and
%     needs the computation of lengths \refLen{overarrowlength}, \refLen{overarrowthickness}
%     and \refLen{overarrowsmallerthickness}. In this case, the arrow macro
%     (defined by \refKey{no arrow macro hook}) is called without argument.
%     |lens| is the mechanism used by the |tikz| and |picture| methods.
%   \end{description}
%   Without optional \meta{stack mechanism}, |fill| is used. The starred variant
%   does not set the key \refKey{no stack macro hook}.
% \end{docCommands}
%
% \subsubsection{Advanced keys}
%
% \begin{docKey}{stack macro}{=\marg{stack definition}}
% {no default, initially unset}
% Defines the stack macro to be \meta{stack definition}. Stack macro is
% a command which takes three arguments: the arrow macro set by \refKey{arrow
% macro}, the math style, and the command content (under or over the arrow).
% \meta{stack definition} can be, for example, the macro \cs{overarrow@} used by
% \pkg{amsmath} \cs{overrightarrow}.
% \end{docKey}
%
% \begin{docKey}{arrow macro}{=\marg{arrow definition}}
% {no default, initially unset}
% Defines the arrow macro (used in the stack macro) by to be \meta{arrow
% definition}.
% \end{docKey}
%
% \begin{docKey}{no stack macro hook}{=\marg{code}}{no default}
%   Sets the \meta{code} executed if \refKey{stack macro} is left unset, after
%   user evaluation of \meta{keys} in \refCom{NewOverArrowCommand},
%   \refCom{RenewOverArrowCommand}, \refCom{ProvideOverArrowCommand} or
%   \refCom{DeclareOverArrowCommand}.
%
%   \meta{code} must configure \refKey{stack macro} accordingly to the user
%   keys setting.
% \end{docKey}
%
% \begin{docKey}{no arrow macro hook}{=\marg{code}}{no default}
%   Sets the \meta{code} executed if \refKey{arrow macro} is left unset, after
%   user evaluation of \meta{keys} in \refCom{NewOverArrowCommand},
%   \refCom{RenewOverArrowCommand}, \refCom{ProvideOverArrowCommand} or
%   \refCom{DeclareOverArrowCommand}.
%
%   \meta{code} must configure \refKey{arrow macro} accordingly to the user
%   keys setting.
% \end{docKey}
%
% \begin{docKey}{fill macro}{=\marg{definition}}{no default, initially unset}
%   Defines the fill macro to be \meta{definition}. The fill macro is used by
%   arrows created with the |symb| method, to set \refKey{arrow macro} in
%   \refKey{no arrow macro hook}. It is called with fours arguments: start,
%   middle and end symbols used to draw the arrow, and the math style.
%   \meta{definition} can be, for example, the macro \cs{arrowfill@} used by
%   \pkg{amsmath} \cs{overrightarrow}.
% \end{docKey}
%
% \section{Complements}
%
% \subsection{Know issues}
%
% \subsubsection{Math font change}
%
% If the math font differs from the default \emph{Computer Modern}, arrow drawn
% with the |symb| method may have a central part of the arrow with inappropriate
% position or line width. This is
% because the default symbol used for the arrow line is \cs{relbareda}
% \symb{\relbareda} from the \pkg{esvect} package. This can be fixed with the
% \refOpt{noesvect} option.
%
% \subsubsection{Detection of non standard subscripts}
% \label{sec:non-standard-subscripts}
%
% The subscript detection enabled by the key \refKey{detect subscripts} is based
% on the \LaTeX{} macro \cs{@ifnextchar}. The detection may fail if the standard
% subscript command is modified of altered. This is the case, as example:
% \begin{itemize}
%  \item with the \pkg{spbmark} package
%   (\url{https://www.ctan.org/pkg/spbmark}), by Qu Yi, which allows a complete
%   customisation of subscripts, through the \cs{sub} command;
%  \item with the \pkg{altsubsup} package
%   (\url{https://www.ctan.org/pkg/altsubsup}), by Julien Labb\'e, which
%   provides an alternative subscript format, and changes, for this purpose, the
%   catcode of the underscore symbol ``|_|" from 8 (\emph{subscript} catcode
%   category) to 12 (\emph{other} catcode category).
% \end{itemize}
%
% To handle theses cases, the command used for subscript detection can be
% redefined with \refCom{SetOverArrowsSubscriptCommand}. Compatibility with the
% \pkg{spbmark} package is then obtained by:
% \iffalse
%<*example>
% \fi
\begin{dispListing}
\SetOverArrowsSubscriptCommand{\sub}
\end{dispListing}
% \iffalse
%</example>
% \fi
% \noindent In the same way, with the \pkg{altsubsup} package, add:
% \iffalse
%<*example>
% \fi
\begin{dispListing}
\SetOverArrowsSubscriptCommand{_}
\end{dispListing}
% \iffalse
%</example>
% \fi
% \noindent after the \cs{begin}|{document}| (namely, after the catcode
% redefinition done by \pkg{altsubsup}).
%
% \medskip
%
% Alternatively, two package options handle the cases where the catcode of the
% underscore ``|_|" symbol is changed: \refOpt{subother} (for catcode 12, or
% \emph{other}) and \refOpt{subactive} (for catcode 13, or \emph{active}).
% Hence, setting the \refOpt{subother} option is sufficient for compatibility
% with the \pkg{altsubsup} package (no need of
% \refCom{SetOverArrowsSubscriptCommand}). Note, that with options
% \refOpt{subother} and \refOpt{subactive}, the command \refCom{TestOverArrow*}
% may give bad results for kerning test, as defined before the
% catcode redefinition.
%
% \subsection{Package dependencies}
%
% The following packages are used by \pkg{overarrows}:
% \begin{itemize}
%  \item \pkg{amsmath}
%  \item \pkg{etoolbox}
%  \item \pkg{pgfkeys}
%  \item \pkg{esvect} (unless the option \refOpt{noesvect} is used)
%  \item \pkg{old-arrows} (when the option \refOpt{old-arrows} is used)
%  \item \pkg{tikz} (when the |tikz| method or the option \refOpt{tikz} is used)
%  \item \pkg{pict2e} (when the option \refOpt{pstarrows} is used)
% \end{itemize}
%
% \LaTeX{} distributions prior to 2020/10/01 must load the \pkg{xparse} package
% before \pkg{overarrows}.
%
% \subsection{Alternatives}
%
% \begin{description}
%  \item[\textsf{esvect} package]
%   (\url{https://www.ctan.org/pkg/esvect}), by Eddie Saudrais, provides the
%   fine vector macro \cs{vv}. This package is loaded by default by
%   \pkg{overarrows}.
%
%  \item[\pkg{letterswitharrows} package]
%   (\url{https://www.ctan.org/pkg/letterswitharrows}), by Max Teegen, provides
%   left and right over arrows commands, which can extend to
%   multiple characters.
%
%  \item[\pkg{overrightarrow} package]
%   (\url{https://www.ctan.org/pkg/overrightarrow}), by Robin Fairbairns,
%   provides the \cs{Overrightarrow} which is an amalgam of \cs{overrightarrow}
%   and \cs{Rightarrow}.
%
%  \item[\pkg{harpoon} package] (\url{https://ctan.org/pkg/harpoon}), by Tobias
%   Kuipers, provides over- and under-harpoon symbol commands.
% \end{description}
%
% \subsection{Changelog}
%
% \begin{description}
%  \item[v1.2]
%    \begin{itemize}
%      \item Fix compatibility issues with \pkg{unicode-math}.
%      \item Allow to draw the arrow with PSTricks.
%      \item Make \pkg{esvect} handle all font sizes.
%      \end{itemize}
%  \item[v1.1] Support for non-standard subscripts.
%  \item[v1.0.1] Bug fix for under* options.
%  \item[v1.0] Initial version.
% \end{description}
%
% \StopEventually{}
%
% \section{Implementation}
%
%\iffalse
%<*package>
%\fi
%
% \setlength{\parindent}{0em}
% \setlength{\parskip}{\smallskipamount}
%
% \subsection*{Management of options}
%
% \subsubsection*{Declaration of conditionals}
%
%    \begin{macrocode}
\newif\ifovar@option@oldarrows@
\newif\ifovar@option@esvect@ \ovar@option@esvect@true \PassOptionsToPackage{f}{esvect}
\newif\ifovar@option@tikz@
\newif\ifovar@option@pstricks@
\newif\ifovar@option@pstarrows@
\newif\ifovar@detectsubscripts@
\newif\ifovar@option@subother@
\newif\ifovar@option@subactive@
\newif\ifovar@option@debug@
%    \end{macrocode}
%
% Following conditionals are for predefined commands.
%
%    \begin{macrocode}
\newif\ifovar@option@overrightarrow@
\newif\ifovar@option@underrightarrow@
\newif\ifovar@option@overleftarrow@
\newif\ifovar@option@underleftarrow@
\newif\ifovar@option@overleftrightarrow@
\newif\ifovar@option@underleftrightarrow@
\newif\ifovar@option@overrightharpoonup@
\newif\ifovar@option@underrightharpoonup@
\newif\ifovar@option@overrightharpoondown@
\newif\ifovar@option@underrightharpoondown@
\newif\ifovar@option@overleftharpoonup@
\newif\ifovar@option@underleftharpoonup@
\newif\ifovar@option@overleftharpoondown@
\newif\ifovar@option@underleftharpoondown@
\newif\ifovar@option@overbar@
\newif\ifovar@option@underbar@
%    \end{macrocode}
%
% \subsubsection*{Declaration of options}
%
% \changes{v1.4}{2024/07/06}{Add option pstricks}
%
%    \begin{macrocode}
\DeclareOption{esvect}{\ovar@option@esvect@true}
\DeclareOption{noesvect}{\ovar@option@esvect@false}
\DeclareOption{esvecta}{\ovar@option@esvect@true\PassOptionsToPackage{a}{esvect}}
\DeclareOption{esvectb}{\ovar@option@esvect@true\PassOptionsToPackage{b}{esvect}}
\DeclareOption{esvectc}{\ovar@option@esvect@true\PassOptionsToPackage{c}{esvect}}
\DeclareOption{esvectd}{\ovar@option@esvect@true\PassOptionsToPackage{d}{esvect}}
\DeclareOption{esvecte}{\ovar@option@esvect@true\PassOptionsToPackage{e}{esvect}}
\DeclareOption{esvectf}{\ovar@option@esvect@true\PassOptionsToPackage{f}{esvect}}
\DeclareOption{esvectg}{\ovar@option@esvect@true\PassOptionsToPackage{g}{esvect}}
\DeclareOption{esvecth}{\ovar@option@esvect@true\PassOptionsToPackage{h}{esvect}}
\DeclareOption{old-arrows}{\ovar@option@oldarrows@true}
\DeclareOption{tikz}{\ovar@option@tikz@true}
\DeclareOption{pstricks}{\ovar@option@pstricks@true}
\DeclareOption{pstarrows}{\ovar@option@pstarrows@true}
\DeclareOption{subscripts}{\ovar@detectsubscripts@true}
\DeclareOption{subother}{\ovar@option@subother@true}
\DeclareOption{subactive}{\ovar@option@subactive@true}
\DeclareOption{debug}{\ovar@option@debug@true}
%    \end{macrocode}
%
% Following options are for predefined commands.
%
% \changes{v1.0.1}{2023/01/19}{Bug fix for under* options}
%
%    \begin{macrocode}
\DeclareOption{overrightarrow}{\ovar@option@overrightarrow@true}
\DeclareOption{underrightarrow}{\ovar@option@underrightarrow@true}
\DeclareOption{overleftarrow}{\ovar@option@overleftarrow@true}
\DeclareOption{underleftarrow}{\ovar@option@underleftarrow@true}
\DeclareOption{overleftrightarrow}{\ovar@option@overleftrightarrow@true}
\DeclareOption{underleftrightarrow}{\ovar@option@underleftrightarrow@true}
\DeclareOption{overrightharpoonup}{\ovar@option@overrightharpoonup@true}
\DeclareOption{underrightharpoonup}{\ovar@option@underrightharpoonup@true}
\DeclareOption{overrightharpoondown}{\ovar@option@overrightharpoondown@true}
\DeclareOption{underrightharpoondown}{\ovar@option@underrightharpoondown@true}
\DeclareOption{overleftharpoonup}{\ovar@option@overleftharpoonup@true}
\DeclareOption{underleftharpoonup}{\ovar@option@underleftharpoonup@true}
\DeclareOption{overleftharpoondown}{\ovar@option@overleftharpoondown@true}
\DeclareOption{underleftharpoondown}{\ovar@option@underleftharpoondown@true}
\DeclareOption{overbar}{\ovar@option@overbar@true}
\DeclareOption{underbar}{\ovar@option@underbar@true}
%    \end{macrocode}
%
% Following options are for sets of predefined commands.
%
%    \begin{macrocode}
\DeclareOption{overcommands}{%
  \ovar@option@overrightarrow@true
  \ovar@option@overleftarrow@true
  \ovar@option@overleftrightarrow@true
  \ovar@option@overrightharpoonup@true
  \ovar@option@overrightharpoondown@true
  \ovar@option@overleftharpoonup@true
  \ovar@option@overleftharpoondown@true
  \ovar@option@overbar@true
}
\DeclareOption{undercommands}{%
  \ovar@option@underrightarrow@true
  \ovar@option@underleftarrow@true
  \ovar@option@underleftrightarrow@true
  \ovar@option@underrightharpoonup@true
  \ovar@option@underrightharpoondown@true
  \ovar@option@underleftharpoonup@true
  \ovar@option@underleftharpoondown@true
  \ovar@option@underbar@true
}
\DeclareOption{allcommands}{%
  \ovar@option@overrightarrow@true
  \ovar@option@underrightarrow@true
  \ovar@option@overleftarrow@true
  \ovar@option@underleftarrow@true
  \ovar@option@overleftrightarrow@true
  \ovar@option@underleftrightarrow@true
  \ovar@option@overrightharpoonup@true
  \ovar@option@underrightharpoonup@true
  \ovar@option@overrightharpoondown@true
  \ovar@option@underrightharpoondown@true
  \ovar@option@overleftharpoonup@true
  \ovar@option@underleftharpoonup@true
  \ovar@option@overleftharpoondown@true
  \ovar@option@underleftharpoondown@true
  \ovar@option@overbar@true
  \ovar@option@underbar@true
}
%    \end{macrocode}
%
% \subsubsection*{Options processing}
%
%    \begin{macrocode}
\DeclareOption*{\PackageWarning{overarrows}{Unknown option: '\CurrentOption'}}
\ProcessOptions\relax
%    \end{macrocode}
%
% \subsection*{Package dependencies}
%
% \LaTeX{} distributions prior to 2020/10/01 must add the \pkg{xparse} package.
%
%    \begin{macrocode}
\RequirePackage{amsmath}
\RequirePackage{etoolbox}
%    \end{macrocode}
% Option \refOpt{old-arrows}. Configuration of arrows used for
% predefined commands.
% \changes{v1.2}{2024/07/06}{Use \cs{def} instead of \cs{let} for
% \cs{ovar@rightarrow} and \cs{ovar@leftarrow}}
%    \begin{macrocode}
\def\ovar@rightarrow{\rightarrow}
\def\ovar@leftarrow{\leftarrow}
\ifovar@option@oldarrows@
  \RequirePackage[old]{old-arrows}
  \def\ovar@rightarrow{\varrightarrow}
  \def\ovar@leftarrow{\varleftarrow}
\fi
%    \end{macrocode}
% Option \refOpt{esvect}.
%    \begin{macrocode}
\ifovar@option@esvect@
  \RequirePackage{esvect}
%    \end{macrocode}
% \changes{v1.2}{2024/07/06}{Fix esvect font sizes}
% Fix font description in |uesvect.fd| to allow any sizes (taken from Enrico
% Gregorio, \url{https://tex.stackexchange.com/a/689863/})
%    \begin{macrocode}
  \DeclareFontFamily{U}{esvect}{}
  \DeclareFontShape{U}{esvect}{m}{n}{
    <-5.5> vect5
    <5.5-6.5> vect6
    <6.5-7.5> vect7
    <7.5-8.5> vect8
    <8.5-9.5> vect9
    <9.5-> vect10
  }{}
\fi
%    \end{macrocode}
% Option \refOpt{tikz}.
%    \begin{macrocode}
\ifovar@option@tikz@
  \RequirePackage{tikz}
  \usetikzlibrary{arrows.meta}
\fi
%    \end{macrocode}
% Option \refOpt{pstricks}.
%    \begin{macrocode}
\ifovar@option@pstricks@
  \RequirePackage{pstricks-add}
\fi
%    \end{macrocode}
% Option \refOpt{pstarrows}.
%    \begin{macrocode}
\ifovar@option@pstarrows@
  \RequirePackage[pstarrows]{pict2e}
\fi
%    \end{macrocode}
%
% \subsection*{Configuration of subscripts detection}
%
% \changes{v1.1}{2023/02/15}{Support for non-standard subscripts}
%
% \begin{macro}{\SetOverArrowsSubscriptCommand}
% Sets the subscript command.
%    \begin{macrocode}
\newcommand{\SetOverArrowsSubscriptCommand}[1]{\global\let\ovar@subcmd=#1}
%    \end{macrocode}
% \end{macro}
%
% Initial configuration.
% \changes{v1.2}{2024/07/06}{Remove useless \cs{AtBeginDocument}}
%    \begin{macrocode}
\SetOverArrowsSubscriptCommand{_}
%    \end{macrocode}
% Option \refOpt{subother} for \emph{other} (catcode 12) subscript commands.
%    \begin{macrocode}
\ifovar@option@subother@
  \begingroup
    \catcode `_=12
    \SetOverArrowsSubscriptCommand{_}%
  \endgroup
\fi
%    \end{macrocode}
% Option \refOpt{subactive} for \emph{active} (catcode 13) subscript commands.
%    \begin{macrocode}
\ifovar@option@subactive@
  \begingroup
    \catcode `_=13
    \SetOverArrowsSubscriptCommand{_}%
  \endgroup
\fi
%    \end{macrocode}
% \subsection*{Management of keys}
%
% \subsubsection*{Family declaration and setters}
%
%    \begin{macrocode}
\RequirePackage{pgfkeys}
\pgfkeys{overarrows/.is family}
%    \end{macrocode}
%
% \begin{macro}{\ovar@set}
%    \begin{macrocode}
\newcommand{\ovar@set}[1]{\pgfqkeys{/overarrows}{#1}}
%    \end{macrocode}
% \end{macro}
%
% \begin{macro}{\SetOverArrowsMethod}
%    \begin{macrocode}
\NewDocumentCommand{\SetOverArrowsMethod}{ s O{fill} m O{} m }{%
  \IfBooleanTF{#1}{%
    \csgdef{ovar@set@#3}{#4\ovar@set{#5}}%
  }{%
    \csgdef{ovar@set@#3}{#4\ovar@set{%
        no stack macro hook/.code={%
          \ovar@set{stack macro/.expanded={%
              \expandafter\expandonce\csname ovar@stack@#2\endcsname%
              {\expandonce\ovar@length@min}%
              {\expandonce\ovar@before@arrow}{\expandonce\ovar@after@arrow}%
            }}%
        },#5}}%
  }%
}
%    \end{macrocode}
% \end{macro}
%
% \subsubsection*{Common keys}
%
%    \begin{macrocode}
\SetOverArrowsMethod*{common}[\undef{\ovar@macro@stack}\undef{\ovar@macro@arrow}]{%
%    \end{macrocode}
%
% \refKey{detect subscripts}.
%
%    \begin{macrocode}
  detect subscripts/.is if=ovar@detectsubscripts@,
%    \end{macrocode}
%
% \refKey{stack macro} and \refKey{arrow macro}.
%
%    \begin{macrocode}
  stack macro/.store in=\ovar@macro@stack,
  arrow macro/.store in=\ovar@macro@arrow,
  stack macro/.value required,
  arrow macro/.value required,
%    \end{macrocode}
%
% \refKey{no stack macro hook}, \refKey{no arrow macro hook}. These two keys
% must be redefined by the command \cs{ovar@set@}\meta{method}.
%
%    \begin{macrocode}
  no stack macro hook/.code={%
    \PackageError{overarrows}{Undefined stack macro}
    {The requested method is perhaps mispelled}
  },
  no arrow macro hook/.code={%
    \PackageError{overarrows}{Undefined arrow macro}
    {The requested method is perhaps mispelled}
  },
%    \end{macrocode}
%
% \refKey{min length}.
%
%    \begin{macrocode}
  min length/.store in=\ovar@length@min,
  min length/.value required,
  min length=0,
%    \end{macrocode}
%
% \refKey{before arrow}, \refKey{after arrow}, \refKey{space before arrow},
% \refKey{space after arrow}.
%
%    \begin{macrocode}
  before arrow/.store in=\ovar@before@arrow,
  after arrow/.store in=\ovar@after@arrow,
  before arrow/.value required,
  after arrow/.value required,
  before arrow=\empty,
  after arrow=\empty,
  space before arrow/.code=\pgfkeysalso{before arrow={\kern ##1}},
  space after arrow/.code=\pgfkeysalso{after arrow={\kern ##1}},
%    \end{macrocode}
%
% \refKey{shift left}, \refKey{shift right}, \refKey{shift leftright},
% \refKey{center arrow}, \refKey{left arrow}, \refKey{right arrow}.
%
%    \begin{macrocode}
  shift left/.store in=\ovar@shift@left,
  shift right/.store in=\ovar@shift@right,
  shift left/.value required,
  shift right/.value required,
  shift leftright/.code=\pgfkeysalso{%
    shift left=##1, shift right=##1,
  },
  center arrow/.code=\pgfkeysalso{shift leftright=0},
  shift leftright/.value required,
  center arrow/.value forbidden,
  left arrow/.code=\pgfkeysalso{%
    shift left=0, shift right=##1,
  },
  right arrow/.code=\pgfkeysalso{%
    shift left=##1, shift right=0,
  },
  left arrow/.default=2,
  right arrow/.default=2,
  right arrow,
%    \end{macrocode}
%
% \refKey{arrow under}.
%
%    \begin{macrocode}
  arrow under/.is choice,
  arrow under/noconfig/.code={
    \def\ovar@stack@fill{\ovar@stackunder@fill}
    \def\ovar@stack@lens{\ovar@stackunder@lens}
  },
  arrow under/autoconfig/.code={
    \pgfkeysalso{%
      arrow under=noconfig,
      detect subscripts=false,
      before arrow={\kern 1.3\ex@\relax},% like underarrow@ from amsmath
    }
  },
  arrow under/.default=autoconfig,
}
%    \end{macrocode}
%
% \subsubsection*{Keys for the \texttt{symb} method}
%
%    \begin{macrocode}
\SetOverArrowsMethod{symb}[\undef{\ovar@macro@arrowfill}]{%
%    \end{macrocode}
% Fill macro.
%    \begin{macrocode}
  fill macro/.store in=\ovar@macro@arrowfill,
  fill macro/.value required,
%    \end{macrocode}
% Arrow macro.
%    \begin{macrocode}
  no arrow macro hook/.code={%
    \ifdef{\ovar@macro@arrowfill}{}{%
      \ovar@set{%
        fill macro/.expanded={%
          \noexpand\ovar@arrow@fill%
          {\expandonce\ovar@shift@left}{\expandonce\ovar@shift@right}%
        }
      }
    }
    \ovar@set{%
      arrow macro/.expanded={%
        \expandonce{\ovar@macro@arrowfill}%
        {\expandonce{\ovar@arrow@start}\expandonce{\ovar@trim@start}}%
        {\expandonce{\ovar@trim@middle}\expandonce{\ovar@arrow@middle}%
          \expandonce{\ovar@trim@middle}}%
        {\expandonce{\ovar@trim@end}\expandonce{\ovar@arrow@end}}%
      }
    }
  },
%    \end{macrocode}
% \refKey{start}, \refKey{middle}, \refKey{end}.
%    \begin{macrocode}
  start/.store in=\ovar@arrow@start,
  middle/.store in=\ovar@arrow@middle,
  end/.store in=\ovar@arrow@end,
  start/.value required,
  middle/.value required,
  end/.value required,
%    \end{macrocode}
% \refKey{trim start}, \refKey{trim middle}, \refKey{trim end}, \refKey{trim},
% \refKey{no trimming}.
%    \begin{macrocode}
  trim start/.code={\def\ovar@trim@start{\xjoinrel[##1]}},
  trim middle/.code={\def\ovar@trim@middle{\xjoinrel[##1]}},
  trim end/.code={\def\ovar@trim@end{\xjoinrel[##1]}},
  trim start/.value required,
  trim middle/.value required,
  trim end/.value required,
  trim/.code={\pgfkeysalso{trim start={##1}, trim middle={##1}, trim end={##1}}},
  trim/.value required,
  no trimming/.code={%
    \let\ovar@trim@start\empty
    \let\ovar@trim@middle\empty
    \let\ovar@trim@end\empty
  },
  no trimming/.value forbidden,
%    \end{macrocode}
% \refKey{middle config}.
%    \begin{macrocode}
  middle config/.is choice,
  middle config/.value required,
  middle config/relbar/.code=\pgfkeysalso{%
    middle={\relbar},
    trim middle={2.5},
  },
  middle config/relbareda/.code={%
    \ifundef{\relbareda}{%
      \PackageWarning{overarrows}{Key 'middle config=relbareda' used,
        \MessageBreak%
        but \protect\relbareda\space is undefined; ignored.
        \MessageBreak%
        Load 'esvect' package, or use 'esvect' option \MessageBreak%
        to remove this warning}
    }{%
      \pgfkeysalso{%
        middle={\relbareda},
        trim middle={1},
      }
    }
  },
%    \end{macrocode}
% \changes{v1.2}{2024/07/06}{Add key \refKey{harrowextender}}
%    \begin{macrocode}
  middle config/harrowextender/.code={%
    \pgfkeysalso{%
      middle={\harrowextender},
      trim middle={0},
    }
  },
%    \end{macrocode}
% \changes{v1.2}{2024/07/06}{Use \cs{harrowextender}, if availlable.}
%    \begin{macrocode}
  middle config/auto/.code={%
    \ifundef{\harrowextender}{%
      \ifovar@option@esvect@
        \pgfkeysalso{middle config=relbareda}
      \else
        \pgfkeysalso{middle config=relbar}
      \fi
    }{%
      \pgfkeysalso{middle config=harrowextender}
    }
  },
%    \end{macrocode}
% \refKey{amsmath}.
%    \begin{macrocode}
  amsmath/.is choice,%
  amsmath/mimic/.code=\pgfkeysalso{%
    start={\relbar}, middle={\relbar}, end={\rightarrow},
    trim start=7,
    trim middle=2,
    trim end=7,
    shift leftright=0,
    after arrow={}, before arrow={},
  },
  amsmath/strict/.code=\pgfkeysalso{%
    amsmath=mimic,
    no trimming,
    fill macro={\arrowfill@}, stack macro={\overarrow@},
  },
  amsmath/.default=mimic,
%    \end{macrocode}
% \refKey{esvect}.
%    \begin{macrocode}
  esvect/.is choice,%
  esvect/mimic/.code=\pgfkeysalso{%
    start={\relbaredd}, middle={\relbareda}, end={\fldr},
    trim start=1.5,
    trim end=1.5,
    trim middle=0,
    right arrow=2,
    space before arrow=-.7pt,
    space after arrow=-.3pt,
  },
  esvect/strict/.code=\pgfkeysalso{%
    esvect=mimic,
    no trimming,
    fill macro={\traitfill@}, stack macro={\overvect@},
  },
  esvect/.default=mimic,
%    \end{macrocode}
% Initial configuration.
%    \begin{macrocode}
  amsmath, middle config=auto, end=\ovar@rightarrow, right arrow,
}
%    \end{macrocode}
%
% \subsubsection*{Keys for the \texttt{tikz} method}
%
%    \begin{macrocode}
\SetOverArrowsMethod[lens]{tikz}[\undef{\ovar@tikz@command}]{%
%    \end{macrocode}
% Arrow macro.
%    \begin{macrocode}
  no arrow macro hook/.code={%
    \ifdef{\ovar@tikz@command}{}{%
      \pgfkeysgetvalue{/overarrows/path options}{\ovar@tikz@pathoptions}
      \ovar@set{%
        tikz command/.expanded={%
          \noexpand\draw[\expandonce\ovar@tikz@pathoptions]\expandonce\ovar@tikz@path;
        }
      }
    }
    \pgfkeysgetvalue{/overarrows/tikz options}{\ovar@tikz@options}
    \ovar@set{%
      arrow macro/.expanded={%
        $\noexpand\mkern \expandonce{\ovar@shift@left} mu\noexpand\relax$%
        \noexpand\tikz[\expandonce{\ovar@tikz@options}]{\expandonce{\ovar@tikz@command}}%
        $\noexpand\mkern \expandonce{\ovar@shift@right} mu\noexpand\relax$%
      }
    }
  },
%    \end{macrocode}
% TikZ parts: \refKey{tikz command}, \refKey{tikz options}, \refKey{path
% options}, \refKey{path}.
%    \begin{macrocode}
  tikz command/.store in=\ovar@tikz@command,
  tikz options/.initial={x=\overarrowlength, line width=\overarrowthickness},
  path options/.initial={arrows={-Classical TikZ Rightarrow}, cap=round},
  path/.store in=\ovar@tikz@path,
  path={(0,0)--(1,0)},
  tikz command/.value required,
  tikz options/.value required,
  path options/.value required,
  path/.value required,
%    \end{macrocode}
% TikZ handy keys: \refKey{add path options}, \refKey{add tikz options},
% \refKey{arrows}, \refKey{line thickness}, \refKey{thinner}.
%    \begin{macrocode}
  add path options/.code=\pgfkeysalso{%
    path options/.append={, ##1}},%
  add tikz options/.code=\pgfkeysalso{%
    tikz options/.append={, ##1}},%
  arrows/.code=\pgfkeysalso{add path options={arrows={##1}}},%
  line thickness/.code=\pgfkeysalso{add path options={line width=##1}},%
  thinner/.code=\pgfkeysalso{line thickness={\overarrowsmallerthickness}},%
  add path options/.value required,%
  add tikz options/.value required,%
  arrows/.value required,%
  line thickness/.value required,%
  thinner/.value forbidden,%
%    \end{macrocode}
% Initial configuration.
%    \begin{macrocode}
  shift right=-2,
  min length=12,
}
%    \end{macrocode}
%
% \subsubsection*{Keys for the \texttt{pstricks} method}
%
% \changes{v1.4}{2024/07/06}{Add the method pstricks}
%
%    \begin{macrocode}
\SetOverArrowsMethod[lens]{pstricks}{%
%    \end{macrocode}
% Arrow macro.
%    \begin{macrocode}
  no arrow macro hook/.code={%
    \ovar@set{%
      arrow macro/.expanded={%
        $\noexpand\mkern \expandonce{\ovar@shift@left} mu\noexpand\relax$%
        \noexpand\begin{pspicture}\expandonce{\ovar@pstricks@geometry}%
          \noexpand\psset{linewidth=\expandonce{\ovar@pstricks@linethickness}}%
          \noexpand\psset{\expandonce{\ovar@pstricks@psset}}%
          \expandonce{\ovar@pstricks@command}%
        \noexpand\end{pspicture}%
        $\noexpand\mkern \expandonce{\ovar@shift@right} mu\noexpand\relax$%
      }
    }
  },
%    \end{macrocode}
% Pstricks parts: \refKey{pstricks command}, \refKey{psset},
% \refDoc{key:pstricks}{geometry}, \refDoc{key:pstricks}{line thickness}.
%    \begin{macrocode}
  pstricks command/.store in=\ovar@pstricks@command,
  psset/.store in=\ovar@pstricks@psset,
  geometry/.store in=\ovar@pstricks@geometry,
  line thickness/.store in=\ovar@pstricks@linethickness,
  pstricks command/.value required,
  psset/.value required,
  geometry/.value required,
  line thickness/.value required,
%    \end{macrocode}
% Pstricks handy key: \refDoc{key:pstricks}{arrow}, \refDoc{key:pstricks}{thinner}.
%    \begin{macrocode}
  arrow/.style={pstricks command={\psline{##1}(0,0)(\overarrowlength,0)}},%
  arrow/.value required,%
  thinner/.style={line thickness={\overarrowsmallerthickness}},%
  thinner/.value forbidden,%
%    \end{macrocode}
% Initial configuration.
%    \begin{macrocode}
  shift right=-2,
  min length=12,
  geometry={(0,-0.5ex)(\overarrowlength,0.5ex)},%
  line thickness={\overarrowthickness},%
  arrow={->},%
  psset={},%
}
%    \end{macrocode}
%
% \subsubsection*{Keys for the \texttt{picture} method}
%
%    \begin{macrocode}
\SetOverArrowsMethod[lens]{picture}{%
%    \end{macrocode}
% Arrow macro.
%    \begin{macrocode}
  no arrow macro hook/.code={%
    \ovar@set{%
      arrow macro/.expanded={%
        $\noexpand\mkern \expandonce{\ovar@shift@left} mu\noexpand\relax$%
        \noexpand\begin{picture}\expandonce{\ovar@picture@geometry}%
          \noexpand\linethickness{\expandonce{\ovar@picture@linethickness}}%
          \expandonce{\ovar@picture@command}%
          \noexpand\end{picture}%
        $\noexpand\mkern \expandonce{\ovar@shift@right} mu\noexpand\relax$%
      }
    }
  },
%    \end{macrocode}
% Picture parts: \refKey{picture command}, \refDoc{key:picture}{geometry},
% \refDoc{key:picture}{line thickness}.
%    \begin{macrocode}
  picture command/.store in=\ovar@picture@command,
  geometry/.store in=\ovar@picture@geometry,
  line thickness/.store in=\ovar@picture@linethickness,
  picture command/.value required,
  geometry/.value required,
  line thickness/.value required,
%    \end{macrocode}
% Picture handy key: \refDoc{key:picture}{thinner}.
%    \begin{macrocode}
  thinner/.code=\pgfkeysalso{line thickness={\overarrowsmallerthickness}},
%    \end{macrocode}
% Initial configuration.
%    \begin{macrocode}
  shift right=-2,
  min length=18,
  geometry={(\overarrowlength,1ex)(0,-0.5ex)},%
  line thickness={\overarrowthickness},%
  picture command={\put(0,0){\vector(1,0){\overarrowlength}}},%
}
%    \end{macrocode}
%
% \subsection*{Commands}
%
% \subsubsection*{Macros for symbols assemblage}
%
% \begin{macro}{\xjoinrel}
%    \begin{macrocode}
\ifdef{\xjoinrel}{%
  \PackageWarning{overarrows}{Command \protect\xjoinrel\space already defined.
    \MessageBreak%
    Previous definition will be overridden}
}{}
%    \end{macrocode}
% Use a default value of |3.5 mu|, as recommended by Enrico Gregorio (see
%  \url{https://tex.stackexchange.com/a/471736}). \cs{joinrel}
% uses a value of 3 mu.
%    \begin{macrocode}
\DeclareRobustCommand{\xjoinrel}[1][3.5]{\mathrel{\mkern-#1mu}}
%    \end{macrocode}
% \end{macro}
%
% \begin{macro}{\smallermathstyle}
%    \begin{macrocode}
\newcommand*{\smallermathstyle}{%
  \mathchoice{\scriptstyle}{\scriptstyle}{\scriptscriptstyle}{}
}
%    \end{macrocode}
% \end{macro}
%
% \begin{macro}{\ovar@arrow@fill}
% Macro used for default \refKey{fill macro}.\\
% |#1|: left shift\\
% |#2|: right shift\\
% |#3|: arrow start\\
% |#4|: arrow middle\\
% |#5|: arrow end\\
% |#6|: math style
%    \begin{macrocode}
\def\ovar@arrow@fill#1#2#3#4#5#6{%
  $\m@th\thickmuskip0mu\medmuskip\thickmuskip\thinmuskip\thickmuskip\relax%
  \mkern #1 mu\relax#6#3%
  \cleaders\hbox{$#6#4$}\hfill%
  #5\mkern #2 mu\relax$%
}
%    \end{macrocode}
% \end{macro}
%
% \subsubsection*{Macros for fixed length arrows}
%
% Lengths declaration.
%    \begin{macrocode}
\newlength{\overarrowlength}
\newlength{\overarrowthickness}
\newlength{\overarrowsmallerthickness}
\newlength{\ovar@tempdim}
%    \end{macrocode}
%
% \begin{macro}{\ovar@set@arrowlength}
% Sets \refLen{overarrowlength}.\\
% |#1|: min length, in math units\\
% |#2|: math style\\
% |#3|: content
%    \begin{macrocode}
\def\ovar@set@arrowlength#1#2#3{%
  \settowidth{\ovar@tempdim}{$\m@th#2\mskip #1 mu\relax$}%
  \settowidth{\overarrowlength}{$\m@th#2#3$}%
  \ifdim \overarrowlength < \ovar@tempdim \overarrowlength=\ovar@tempdim\fi%
}
%    \end{macrocode}
% \end{macro}
%
% \begin{macro}{\ovar@set@arrowthickness}
% \begin{macro}{\ovar@set@arrowthickness@UM@lua}
% Sets \refLen{overarrowthickness} and \refLen{overarrowsmallerthickness}.\\
% |#1|: math style
%
% Set to the default rule thickness of the current math style, normaly given by
% \cs{fontdimen} 8 family 3. With \pkg{unicode-math}, use instead:
% \begin{itemize}
%  \item \cs{fontdimen} 54 family 2 with XeTeX,
%  \item \cs{Umathoverbarrule} with LuaTex.
% \end{itemize}
% \changes{v1.2}{2024/07/06}{Get the correct rule thickness with unicode-math}
%    \begin{macrocode}
\def\ovar@rulethickness@fontdimen{8}
\def\ovar@rulethickness@family{3}
\def\ovar@set@arrowthickness#1{%
  \ifx#1\displaystyle%
    \overarrowthickness =
      \fontdimen \ovar@rulethickness@fontdimen  \textfont \ovar@rulethickness@family%
    \overarrowsmallerthickness =
      \fontdimen \ovar@rulethickness@fontdimen \scriptfont \ovar@rulethickness@family%
  \else\ifx#1\textstyle%
    \overarrowthickness =
      \fontdimen \ovar@rulethickness@fontdimen \textfont \ovar@rulethickness@family%
    \overarrowsmallerthickness =
      \fontdimen \ovar@rulethickness@fontdimen \scriptfont \ovar@rulethickness@family%
  \else\ifx#1\scriptstyle%
    \overarrowthickness =
      \fontdimen \ovar@rulethickness@fontdimen \scriptfont \ovar@rulethickness@family%
    \overarrowsmallerthickness =
      \fontdimen \ovar@rulethickness@fontdimen \scriptscriptfont \ovar@rulethickness@family%
  \else%
    \overarrowthickness =
      \fontdimen \ovar@rulethickness@fontdimen \scriptscriptfont \ovar@rulethickness@family%
    \overarrowsmallerthickness = \overarrowthickness%
  \fi\fi\fi%
}
%    \end{macrocode}
%  \pkg{unicode-math} with LuaTeX version.
%    \begin{macrocode}
\def\ovar@set@arrowthickness@UM@lua#1{%
  \overarrowthickness = \Umathoverbarrule #1
  \ifx#1\displaystyle%
    \overarrowsmallerthickness = \Umathoverbarrule \textstyle%
  \else\ifx#1\textstyle%
    \overarrowsmallerthickness = \Umathoverbarrule \scriptstyle%
  \else%
    \overarrowsmallerthickness = \Umathoverbarrule \scriptscriptstyle%
  \fi\fi%
}
%    \end{macrocode}
% Test which version to use.
%    \begin{macrocode}
\AtBeginDocument{%
  \@ifpackageloaded{unicode-math-luatex}
    {%
      \global\let\ovar@set@arrowthickness\ovar@set@arrowthickness@UM@lua
    }
    {%
      \@ifpackageloaded{unicode-math-xetex}
        {%
          \gdef\ovar@rulethickness@fontdimen{54}
          \gdef\ovar@rulethickness@family{2}
        }
        {}
    }
}
%    \end{macrocode}
% \end{macro}
% \end{macro}
%
% \subsubsection*{Stack macros}
%
% \begin{macro}{\ovar@stackover@@}
% \begin{macro}{\ovar@stackunder@@}
% Bases of all stack macros.\\
% |#1|: min length, in math units\\
% |#2|: vertical mode material before arrow\\
% |#3|: vertical mode material after arrow\\
% |#4|: arrow\\
% |#5|: math style\\
% |#6|: content
%    \begin{macrocode}
\def\ovar@stackover@@#1#2#3#4#5#6{\vbox{\ialign{##\crcr%
      $#5\mskip #1 mu\relax$\crcr%
      \noalign{#2\nointerlineskip}#4\crcr%
      \noalign{#3\nointerlineskip}%
      $\m@th\hfil#5#6\hfil$\crcr%
    }%
  }%
}
\def\ovar@stackunder@@#1#2#3#4#5#6{\vtop{\ialign{##\crcr%
      $\m@th\hfil#5#6\hfil$\crcr%
      \noalign{#2\nointerlineskip}#4\crcr%
      \noalign{#3\nointerlineskip}%
      $#5\mskip #1 mu\relax$\crcr%
    }%
  }%
}
%    \end{macrocode}
% \end{macro}
% \end{macro}
%
% \begin{macro}{\ovar@stackover@}
% \begin{macro}{\ovar@stackunder@}
% Stack macros without min arrow length.\\
% |#1|: vertical mode material before arrow\\
% |#2|: vertical mode material after arrow\\
% |#3|: arrow macro\\
% |#4|: math style\\
% |#5|: content
%    \begin{macrocode}
\def\ovar@stackover@#1#2#3#4#5{\ovar@stackover@@{0}{#1}{#2}{#3}{#4}{#5}}
\def\ovar@stackunder@#1#2#3#4#5{\ovar@stackunder@@{0}{#1}{#2}{#3}{#4}{#5}}
%    \end{macrocode}
% \end{macro}
% \end{macro}
%
% \begin{macro}{\ovar@stackover@fill}
% \begin{macro}{\ovar@stackunder@fill}
% \begin{macro}{\ovar@stack@fill}
% Stack macros for extensible arrows.\\
% |#1|: min length, in math units\\
% |#2|: vertical mode material before arrow\\
% |#3|: vertical mode material after arrow\\
% |#4|: arrow filler macro\\
% |#5|: math style\\
% |#6|: content
%    \begin{macrocode}
\def\ovar@stackover@fill#1#2#3#4#5#6{\ovar@stackover@@{#1}{#2}{#3}{#4#5}{#5}{#6}}
\def\ovar@stackunder@fill#1#2#3#4#5#6{\ovar@stackunder@@{#1}{#2}{#3}{#4#5}{#5}{#6}}
%    \end{macrocode}
% \cs{ovar@stack@fill} matches the macro \cs{ovar@stackover@fill} by default, or
% \cs{ovar@stackunder@fill} with \refKey{arrow under}.
%    \begin{macrocode}
\def\ovar@stack@fill{\ovar@stackover@fill}
%    \end{macrocode}
% \end{macro}
% \end{macro}
% \end{macro}
%
% \begin{macro}{\ovar@stackover@lens}
% \begin{macro}{\ovar@stackunder@lens}
% \begin{macro}{\ovar@stack@lens}
% Stack macros for fixed-length arrows (these call \cs{ovar@set@arrowlength} and
% \cs{ovar@set@arrowthickness}).\\
% |#1|: min length, in math units\\
% |#2|: vertical mode material before arrow\\
% |#3|: vertical mode material after arrow\\
% |#4|: arrow content macro\\
% |#5|: math style\\
% |#6|: content
%    \begin{macrocode}
\def\ovar@stackover@lens#1#2#3#4#5#6{%
  \ovar@set@arrowlength{#1}{#5}{#6}%
  \ovar@set@arrowthickness{#5}%
  \ovar@stackover@{#2}{#3}{#4}{#5}{#6}%
}
\def\ovar@stackunder@lens#1#2#3#4#5#6{%
  \ovar@set@arrowlength{#1}{#5}{#6}%
  \ovar@set@arrowthickness{#5}%
  \ovar@stackunder@{#2}{#3}{#4}{#5}{#6}%
}
%    \end{macrocode}
% \cs{ovar@stack@lens} matches the macro \cs{ovar@stackover@lens} by default, or
% \cs{ovar@stackunder@lens} with \refKey{arrow under}.
%    \begin{macrocode}
\def\ovar@stack@lens{\ovar@stackover@lens}
%    \end{macrocode}
% \end{macro}
% \end{macro}
% \end{macro}
%
% \subsubsection*{Macro for commands creation}
%
% \changes{v1.4}{2024/07/06}{Allow backslash in command name}
% In the initial version, the commands names must be given as csname (without
% backslash). To harmonize the syntax with standard \cs{NewDocumentCommand},
% define an argument processor so that both
% \cs{NewOverArrowCommand}|\{\myarrow\}| and
% \cs{NewOverArrowCommand}|\{myarrow\}| are accepted.
%
%    \begin{macrocode}
\ExplSyntaxOn
\cs_new_protected:Npn \__overarrows_processor_strip_escape_char:n #1
  {
     \regex_match:nnTF { ^\cC. } { #1 }
     { \tl_set:Nx \ProcessedArgument { \cs_to_str:N #1 } }
     { \tl_set:Nx \ProcessedArgument { #1 } }
  }
\cs_new_eq:NN \ovar@cmdname@processor \__overarrows_processor_strip_escape_char:n
\ExplSyntaxOff
%    \end{macrocode}
%
% \begin{macro}{\DeclareOverArrowCommand}
%    \begin{macrocode}
\NewDocumentCommand{\DeclareOverArrowCommand}{
 O{symb} >{\ovar@cmdname@processor} m m
}{%
  \begingroup
  \ovar@set@common
  \ifcsdef{ovar@set@#1}{%
    \csuse{ovar@set@#1}
  }{%
    \PackageError{overarrows}{Unknown method #1}
    {Try with 'symb', 'tikz' or 'picture'}
  }
  \ovar@set{#3 }
  \ifdef{\ovar@macro@arrow}{}{%
    \ovar@set{no arrow macro hook}
  }
  \ifdef{\ovar@macro@stack}{}{%
    \ovar@set{no stack macro hook}
  }
  \csxdef{ovar@#2@normal}{%
    \noexpand\mathpalette{%
      \expandonce{\ovar@macro@stack}{\expandonce{\ovar@macro@arrow}}%
    }
  }
  \csxdef{ovar@#2@starred}{%
    \noexpand\mathpalette{%
      \noexpand\ovar@starversion{%
        \expandonce{\ovar@macro@stack}{\expandonce{\ovar@macro@arrow}}%
      }
    }
  }
  \ifovar@detectsubscripts@%
  \csgdef{ovar@#2@auto}##1{%
%    \end{macrocode}
% \vspace{-2\smallskipamount}%^^A remove space between code blocks
% \changes{v1.1}{2023/02/15}{Support for non-standard subscripts}
%    \begin{macrocode}
    \@ifnextchar \ovar@subcmd {%
      \csuse{ovar@#2@starred}{##1}%
    }{%
      \csuse{ovar@#2@normal}{##1}%
    }%
  }
  \csgdef{#2}{%
    \@ifstar{\csuse{ovar@#2@starred}}{\csuse{ovar@#2@auto}}%
  }
  \else
  \csgdef{#2}{%
    \@ifstar{\csuse{ovar@#2@starred}}{\csuse{ovar@#2@normal}}%
  }
  \fi
  \ifovar@option@debug@
  \PackageInfo{overarrows}{%
    Meaning of \protect\ovar@#2@normal\MessageBreak
    used for \@backslashchar#2:\MessageBreak%
      \expandafter\meaning\csname ovar@#2@normal\endcsname}
  \fi
  \endgroup
}
%    \end{macrocode}
% \end{macro}
%
% \begin{macro}{\ProvideOverArrowCommand}
%    \begin{macrocode}
\NewDocumentCommand{\ProvideOverArrowCommand}{
 O{symb} >{\ovar@cmdname@processor} m m
}{%
  \ifcsdef{#2}{}{
    \DeclareOverArrowCommand[#1]{#2}{#3}
  }
}
%    \end{macrocode}
% \end{macro}
%
% \begin{macro}{\NewOverArrowCommand}
%    \begin{macrocode}
\NewDocumentCommand{\NewOverArrowCommand}{
 O{symb} >{\ovar@cmdname@processor} m m
}{%
  \ifcsdef{#2}{%
    \PackageError{overarrows}{Command \csname #2\endcsname already defined}%
    {You have used \protect\NewOverArrowCommand\space with a command that
      already has a definition. \MessageBreak%
      Choose another name, or use instead \protect\DeclareOverArrowCommand.}
  }{%
    \DeclareOverArrowCommand[#1]{#2}{#3}
  }
}
%    \end{macrocode}
% \end{macro}
%
% \begin{macro}{\RenewOverArrowCommand}
%    \begin{macrocode}
\NewDocumentCommand{\RenewOverArrowCommand}{
 O{symb} >{\ovar@cmdname@processor} m m
}{%
  \ifcsundef{#2}{%
    \PackageError{overarrows}{Command  \csname #2\endcsname undefined}%
    {You have used \protect\RenewOverArrowCommand\space with a command that was
      never defined. \MessageBreak%
      Check the requested name, or use instead \protect\NewOverArrowCommand.}
  }{%
    \DeclareOverArrowCommand[#1]{#2}{#3}
  }
}
%    \end{macrocode}
% \end{macro}
%
% \subsubsection*{Starred variant}
%
% \changes{v1.4}{2024/07/06}{Use boxes in starred variant for better performances}
% \begin{macro}{\ovar@starversion}
% |#1|: definition (stack macro + arrow macro)\\
% |#2|: math style\\
% |#3|: content
%    \begin{macrocode}
\newsavebox\ovar@tempbox
\def\ovar@starversion#1#2#3{%
  \sbox{\ovar@tempbox}{$\m@th #1#2{#3}$}%
  \usebox{\ovar@tempbox}%
%    \end{macrocode}
%   Remove the extra space added by the arrow.
%    \begin{macrocode}
  \settowidth{\ovar@tempdim}{$\m@th #2{#3}$}%
  \kern\dimeval{0.5\ovar@tempdim - 0.5\wd\ovar@tempbox}%
}
%    \end{macrocode}
% \end{macro}
%
% \subsubsection*{\cs{vv} vector command}
%
% \begin{macro}{\vv}
% \begin{macro}{\esvectvv}
% Backup and redefinition of \pkg{esvect} \refCom{vv} vector command.
%    \begin{macrocode}
\ifovar@option@esvect@
  \let\esvectvv\vv
  \undef\vv
%    \end{macrocode}
% \changes{v1.2}{2024/07/06}{Use \refKey{middle config/relbareda} instead of
% \refKey{middle config/auto}}
%    \begin{macrocode}
  \NewOverArrowCommand{\vv}{esvect, middle config = relbareda}
\fi
%    \end{macrocode}
% \end{macro}
% \end{macro}
%
% \subsubsection*{Predefined commands}
%
% \changes{v1.2}{2024/07/06}{Declare Declare predefined commands after \pkg{unicode-math} settings.}
% Declare predefined commands after \pkg{unicode-math} settings.
%
%    \begin{macrocode}
\DeclareHookRule{begindocument}{overarrows}{after}{unicode-math-luatex}
\DeclareHookRule{begindocument}{overarrows}{after}{unicode-math-xetex}
\AddToHook{begindocument}[overarrows]
  {
%    \end{macrocode}
%
% \begin{macro}{\overrightarrow}
%    \begin{macrocode}
    \ifovar@option@overrightarrow@
      \DeclareOverArrowCommand{\overrightarrow}{%
        amsmath, middle config=relbar,
        end=\ovar@rightarrow,
        right arrow,
      }
    \fi
%    \end{macrocode}
% \end{macro}
%
% \begin{macro}{\underrightarrow}
%    \begin{macrocode}
    \ifovar@option@underrightarrow@
      \DeclareOverArrowCommand{\underrightarrow}{%
        amsmath, middle config=relbar,
        end=\ovar@rightarrow,
        right arrow,
        arrow under,
      }
    \fi
%    \end{macrocode}
% \end{macro}
%
% \begin{macro}{\overleftarrow}
%    \begin{macrocode}
    \ifovar@option@overleftarrow@
      \DeclareOverArrowCommand{\overleftarrow}{%
        amsmath, middle config=relbar,
        start=\ovar@leftarrow,
        end=\relbar,
        left arrow,
      }
    \fi
%    \end{macrocode}
% \end{macro}
%
% \begin{macro}{\underleftarrow}
%    \begin{macrocode}
    \ifovar@option@underleftarrow@
      \DeclareOverArrowCommand{\underleftarrow}{%
        amsmath, middle config=relbar,
        start=\ovar@leftarrow,
        end=\relbar,
        left arrow,
        arrow under,
      }
    \fi
%    \end{macrocode}
% \end{macro}
%
% \begin{macro}{\overleftrightarrow}
%    \begin{macrocode}
    \ifovar@option@overleftrightarrow@
      \DeclareOverArrowCommand{\overleftrightarrow}{%
        amsmath, middle config=relbar,
        start=\ovar@leftarrow,
        end=\ovar@rightarrow,
        center arrow,
      }
    \fi
%    \end{macrocode}
% \end{macro}
%
% \begin{macro}{\underleftrightarrow}
%    \begin{macrocode}
    \ifovar@option@underleftrightarrow@
      \DeclareOverArrowCommand{\underleftrightarrow}{%
        amsmath, middle config=relbar,
        start=\ovar@leftarrow,
        end=\ovar@rightarrow,
        center arrow,
        arrow under,
      }
    \fi
%    \end{macrocode}
% \end{macro}
%
% \begin{macro}{\overrightharpoonup}
%    \begin{macrocode}
    \ifovar@option@overrightharpoonup@
      \DeclareOverArrowCommand{\overrightharpoonup}{%
        amsmath, middle config=relbar,
        end=\rightharpoonup,
        right arrow,
      }
    \fi
%    \end{macrocode}
% \end{macro}
%
% \begin{macro}{\underrightharpoonup}
%    \begin{macrocode}
    \ifovar@option@underrightharpoonup@
      \DeclareOverArrowCommand{\underrightharpoonup}{%
        amsmath, middle config=relbar,
        end=\rightharpoonup,
        right arrow,
        arrow under,
      }
    \fi
%    \end{macrocode}
% \end{macro}
%
% \begin{macro}{\overrightharpoondown}
%    \begin{macrocode}
    \ifovar@option@overrightharpoondown@
      \DeclareOverArrowCommand{\overrightharpoondown}{%
        amsmath, middle config=relbar,
        end=\rightharpoondown,
        right arrow,
      }
    \fi
%    \end{macrocode}
% \end{macro}
%
% \begin{macro}{\underrightharpoondown}
%    \begin{macrocode}
    \ifovar@option@underrightharpoondown@
      \DeclareOverArrowCommand{\underrightharpoondown}{%
        amsmath, middle config=relbar,
        end=\rightharpoondown,
        right arrow,
        arrow under,
      }
    \fi
%    \end{macrocode}
% \end{macro}
%
% \begin{macro}{\overleftharpoonup}
%    \begin{macrocode}
    \ifovar@option@overleftharpoonup@
      \DeclareOverArrowCommand{\overleftharpoonup}{%
        amsmath, middle config=relbar,
        start=\leftharpoonup,
        end=\relbar,
        left arrow,
      }
    \fi
%    \end{macrocode}
% \end{macro}
%
% \begin{macro}{\underleftharpoonup}
%    \begin{macrocode}
    \ifovar@option@underleftharpoonup@
      \DeclareOverArrowCommand{\underleftharpoonup}{%
        amsmath, middle config=relbar,
        start=\leftharpoonup,
        end=\relbar,
        left arrow,
        arrow under,
      }
    \fi
%    \end{macrocode}
% \end{macro}
%
% \begin{macro}{\overleftharpoondown}
%    \begin{macrocode}
    \ifovar@option@overleftharpoondown@
      \DeclareOverArrowCommand{\overleftharpoondown}{%
        amsmath, middle config=relbar,
        start=\leftharpoondown,
        end=\relbar,
        left arrow,
      }
    \fi
%    \end{macrocode}
% \end{macro}
%
% \begin{macro}{\underleftharpoondown}
%    \begin{macrocode}
    \ifovar@option@underleftharpoondown@
      \DeclareOverArrowCommand{\underleftharpoondown}{%
        amsmath, middle config=relbar,
        start=\leftharpoondown,
        end=\relbar,
        left arrow,
        arrow under,
      }
    \fi
%    \end{macrocode}
% \end{macro}
%
% \begin{macro}{\overbar}
%    \begin{macrocode}
    \ifovar@option@overbar@
      \DeclareOverArrowCommand{\overbar}{%
        amsmath, middle config=relbar,
        start={\std@minus}, end={\std@minus},% \relbar is defined with \mathsm@sh
        shift leftright=0,
        space after arrow=-0.3ex,
      }
    \fi
%    \end{macrocode}
% \end{macro}
%
% With \pkg{unicode-math}, add |\vphantom{+}| to get the correct position.
% \begin{macro}{\underbar}
%    \begin{macrocode}
    \ifovar@option@underbar@
      \DeclareOverArrowCommand{\underbar}{%
        amsmath, middle config=relbar,
        start={\vphantom{+}\std@minus}, end={\std@minus},% \relbar is defined with \mathsm@sh
        shift leftright=0,
        arrow under,
        space before arrow=-0.3ex,
      }
    \fi
%    \end{macrocode}
% \end{macro}
%
% End of |\AddToHook{begindocument}| hook.
%    \begin{macrocode}
  }
%    \end{macrocode}
%
%
% \subsubsection*{Test macros}
%
% \begin{macro}{\ovar@testmathstyles}
% Tabular containing the output of a command for the four math styles and
% different patterns.
%    \begin{macrocode}
\newcommand{\ovar@testmathstyles}[2][]{
  \begingroup
  \newcommand*{\ovar@row@teststyle}[1]{%
    $\displaystyle ##1$
    & $\textstyle ##1$
    & $\scriptstyle ##1$
    & $\scriptscriptstyle ##1$
    \\
  }
  \renewcommand*{\arraystretch}{1.5}
  \begin{tabular*}{0.95\linewidth}{@{\extracolsep{\fill}} cccc}
    \hline
    \footnotesize\texttt{\texttt{\textbackslash displaystyle}}
    & \footnotesize\texttt{\texttt{\textbackslash textstyle}}
    & \footnotesize\texttt{\texttt{\textbackslash scriptstyle}}
    & \footnotesize\texttt{\texttt{\textbackslash scriptscriptstyle}}
    \\
    \hline
    \ovar@row@teststyle{\csuse{#2}{v}}
    \ovar@row@teststyle{\csuse{#2}{AB}}
    \ovar@row@teststyle{\csuse{#2}{\mathrm{grad}}}
    \ovar@row@teststyle{\csuse{#2}{my~long~vector}}
    \IfValueT{#1}{\ovar@row@teststyle{\csuse{#2}{#1}}}
    \hline
  \end{tabular*}
  \endgroup
}
%    \end{macrocode}
% \end{macro}
%
% \changes{v1.1}{2023/02/15}{Support for non-standard subscripts}
%
% \begin{macro}{\ovar@testkerning}
%    \begin{macrocode}
\begingroup
\ifovar@option@subother@  \catcode `_=12 \fi
\ifovar@option@subactive@ \catcode `_=13 \fi
\gdef\ovar@testkerning#1{%
  \begin{displaymath}
    #1{t}_{#1{u}_{#1{v}}}
    \qquad
    #1{\imath}_0
    \qquad
    #1{v}
    = #1{v}_x + #1{v}_y + #1{v}_z
    = v_x #1{\imath} + v_y #1{\jmath} + v_z #1{k}
  \end{displaymath}
}
\endgroup
%    \end{macrocode}
% \end{macro}
%
% \begin{macro}{\TestOverArrow}
%    \begin{macrocode}
\NewDocumentCommand{\TestOverArrow}{
  s o >{\ovar@cmdname@processor} m
}{%
  \ifcsdef{#3}{}{%
    \PackageWarning{overarrows}{Unknown name '#3' passed to
      \protect\TestOverArrow}
  }
  \IfBooleanTF{#1}{%
    \noindent\framebox{%
      \begin{minipage}{0.95\linewidth}
        \centering
        \noindent\textbf{\large%
          Test of \texttt{\textbackslash#3} and \texttt{\textbackslash#3*} macros}
        \bigskip\par
        \textbf{\texttt{\textbackslash#3} for different math styles}
        \smallskip\par
        \ovar@testmathstyles[#2]{#3}%
        \bigskip\par
        \textbf{\texttt{\textbackslash#3} kerning}
        \ovar@testkerning{\csuse{#3}}
        \textbf{\texttt{\textbackslash#3*} kerning}
        \ovar@testkerning{\csuse{#3}*}
      \end{minipage}%
    }\bigskip\par
  }{%
    \ovar@testmathstyles[#2]{#3}%
  }
}
%    \end{macrocode}
% \end{macro}
%
%\iffalse
%</package> ^^A End of package implementation.
%\fi
%
% \iffalse % ^^A The following is not for doc
%
%<*docstyle>
%
% \section*{overarrows-doc package}
%
% The following define some styles used for documentation and should not be
% included in the documentation itself.
%
% \subsection*{Packages}
%
%    \begin{macrocode}
\RequirePackage{xcolor}
\PassOptionsToPackage{hyperref}{xcolor}
\RequirePackage[many]{tcolorbox}
\tcbuselibrary{listings}
\tcbuselibrary{documentation}
\hypersetup{hidelinks}
%    \end{macrocode}
%
% \subsection*{Colors}
%
%    \begin{macrocode}
\definecolor{ovar_lavender}{rgb}{0.92,0.92,1}
\definecolor{ovar_darkblue}{rgb}{0.1,0.2,0.5}
\definecolor{ovar_darkgreen}{rgb}{0,0.39,0}
\definecolor{ovar_beige}{rgb}{.96,.96,.86}
\colorlet{ovar_commands}{ovar_darkblue}
\colorlet{ovar_keys}{ovar_darkgreen}
\colorlet{ovar_lengths}{violet}
\colorlet{ovar_options}{Definition} % from tcolorbox documentation
%    \end{macrocode}
%
% \subsection*{listings styles}
%
%    \begin{macrocode}
\lstdefinestyle{lstovardoc}{%
  language = [LaTeX]TeX,
  columns         = flexible,
  keepspaces      = true,
  keywordstyle    = {\bfseries\color{ovar_darkblue}},
  texcsstyle      = *{\bfseries\color{ovar_darkblue}},
  commentstyle    = {\color{gray}},
  identifierstyle = {\color{ovar_darkgreen}},
  morekeywords    = {},% keywordsprefix needs a morekeywords before
  keywordsprefix  = {\\},
%    \end{macrocode}
% For literate, the extra braces around the replacement text avoid the error: "Argument of |lst@FillFixed@| has an extra \}."
%    \begin{macrocode}
  literate        = *{\$}{{\textcolor{red}{\$}}}{1}
                    {\&}{{\textcolor{red}{\&}}}{1}
                    {\}}{{\textcolor{darkgray}{\}}}}{1}
                    {\{}{{\textcolor{darkgray}{\{}}}{1}
                    {\\\\}{{\textcolor{red}{\textbackslash\textbackslash}}}{2}
%    \begin{macrocode}
% One \% and four spaces to remove extra lines added by \cs{macrocode}
%    \end{macrocode}
                    {\%\ \ \ \ }{}0,
  basicstyle      = \ttfamily\footnotesize,
  frame           = none,
  framesep        = 0pt,
  aboveskip       = 0pt,
  belowskip       = 0pt,
}
\lstdefinestyle{lstovarcode}{%
  style           = lstovardoc,
  backgroundcolor = \color{ovar_beige},
  breakindent     = 1cm,
  numbers         = left,
  numberstyle     = \scriptsize\ttfamily\color{gray},
  aboveskip       = \smallskipamount,
  belowskip       = \smallskipamount,
}
%    \end{macrocode}
%
% \subsection*{tcolorbox}
%
%    \begin{macrocode}
\tcbset{% copied and adapted from tcolorbox.doc.s_main.sty
  documentation listing style=lstovardoc,%
  doc keypath=overarrows,
  size=small,%
  % verbatim ignore percent,
  %% styles
  docexample/.style={%
    bicolor jigsaw,
    before skip balanced=\medskipamount,
    after skip balanced=\medskipamount,
    fonttitle=\bfseries,
    fontlower=\normalfont,
    halign lower=center,
    colframe=ovar_darkblue,
    colback=ovar_lavender,
    colbacklower=white,
    drop fuzzy shadow,
  },
  color key=ovar_keys,
  color command=ovar_commands,
  color length=ovar_lengths,
  doc head key={fontlower=\footnotesize, collower=darkgray},
  before doc body={\parskip=\smallskipamount},
}
%    \end{macrocode}
%
% \subsection*{Index}
%
% To deal with internal macros, use '=' as index actual, in place of '@'.
%    \begin{macrocode}
\begin{filecontents*}{\jobname.mst}
actual '='
\end{filecontents*}
\tcbset{%
  index actual={=},
  index colorize=true,
  index gather all,
  index format=pgfsection,
}
%    \end{macrocode}
% Redefine \cs{tcb@doc@index@pgfsection} to add prologue (adapted from
% tcbdocumentation.code.tex).
%    \begin{macrocode}
\def\tcb@doc@index@pgfsection{%
  \def\index@prologue{%
    \section*{\kvtcb@text@index}\addcontentsline{toc}{section}{\kvtcb@text@index}
    {\small%
      Entries listed in the categories ``commands", ``lengths", and
      ``internal macros" also include references to package implementation. }
    \bigskip\par\noindent%
  }%
  \tcb@doc@index@pgf@%
}
%    \end{macrocode}
% Unset |\lst@UM| before indexing, to deals with comma (avoid |\unhbox \voidb@x
% \kern \z@| in index file, see: \url{https://tex.stackexchange.com/a/510108}).
%    \begin{macrocode}
\newcommand{\indexmacro}[1]{%
  \begingroup
  \let\lst@UM\@empty%
  \index{{Commands!#1=\tcbIndexPrintComC {#1}}}%
  \endgroup
}
\newcommand{\indexmacrointernal}[1]{%
  \begingroup
  \let\lst@UM\@empty%
  \index{{Internal macros!#1=\tcbIndexPrintComC {#1}}}%
  \endgroup
}
\newcommand{\indexlength}[1]{%
  \begingroup
  \let\lst@UM\@empty%
  \index{{Lengths!#1=\tcbIndexPrintLenC {#1}}}%
  \endgroup
}
%    \end{macrocode}
%
% \subsection*{Redefinition of macrocode environment}
%
%    \begin{macrocode}
\let\macrocode\relax
\lstnewenvironment{macrocode}{%
  \lstset{%
    name=macrocode,
    firstnumber=auto,
    style       = lstovarcode,
    showlines   = false,
%    \end{macrocode}
% Commands and lengths identification and indexing
%
% Note: index works badly with inherited styles. It seems that "texcs" and
% "index" keys should be set together.
%    \begin{macrocode}
    texcs       = [2]{% commands
      DeclareOverArrowCommand, ProvideOverArrowCommand,%
      NewOverArrowCommand, RenewOverArrowCommand,%
      TestOverArrow, xjoinrel, smallermathstyle,%
      esvectvv, vv,%
      overrightarrow, overleftarrow, overleftrightarrow, overrightharpoonup,%
      overrightharpoondown, overleftharpoonup, overleftharpoondown, overbar,%
      underrightarrow, underleftarrow, underleftrightarrow, underrightharpoonup,%
      underrightharpoondown, underleftharpoonup, underleftharpoondown, underbar,%
      SetOverArrowsMethod, SetOverArrowsSubscriptCommand,%
    },
    index       = [2][texcs2],%
    indexstyle  = [2]\indexmacro,%
    texcs       = [3]{% lengths
      overarrowlength, overarrowthickness, overarrowsmallerthickness,%
    },
    index       = [3][texcs3],%
    indexstyle  = [3]\indexlength,%
%    \end{macrocode}
% Internal macros identification and indexing
%    \begin{macrocode}
    texcs       = [4]{% internal macros
      ovar@rightarrow, ovar@leftarrow, ifovar@option@oldarrows@,
      ifovar@option@esvect@, ifovar@option@tikz@, ifovar@option@pstricks@,
      ifovar@option@pstarrows@, ifovar@detectsubscripts@,
      ifovar@option@subother@, ifovar@option@subactive@, ifovar@option@debug@,
      ifovar@option@overrightarrow@, ifovar@option@underrightarrow@,
      ifovar@option@overleftarrow@, ifovar@option@underleftarrow@,
      ifovar@option@overleftrightarrow@, ifovar@option@underleftrightarrow@,
      ifovar@option@overrightharpoonup@, ifovar@option@underrightharpoonup@,
      ifovar@option@overrightharpoondown@, ifovar@option@underrightharpoondown@,
      ifovar@option@overleftharpoonup@, ifovar@option@underleftharpoonup@,
      ifovar@option@overleftharpoondown@, ifovar@option@underleftharpoondown@,
      ifovar@option@overbar@, ifovar@option@underbar@, ovar@subcmd, ovar@set,
      ovar@length@min, ovar@before@arrow, ovar@after@arrow, ovar@macro@stack,
      ovar@macro@arrow, ovar@shift@left, ovar@shift@right, ovar@stack@fill,
      ovar@stackunder@fill, ovar@stack@lens, ovar@stackunder@lens,
      ovar@macro@arrowfill, ovar@arrow@fill, ovar@arrow@start, ovar@trim@start,
      ovar@trim@middle, ovar@arrow@middle, ovar@trim@end, ovar@arrow@end,
      ovar@tikz@command, ovar@tikz@pathoptions, ovar@tikz@path,
      ovar@tikz@options, ovar@pstricks@geometry, ovar@pstricks@linethickness,
      ovar@pstricks@psset, ovar@pstricks@command, ovar@picture@geometry,
      ovar@picture@linethickness, ovar@picture@command, ovar@tempdim,
      ovar@set@arrowlength, ovar@set@arrowthickness,
      ovar@set@arrowthickness@UM@lua, ovar@rulethickness@fontdimen,
      ovar@rulethickness@family, ovar@stackover@@, ovar@stackunder@@,
      ovar@stackover@, ovar@stackunder@, ovar@stackover@fill,
      ovar@stackover@lens, ovar@cmdname@processor, ovar@set@common,
      ovar@starversion, ovar@tempbox, ovar@testmathstyles, ovar@row@teststyle,
      ovar@testkerning%
    },
    index       = [4][texcs4],%
    indexstyle  = [4]\indexmacrointernal,%
%    \end{macrocode}
% Other macros identification
%    \begin{macrocode}
    moretexcs   = [5]{% other macros
      Relbar, relbareda, relbaredd, rightarrow, Rightarrow, varleftarrow,
      varrightarrow, twoheadrightarrow,%
    },
%    \end{macrocode}
% Il would be nice to also index keys, but spaces are badly handled by
% \pkg{listings}.
%    \begin{macrocode}
%^^A    morekeywords = [2]{%
%^^A      overarrows, no stack macro hook, stack macro, detect subscripts, arrow
%^^A      macro, no arrow macro hook, min length, before arrow, after arrow, space
%^^A      before arrow, space after arrow, shift left, shift right, shift leftright,
%^^A      center arrow, left arrow, right arrow, arrow under, arrow under/noconfig,
%^^A      arrow under/autoconfig, fill macro, start, middle, end, trim start, trim
%^^A      middle, trim end, trim, no trimming, middle config, middle config/relbar,
%^^A      middle config/relbareda, middle config/harrowextender, middle config/auto,
%^^A      amsmath, amsmath/mimic, amsmath/strict, esvect, esvect/mimic,
%^^A      esvect/strict, tikz command, tikz options, path options, path, add path
%^^A      options, add tikz options, arrows, line thickness, thinner, pstricks
%^^A      command, psset, geometry, arrow, picture command,%
%^^A    },
%^^A    index       = [3][keywords2],%
%^^A    indexstyle  = [3]\lstindexmacro,%
  }
%    \end{macrocode}
% Save and restore the line number (not done by environments defined with
% |\lstnewenvironment|, see \url{https://tex.stackexchange.com/a/95048}):
%    \begin{macrocode}
  \csname\@lst @SetFirstNumber\endcsname
}{%
  \csname \@lst @SaveFirstNumber\endcsname
}
%    \end{macrocode}
%
% \subsection*{Redefinition of macro environment}
%
%    \begin{macrocode}
\DeclareDocumentEnvironment{macro}{m}{%
%    \end{macrocode}
% Extraction of macro name. Save/restore current \cs{escapechar} in \cs{count@}.
%    \begin{macrocode}
 \count@\escapechar \escapechar=-1
 \edef\macroname{\string#1}
 \escapechar\count@
 \marginpar{%
   \parbox{0.88\linewidth}{%
     \scriptsize\flushright\ifcsdef{r@com:\macroname}{\refCom*{\macroname}}{\cs{\macroname}}%
   }%
 }
}{% end of macro environment
}
%    \end{macrocode}
%
% \subsection*{Documentation styles}
%
% \subsubsection*{Tool for references}
%
%    \begin{macrocode}
\def\refDoc#1{\tcb@ref@doc{#1}}
%    \end{macrocode}
%
% \subsubsection*{Package name}
%
%    \begin{macrocode}
\newcommand{\pkg}[1]{\textsf{#1}}
%    \end{macrocode}
%
% \subsubsection*{Package options}
%
% Definition of docOption environment to document package options, like keys.
%    \begin{macrocode}
\newrobustcmd{\tcbIndexPrintOptionCA}[1]{% adapted from \tcbIndexPrintKeyCA
  \tcb@Index@Print@CA{#1}{ovar_options}{package option}%
}
\newrobustcmd{\tcbIndexPrintOptionC}[1]{% adapted from \tcbIndexPrintKeyCA
  \tcb@Index@Print@C{#1}{ovar_options}%
}
\DeclareDocumentEnvironment{docOption}{ O{} m }{%
  \begin{docKey*}[]
    [doc name={#2}, doc label={opt:#2}, color key=ovar_options, #1]
    {#2}{}{}%
    \kvtcb@index@command{%
      \kvtcb@doc@sortindex\idx@actual\tcbIndexPrintOptionCA{#2}%
    }%
    \kvtcb@index@command{%
      {Package options}\idx@level\kvtcb@doc@sortindex\idx@actual%
      \tcbIndexPrintOptionC{#2}%
    }%
}{%
  \end{docKey*}%
}
\def\refOpt{\refDoc{key:opt}}
%    \end{macrocode}
%
% \subsubsection*{Lengths}
%
% Definition of docLength environment to document lengths, like commands.
%    \begin{macrocode}
\let\tcbdocLength\docLength% save original definition
\DeclareDocumentEnvironment{docLength}{ O{} m }{%
  \begin{docCommand*}[doc name=#2, color command=ovar_lengths,#1]{#2}{}
    \protected@edef\@currentlabel{\cs{#2}}%
    \label{len:#2}%
    %% add length index entry (from \docLength original definition)
    \kvtcb@index@command{%
      \kvtcb@doc@sortindex\idx@actual\tcbIndexPrintLenCA{\kvtcb@doc@index}%
    }%
    \kvtcb@index@command{%
      \kvtcb@text@lengths\idx@level\kvtcb@doc@sortindex\idx@actual%
      \tcbIndexPrintLenC{\kvtcb@doc@index}%
    }%
}{%
  \end{docCommand*}%
}
\def\refLen{\refDoc{len}}
%    \end{macrocode}
%
%</docstyle>
%
% \fi ^^A end of \iffalse
%
% \Finale
%
\endinput
